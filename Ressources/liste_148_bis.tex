\documentclass{article}
\usepackage{geometry}
\geometry{a4paper, margin=1in}
\usepackage{tabularx}
\begin{document}
\noindent\begin{tabularx}{\textwidth}{|X|X|X|X|X|}
\hline
\rowcolor{headerbg} \textcolor{white}{\textbf{Nom français}} & \textcolor{white}{\textbf{Nom latin}} & \textcolor{white}{\textbf{Famille}} & \textcolor{white}{\textbf{Parties utilisées}} & \textcolor{white}{\textbf{Forme de préparation}}  \\ \hline
\vocref{https://fr.wikipedia.org/wiki/Acacia}{Acacia à gomme.} & Acacia senegal (L.) Willd. et autres espèces d’acacias d’origine africaine. & Fabaceae & Exsudation gommeuse = gomme arabique. & En l’étatEn poudreExtrait sec aqueux \\ \hline
\vocref{https://fr.wikipedia.org/wiki/Ache}{Ache des marais.} & Apium graveolens L. & Apiaceae & Souche radicante. & En l’étatEn poudre \\ \hline
\vocref{https://fr.wikipedia.org/wiki/Achillée}{Achillée millefeuille.Millefeuille.} & Achillea millefolium L. & Asteraceae & Sommité fleurie. & En l’état \\ \hline
\vocref{https://fr.wikipedia.org/wiki/Agar-agar.}{Agar-agar.} & Gelidium sp., Euchema sp., Gracilaria sp. & Rhodophyceae & Mucilage = gélose. & En l’étatEn poudre \\ \hline
\vocref{https://fr.wikipedia.org/wiki/Ail.}{Ail.} & Allium sativum L. & Liliaceae & Bulbe. & En l’étatEn poudre \\ \hline
\vocref{https://fr.wikipedia.org/wiki/Airelle}{Airelle myrtille.} & Voir : Myrtille. &  &  &  \\ \hline
\vocref{https://fr.wikipedia.org/wiki/Ajowan.}{Ajowan.} & Carum copticum Benth. et Hook. f.(= Psychotis ajowan DC.). & Apiaceae & Fruit. & En l’étatEn poudre \\ \hline
\vocref{https://fr.wikipedia.org/wiki/Alchémille.}{Alchémille.} & Alchemilla vulgaris L. (sensu latiore). & Rosaceae & Partie aérienne. & En l’état \\ \hline
\vocref{https://fr.wikipedia.org/wiki/Alkékenge.coqueret.}{Alkékenge.Coqueret.} & Physalis alkekengi L. & Solanaceae & Fruit. & En l’état \\ \hline
\vocref{https://fr.wikipedia.org/wiki/Alliaire.}{Alliaire.} & Sisymbrium alliaria Scop. & Brassicaceae & Plante entière. & En l’étatEn poudre \\ \hline
\vocref{https://fr.wikipedia.org/wiki/Aloès}{Aloès des Barbades.} & Aloe barbadensis Mill.(= Aloe vera L.). & Liliaceae & Mucilage. & En l’étatEn poudre \\ \hline
\vocref{https://fr.wikipedia.org/wiki/Amandier}{Amandier doux.} & Prunus dulcis (Mill.) D. Webb var. dulcis. & Rosaceae & Graine, graine mondée. & En l’étatEn poudre \\ \hline
\vocref{https://fr.wikipedia.org/wiki/Ambrette.}{Ambrette.} & Hibiscus abelmoschus L. & Malvaceae & Graine. & En l’étatEn poudre \\ \hline
\vocref{https://fr.wikipedia.org/wiki/Aneth.}{Aneth.} & Anethum graveolens L.(= Peucedanum graveolens Benth. et Hook.). & Apiaceae & Fruit. & En l’état.En poudre \\ \hline
\end{tabularx}
\newpage
\noindent\begin{tabularx}{\textwidth}{|X|X|X|X|X|}
\hline
\rowcolor{headerbg} \textcolor{white}{\textbf{Nom français}} & \textcolor{white}{\textbf{Nom latin}} & \textcolor{white}{\textbf{Famille}} & \textcolor{white}{\textbf{Parties utilisées}} & \textcolor{white}{\textbf{Forme de préparation}}  \\ \hline
\vocref{https://fr.wikipedia.org/wiki/Aneth}{Aneth fenouil.} & Voir : Fenouil doux. &  &  &  \\ \hline
\vocref{https://fr.wikipedia.org/wiki/Angélique.angélique}{Angélique.Angélique officinale.} & Angelica archangelica L.(= Archangelica officinalis Hoffm.). & Apiaceae & Fruit. & En l’étatEn poudre \\ \hline
\vocref{https://fr.wikipedia.org/wiki/Anis.anis}{Anis.Anis vert.} & Pimpinella anisum L. & Apiaceae & Fruit. & En l’étatEn poudre \\ \hline
\vocref{https://fr.wikipedia.org/wiki/Anis}{Anis étoilé.} & Voir : Badianier de Chine. &  &  &  \\ \hline
\vocref{https://fr.wikipedia.org/wiki/Ascophyllum.}{Ascophyllum.} & Ascophyllum nodosum Le Jol. & Phaeophyceae & Thalle. & En l’étatEn poudreExtrait sec aqueux \\ \hline
\vocref{https://fr.wikipedia.org/wiki/Aspérule}{Aspérule odorante.} & Galium odoratum (L.) Scop.(= Asperula odorata L.). & Rubiaceae & Partie aérienne fleurie. & En l’état \\ \hline
\vocref{https://fr.wikipedia.org/wiki/Aspic.lavande}{Aspic.Lavande aspic.} & Lavandula latifolia (L. f.) Medik. & Lamiaceae & Sommité fleurie. & En l’état \\ \hline
\vocref{https://fr.wikipedia.org/wiki/Astragale}{Astragale à gomme.Gomme adragante.} & Astragalus gummifer (Labill.) et certaines espèces du genre Astragalus d’Asie occidentale. & Fabaceae & Exsudation gommeuse = gomme adragante. & En l’étatEn poudreExtrait sec aqueux \\ \hline
\vocref{https://fr.wikipedia.org/wiki/Aubépine.epine}{Aubépine.Epine blanche.} & Crataegus laevigata (Poir.) DC.,C. monogyna Jacq. (Lindm.)(= C. oxyacanthoïdes Thuill.). & Rosaceae & Fruit. & En l’état \\ \hline
\vocref{https://fr.wikipedia.org/wiki/Aunée.aunée}{Aunée.Aunée officinale.} & Inula helenium L. & Asteraceae & Partie souterraine. & En l’étatEn poudre \\ \hline
\vocref{https://fr.wikipedia.org/wiki/Avoine.}{Avoine.} & Avena sativa L. & Poaceae & Fruit. & En l’étatEn poudre \\ \hline
\end{tabularx}
\newpage
\noindent\begin{tabularx}{\textwidth}{|X|X|X|X|X|}
\hline
\rowcolor{headerbg} \textcolor{white}{\textbf{Nom français}} & \textcolor{white}{\textbf{Nom latin}} & \textcolor{white}{\textbf{Famille}} & \textcolor{white}{\textbf{Parties utilisées}} & \textcolor{white}{\textbf{Forme de préparation}}  \\ \hline
\vocref{https://fr.wikipedia.org/wiki/Balsamite}{Balsamite odorante.Menthe coq.} & Balsamita major Desf.(= Chrysanthemum balsamita [L.] Baill.). & Asteraceae & Feuille, sommité fleurie. & En l’état \\ \hline
\vocref{https://fr.wikipedia.org/wiki/Bardane}{Bardane (grande).} & Arctium lappa L.(= A. majus [Gaertn.] Bernh.)(= Lappa major Gaertn.). & Asteraceae & Feuille, racine. & En l’état \\ \hline
\vocref{https://fr.wikipedia.org/wiki/Basilic.basilic}{Basilic.Basilic doux.} & Ocimum basilicum L. & Lamiaceae & Feuille. & En l’étatEn poudre \\ \hline
\vocref{https://fr.wikipedia.org/wiki/Baumier}{Baumier de Copahu.Baume de Copahu.} & Copaifera officinalis L.,C. guyanensis Desf.,C. lansdorfii Desf. & Fabaceae & Oléo-résine dite baume de copahu » . & En l’état \\ \hline
\vocref{https://fr.wikipedia.org/wiki/Bétoine.}{Bétoine.} & Stachys officinalis (L.) Trevis.(= Betonica officinalis L.). & Lamiaceae & Feuille. & En l’état \\ \hline
\vocref{https://fr.wikipedia.org/wiki/Bigaradier.}{Bigaradier.} & Voir : Oranger amer. &  &  &  \\ \hline
\vocref{https://fr.wikipedia.org/wiki/Blé.}{Blé.} & Triticum aestivum L. et cultivars(= T. vulgare Host)(= T. sativum Lam.). & Poaceae & Son. & En l’étatEn poudre \\ \hline
\vocref{https://fr.wikipedia.org/wiki/Bouillon}{Bouillon blanc.} & Verbascum thapsus L.,V. densiflorum Bertol.(= V. thapsiforme Schrad.),V. phlomoides L. & Scrophulariaceae & Corolle mondée. & En l’état \\ \hline
\vocref{https://fr.wikipedia.org/wiki/Bourrache.}{Bourrache.} & Borago officinalis L. & Boraginaceae & Fleur. & En l’état \\ \hline
\vocref{https://fr.wikipedia.org/wiki/Bruyère}{Bruyère cendrée.} & Erica cinerea L. & Ericaceae & Fleur. & En l’état \\ \hline
\vocref{https://fr.wikipedia.org/wiki/Camomille}{Camomille allemande.} & Voir : Matricaire. &  &  &  \\ \hline
\vocref{https://fr.wikipedia.org/wiki/Camomille}{Camomille romaine.} & Chamaemelum nobile (L.) All.(= Anthemis nobilis L.). & Asteraceae & Capitule. & En l’état \\ \hline
\vocref{https://fr.wikipedia.org/wiki/Camomille}{Camomille vulgaire.} & Voir : Matricaire. &  &  &  \\ \hline
\vocref{https://fr.wikipedia.org/wiki/Canéficier.}{Canéficier.} & Cassia fistula L. & Fabaceae & Pulpe de fruit. & En l’état \\ \hline
\end{tabularx}
\newpage
\noindent\begin{tabularx}{\textwidth}{|X|X|X|X|X|}
\hline
\rowcolor{headerbg} \textcolor{white}{\textbf{Nom français}} & \textcolor{white}{\textbf{Nom latin}} & \textcolor{white}{\textbf{Famille}} & \textcolor{white}{\textbf{Parties utilisées}} & \textcolor{white}{\textbf{Forme de préparation}}  \\ \hline
\vocref{https://fr.wikipedia.org/wiki/Cannelier}{Cannelier de Ceylan.Cannelle de Ceylan.} & Cinnamomum zeylanicum Nees. & Lauraceae & Ecorce de tige raclée = cannelle de Ceylan. & En l’étatEn poudre \\ \hline
\vocref{https://fr.wikipedia.org/wiki/Cannelier}{Cannelier de Chine.Cannelle de Chine.} & Cinnamomum aromaticum Nees,C. cassia Nees ex Blume. & Lauraceae & Ecorce de tige = cannelle de Chine. & En l’étatEn poudre \\ \hline
\vocref{https://fr.wikipedia.org/wiki/Capucine.}{Capucine.} & Tropaeolum majus L. & Tropaeolaceae & Feuille. & En l’état \\ \hline
\vocref{https://fr.wikipedia.org/wiki/Cardamome.}{Cardamome.} & Elettaria cardamomum (L.) Maton. & Zingiberaceae & Fruit. & En l’étatEn poudre \\ \hline
\vocref{https://fr.wikipedia.org/wiki/Caroubier.gomme}{Caroubier.Gomme caroube.} & Ceratonia siliqua L. & Fabaceae & Graine mondée = gomme caroube. & En l’étatEn poudre \\ \hline
\vocref{https://fr.wikipedia.org/wiki/Carragaheen.mousse}{Carragaheen.Mousse d’Irlande.} & Chondrus crispus Lingby. & Gigartinaceae & Thalle. & En l’état \\ \hline
\vocref{https://fr.wikipedia.org/wiki/Carthame.}{Carthame.} & Carthamus tinctorius L. & Asteraceae & Fleur. & En l’état \\ \hline
\vocref{https://fr.wikipedia.org/wiki/Carvi.cumin}{Carvi.Cumin des prés.} & Carum carvi L. & Apiaceae & Fruit. & En l’étatEn poudre \\ \hline
\vocref{https://fr.wikipedia.org/wiki/Cassissier.groseiller}{Cassissier.Groseiller noir.} & Ribes nigrum L. & Grossulariaceae & Feuille, fruit. & En l’état \\ \hline
\vocref{https://fr.wikipedia.org/wiki/Centaurée}{Centaurée (petite).} & Centaurium erythraea Raf.(= Erythraea centaurium [L.] Persoon)(= C. minus Moench)(= C. umbellatum Gilib.). & Gentianaceae & Sommité fleurie. & En l’état \\ \hline
\vocref{https://fr.wikipedia.org/wiki/Cerisier}{Cerisier griottier.} & Voir : Griottier. &  &  &  \\ \hline
\vocref{https://fr.wikipedia.org/wiki/Chicorée.}{Chicorée.} & Cichorium intybus L. & Asteraceae & Feuille, racine. & En l’état \\ \hline
\vocref{https://fr.wikipedia.org/wiki/Chiendent}{Chiendent (gros).Chiendent pied de poule.} & Cynodon dactylon (L.) Pers. & Poaceae & Rhizome. & En l’état \\ \hline
\vocref{https://fr.wikipedia.org/wiki/Chiendent.chiendent}{Chiendent.Chiendent (petit).} & Elytrigia repens [L.] Desv. ex Nevski(= Agropyron repens [L.] Beauv.)(= Elymus repens [L.] Goudl.). & Poaceae & Rhizome. & En l’état \\ \hline
\vocref{https://fr.wikipedia.org/wiki/Citronnelles.}{Citronnelles.} & Cymbopogon sp. & Poaceae & Feuille. & En l’étatEn poudre \\ \hline
\end{tabularx}
\newpage
\noindent\begin{tabularx}{\textwidth}{|X|X|X|X|X|}
\hline
\rowcolor{headerbg} \textcolor{white}{\textbf{Nom français}} & \textcolor{white}{\textbf{Nom latin}} & \textcolor{white}{\textbf{Famille}} & \textcolor{white}{\textbf{Parties utilisées}} & \textcolor{white}{\textbf{Forme de préparation}}  \\ \hline
\vocref{https://fr.wikipedia.org/wiki/Citrouille.}{Citrouille.} & Voir : Courge citrouille. &  &  &  \\ \hline
\vocref{https://fr.wikipedia.org/wiki/Clou}{Clou de girofle.} & Voir : Giroflier. &  &  &  \\ \hline
\vocref{https://fr.wikipedia.org/wiki/Cochléaire.}{Cochléaire.} & Cochlearia officinalis L. & Brassicaceae & Feuille. & En l’état \\ \hline
\vocref{https://fr.wikipedia.org/wiki/Colatier.}{Colatier.} & Voir : Kolatier. &  &  &  \\ \hline
\vocref{https://fr.wikipedia.org/wiki/Coquelicot.}{Coquelicot.} & Papaver rhoeas L.,P. dubium L. & Papaveraceae & Pétale. & En l’état \\ \hline
\vocref{https://fr.wikipedia.org/wiki/Coqueret.}{Coqueret.} & Voir : Alkékenge. &  &  &  \\ \hline
\vocref{https://fr.wikipedia.org/wiki/Coriandre.}{Coriandre.} & Coriandrum sativum L. & Apiaceae & Fruit. & En l’étatEn poudre \\ \hline
\vocref{https://fr.wikipedia.org/wiki/Courge}{Courge citrouille.Citrouille.} & Cucurbita pepo L.. & Cucurbitaceae & Graine. & En l’état \\ \hline
\vocref{https://fr.wikipedia.org/wiki/Courge.potiron.}{Courge.Potiron.} & Cucurbita maxima Lam. & Cucurbitaceae & Graine. & En l’état \\ \hline
\vocref{https://fr.wikipedia.org/wiki/Criste}{Criste marine.Perce-pierre.} & Crithmum maritimum L.. & Apiaceae & Partie aérienne. & En l’état \\ \hline
\vocref{https://fr.wikipedia.org/wiki/Cumin}{Cumin des prés.} & Voir : Carvi. &  &  &  \\ \hline
\vocref{https://fr.wikipedia.org/wiki/Curcuma}{Curcuma long.} & Curcuma domestica Vahl(= C. longa L.). & Zingiberaceae & Rhizome. & En l’étatEn poudre \\ \hline
\vocref{https://fr.wikipedia.org/wiki/Cyamopsis.gomme}{Cyamopsis.Gomme guar.Guar.} & Cyamopsis tetragonolobus (L.) Taub. & Fabaceae & Graine mondée = gomme guar. & En l’étatEn poudreExtrait sec aqueux \\ \hline
\vocref{https://fr.wikipedia.org/wiki/Eglantier.cynorrhodon.rosier}{Eglantier.Cynorrhodon.Rosier sauvage.} & Rosa canina L., R. pendulina L. et autres espèces de Rosa. & Rosaceae & Pseudo-fruit = cynorrhodon. & En l’état \\ \hline
\vocref{https://fr.wikipedia.org/wiki/Eleuthérocoque.}{Eleuthérocoque.} & Eleutherococcus senticosus Maxim. & Araliaceae & Partie souterraine. & En l’état \\ \hline
\end{tabularx}
\newpage
\noindent\begin{tabularx}{\textwidth}{|X|X|X|X|X|}
\hline
\rowcolor{headerbg} \textcolor{white}{\textbf{Nom français}} & \textcolor{white}{\textbf{Nom latin}} & \textcolor{white}{\textbf{Famille}} & \textcolor{white}{\textbf{Parties utilisées}} & \textcolor{white}{\textbf{Forme de préparation}}  \\ \hline
\vocref{https://fr.wikipedia.org/wiki/Estragon.}{Estragon.} & Artemisia dracunculus L. & Asteraceae & Partie aérienne. & En l’étatEn poudre \\ \hline
\vocref{https://fr.wikipedia.org/wiki/Eucalyptus.eucalyptus}{Eucalyptus.Eucalyptus globuleux.} & Eucalyptus globulus Labill. & Myrtaceae & Feuille. & En l’état \\ \hline
\vocref{https://fr.wikipedia.org/wiki/Fenouil}{Fenouil amer.} & Foeniculum vulgare Mill. var. vulgare. & Apiaceae & Fruit. & En l’étatEn poudre \\ \hline
\vocref{https://fr.wikipedia.org/wiki/Fenouil}{Fenouil doux.Aneth fenouil.} & Foeniculum vulgare Mill. var. dulcis. & Apiaceae & Fruit. & En l’étatEn poudre \\ \hline
\vocref{https://fr.wikipedia.org/wiki/Fenugrec.}{Fenugrec.} & Trigonella foenum-graecum L. & Fabaceae & Graine. & En l’étatEn poudre \\ \hline
\vocref{https://fr.wikipedia.org/wiki/Févier.}{Févier.} & Voir : Gléditschia. &  &  &  \\ \hline
\vocref{https://fr.wikipedia.org/wiki/Figuier.}{Figuier.} & Ficus carica L. & Moraceae & Pseudo-fruit. & En l’état \\ \hline
\vocref{https://fr.wikipedia.org/wiki/Frêne.}{Frêne.} & Fraxinus excelsior L.,F. oxyphylla M. Bieb. & Oleaceae & Feuille. & En l’état \\ \hline
\vocref{https://fr.wikipedia.org/wiki/Frêne}{Frêne à manne.} & Fraxinus ornus L. & Oleaceae & Suc épaissi dit manne ». & En l’étatEn poudre \\ \hline
\vocref{https://fr.wikipedia.org/wiki/Fucus.}{Fucus.} & Fucus serratus L.,F. vesiculosus L. & Fucaceae & Thalle. & En l’étatEn poudre \\ \hline
\end{tabularx}
\newpage
\noindent\begin{tabularx}{\textwidth}{|X|X|X|X|X|}
\hline
\rowcolor{headerbg} \textcolor{white}{\textbf{Nom français}} & \textcolor{white}{\textbf{Nom latin}} & \textcolor{white}{\textbf{Famille}} & \textcolor{white}{\textbf{Parties utilisées}} & \textcolor{white}{\textbf{Forme de préparation}}  \\ \hline
\vocref{https://fr.wikipedia.org/wiki/Galanga}{Galanga (petit).} & Alpinia officinarum Hance. & Zingiberaceae & Rhizome. & En l’étatEn poudre \\ \hline
\vocref{https://fr.wikipedia.org/wiki/Genévrier.genièvre.}{Genévrier.Genièvre.} & Juniperus communis L. & Cupressaceae & Cône femelle dit baie de genièvre ». & En l’état \\ \hline
\vocref{https://fr.wikipedia.org/wiki/Gentiane.gentiane}{Gentiane.Gentiane jaune.} & Gentiana lutea L. & Gentianaceae & Partie souterraine. & En l’étatEn poudre \\ \hline
\vocref{https://fr.wikipedia.org/wiki/Gingembre.}{Gingembre.} & Zingiber officinale Roscoe. & Zingiberaceae & Rhizome. & En l’étatEn poudre \\ \hline
\vocref{https://fr.wikipedia.org/wiki/Ginseng.panax}{Ginseng.Panax de Chine.} & Panax ginseng C.A. Meyer(= Aralia quinquefolia Decne. et Planch.). & Araliaceae & Partie souterraine. & En l’étatEn poudreExtrait sec aqueux \\ \hline
\vocref{https://fr.wikipedia.org/wiki/Giroflier.}{Giroflier.} & Syzygium aromaticum (L.) Merr. et Perry(= Eugenia caryophyllus (Sprengel) Bull. et Harr.). & Myrtaceae & Bouton floral = clou de girofle. & En l’étatEn poudre \\ \hline
\vocref{https://fr.wikipedia.org/wiki/Gléditschia.févier.}{Gléditschia.Févier.} & Gleditschia triacanthos L.,G. ferox Desf. & Fabaceae & Graine. & En l’étatEn poudreExtrait sec aqueux \\ \hline
\vocref{https://fr.wikipedia.org/wiki/Gomme}{Gomme adragante.} & Voir : Astragale à gomme. &  &  &  \\ \hline
\vocref{https://fr.wikipedia.org/wiki/Gomme}{Gomme arabique.} & Voir : Acacia à gomme. &  &  &  \\ \hline
\vocref{https://fr.wikipedia.org/wiki/Gomme}{Gomme caroube.} & Voir : Caroubier. &  &  &  \\ \hline
\vocref{https://fr.wikipedia.org/wiki/Gomme}{Gomme de sterculia.} & Voir : Sterculia. &  &  &  \\ \hline
\vocref{https://fr.wikipedia.org/wiki/Gomme}{Gomme guar.} & Voir : Cyamopsis. &  &  &  \\ \hline
\vocref{https://fr.wikipedia.org/wiki/Gomme}{Gomme Karaya.} & Voir : Sterculia. &  &  &  \\ \hline
\vocref{https://fr.wikipedia.org/wiki/Gomme}{Gomme M’Bep.} & Voir : Sterculia. &  &  &  \\ \hline
\end{tabularx}
\newpage
\noindent\begin{tabularx}{\textwidth}{|X|X|X|X|X|}
\hline
\rowcolor{headerbg} \textcolor{white}{\textbf{Nom français}} & \textcolor{white}{\textbf{Nom latin}} & \textcolor{white}{\textbf{Famille}} & \textcolor{white}{\textbf{Parties utilisées}} & \textcolor{white}{\textbf{Forme de préparation}}  \\ \hline
\vocref{https://fr.wikipedia.org/wiki/Griottier.cerisier}{Griottier.Cerisier griottier.Queue de cerise.} & Prunus cerasus L.,P. avium (L.) L. & Rosaceae & Pédoncule du fruit = queue de cerise. & En l’état \\ \hline
\vocref{https://fr.wikipedia.org/wiki/Groseiller}{Groseiller noir.} & Voir : Cassissier. &  &  &  \\ \hline
\vocref{https://fr.wikipedia.org/wiki/Guar.}{Guar.} & Voir : Cyamopsis. &  &  &  \\ \hline
\vocref{https://fr.wikipedia.org/wiki/Guarana.}{Guarana.} & Voir : Paullinia. &  &  &  \\ \hline
\vocref{https://fr.wikipedia.org/wiki/Guimauve.}{Guimauve.} & Althaea officinalis L. & Malvaceae & Feuille, fleur, racine. & En l’étatEn poudre (racine) \\ \hline
\vocref{https://fr.wikipedia.org/wiki/Hibiscus.}{Hibiscus.} & Voir : Karkadé. &  &  &  \\ \hline
\vocref{https://fr.wikipedia.org/wiki/Houblon.}{Houblon.} & Humulus lupulus L. & Cannabaceae & Inflorescence femelle dite cône de houblon ». & En l’état \\ \hline
\vocref{https://fr.wikipedia.org/wiki/Jujubier.}{Jujubier.} & Ziziphus jujuba Mill.(= Z. sativa Gaertn.)(= Z. vulgaris Lam.)(= Rhamnus zizyphus L.). & Rhamnaceae & Fruit privé de graines. & En l’état \\ \hline
\vocref{https://fr.wikipedia.org/wiki/Karkadé.oseille}{Karkadé.Oseille de Guinée.Hibiscus.} & Hibiscus sabdariffa L. & Malvaceae & Calice et calicule. & En l’état \\ \hline
\vocref{https://fr.wikipedia.org/wiki/Kolatier.colatier.kola.}{Kolatier.Colatier.Kola.} & Cola acuminata (P. Beauv.) Schott et Endl.(= Sterculia acuminata P. Beauv.),C. nitida (Vent.) Schott et Endl.(= C. vera K. Schum.) et variétés. & Sterculiaceae & Amande dite noix de kola ». & En l’étatEn poudre \\ \hline
\vocref{https://fr.wikipedia.org/wiki/Lamier}{Lamier blanc.Ortie blanche.} & Lamium album L. & Lamiaceae & Corolle mondée, sommité fleurie. & En l’état \\ \hline
\vocref{https://fr.wikipedia.org/wiki/Laminaire.}{Laminaire.} & Laminaria digitata J.P. Lamour.,L. hyperborea (Gunnerus) Foslie,L. cloustonii Le Jol. & Laminariaceae & Stipe, thalle. & En l’étatExtrait sec aqueux (thalle) \\ \hline
\vocref{https://fr.wikipedia.org/wiki/Laurier}{Laurier commun.Laurier sauce.} & Laurus nobilis L. & Lauraceae & Feuille. & En l’étatEn poudre \\ \hline
\vocref{https://fr.wikipedia.org/wiki/Lavande.lavande}{Lavande.Lavande vraie.} & Lavandula angustifolia Mill.(= L. vera DC.). & Lamiaceae & Fleur, sommité fleurie. & En l’état \\ \hline
\vocref{https://fr.wikipedia.org/wiki/Lavande}{Lavande aspic.} & Voir : Aspic. &  &  &  \\ \hline
\end{tabularx}
\newpage
\noindent\begin{tabularx}{\textwidth}{|X|X|X|X|X|}
\hline
\rowcolor{headerbg} \textcolor{white}{\textbf{Nom français}} & \textcolor{white}{\textbf{Nom latin}} & \textcolor{white}{\textbf{Famille}} & \textcolor{white}{\textbf{Parties utilisées}} & \textcolor{white}{\textbf{Forme de préparation}}  \\ \hline
\vocref{https://fr.wikipedia.org/wiki/Lavande}{Lavande stoechas.} & Lavandula stoechas L. & Lamiaceae & Fleur, sommité fleurie. & En l’état \\ \hline
\vocref{https://fr.wikipedia.org/wiki/Lavande}{Lavande vraie.} & Voir : Lavande. &  &  &  \\ \hline
\vocref{https://fr.wikipedia.org/wiki/Lavandin}{Lavandin Grosso ».} & Lavandula × intermedia Emeric ex Loisel. & Lamiaceae & Fleur, sommité fleurie. & En l’état \\ \hline
\vocref{https://fr.wikipedia.org/wiki/Lemongrass}{Lemongrass de l’Amérique centrale.} & Cymbopogon citratus (DC.) Stapf. & Poaceae & Feuille. & En l’étatEn poudre \\ \hline
\vocref{https://fr.wikipedia.org/wiki/Lemongrass}{Lemongrass de l’Inde.} & Cymbopogon flexuosus (Nees ex Steud.) J.F. Wats. & Poaceae & Feuille. & En l’étatEn poudre \\ \hline
\vocref{https://fr.wikipedia.org/wiki/Lichen}{Lichen d’Islande.} & Cetraria islandica (L.) Ach. sensu latiore. & Parmeliaceae & Thalle. & En l’état \\ \hline
\vocref{https://fr.wikipedia.org/wiki/Lierre}{Lierre terrestre.} & Glechoma hederacea L.(= Nepeta glechoma Benth.). & Lamiaceae & Partie aérienne fleurie. & En l’état \\ \hline
\vocref{https://fr.wikipedia.org/wiki/Lin.}{Lin.} & Linum usitatissimum L. & Linaceae & Graine. & En l’étatEn poudre \\ \hline
\vocref{https://fr.wikipedia.org/wiki/Livèche.}{Livèche.} & Levisticum officinale Koch. & Apiaceae & Feuille, fruit, partie souterraine. & En l’étatEn poudre \\ \hline
\end{tabularx}
\newpage
\noindent\begin{tabularx}{\textwidth}{|X|X|X|X|X|}
\hline
\rowcolor{headerbg} \textcolor{white}{\textbf{Nom français}} & \textcolor{white}{\textbf{Nom latin}} & \textcolor{white}{\textbf{Famille}} & \textcolor{white}{\textbf{Parties utilisées}} & \textcolor{white}{\textbf{Forme de préparation}}  \\ \hline
\vocref{https://fr.wikipedia.org/wiki/Marjolaine.origan}{Marjolaine.Origan marjolaine.} & Origanum majorana L.(= Majorana hortensis Moench). & Lamiaceae & Feuille, sommité fleurie. & En l’étatEn poudre \\ \hline
\vocref{https://fr.wikipedia.org/wiki/Maté.thé}{Maté.Thé du Paraguay.} & Ilex paraguariensis St.-Hil.(= I. paraguayensis Lamb.). & Aquifoliaceae & Feuille. & En l’étatExtrait sec aqueux \\ \hline
\vocref{https://fr.wikipedia.org/wiki/Matricaire.camomille}{Matricaire.Camomille allemande.Camomille vulgaire.} & Matricaria recutita L.(= Chamomilla recutita [L.] Rausch.)(= M. chamomilla L.). & Asteraceae & Capitule. & En l’état \\ \hline
\vocref{https://fr.wikipedia.org/wiki/Mauve.}{Mauve.} & Malva sylvestris L. & Malvaceae & Feuille, fleur. & En l’état \\ \hline
\vocref{https://fr.wikipedia.org/wiki/Mélisse.}{Mélisse.} & Melissa officinalis L. & Lamiaceae & Feuille, sommité fleurie. & En l’état \\ \hline
\vocref{https://fr.wikipedia.org/wiki/Menthe}{Menthe coq.} & Voir : Balsamite odorante. &  &  &  \\ \hline
\vocref{https://fr.wikipedia.org/wiki/Menthe}{Menthe poivrée.} & Mentha × piperita L. & Lamiaceae & Feuille, sommité fleurie. & En l’état \\ \hline
\vocref{https://fr.wikipedia.org/wiki/Menthe}{Menthe verte.} & Mentha spicata L. (= M. viridis L.). & Lamiaceae & Feuille, sommité fleurie. & En l’état \\ \hline
\vocref{https://fr.wikipedia.org/wiki/Ményanthe.trèfle}{Ményanthe.Trèfle d’eau.} & Menyanthes trifoliata L. & Menyanthaceae & Feuille. & En l’état \\ \hline
\vocref{https://fr.wikipedia.org/wiki/Millefeuille.}{Millefeuille.} & Voir : Achillée millefeuille. &  &  &  \\ \hline
\vocref{https://fr.wikipedia.org/wiki/Mousse}{Mousse d’Irlande.} & Voir : Carragaheen. &  &  &  \\ \hline
\vocref{https://fr.wikipedia.org/wiki/Moutarde}{Moutarde junciforme.} & Brassica juncea (L.) Czern. & Brassicaceae & Graine. & En l’étatEn poudre \\ \hline
\vocref{https://fr.wikipedia.org/wiki/Muscadier}{Muscadier aromatique.Macis.Muscade.} & Myristica fragrans Houtt.(= M. moschata Thunb.). & Myristicaceae & Graine dite muscade » ou noix de muscade », arille dite macis ». & En l’étatEn poudre (graine) \\ \hline
\vocref{https://fr.wikipedia.org/wiki/Myrte.}{Myrte.} & Myrtus communis L. & Myrtaceae & Feuille. & En l’état \\ \hline
\end{tabularx}
\newpage
\noindent\begin{tabularx}{\textwidth}{|X|X|X|X|X|}
\hline
\rowcolor{headerbg} \textcolor{white}{\textbf{Nom français}} & \textcolor{white}{\textbf{Nom latin}} & \textcolor{white}{\textbf{Famille}} & \textcolor{white}{\textbf{Parties utilisées}} & \textcolor{white}{\textbf{Forme de préparation}}  \\ \hline
\vocref{https://fr.wikipedia.org/wiki/Myrtille.airelle}{Myrtille.Airelle myrtille.} & Vaccinium myrtillus L. & Ericaceae & Feuille, fruit. & En l’état \\ \hline
\vocref{https://fr.wikipedia.org/wiki/Olivier.}{Olivier.} & Olea europaea L. & Oleaceae & Feuille. & En l’état \\ \hline
\vocref{https://fr.wikipedia.org/wiki/Oranger}{Oranger amer.Bigaradier.} & Citrus aurantium L.(= C. bigaradia Duch.)(= C. vulgaris Risso). & Rutaceae & Feuille, fleur, péricarpe dit écorce » ou zeste. & En l’étatEn poudre (péricarpe) \\ \hline
\vocref{https://fr.wikipedia.org/wiki/Oranger}{Oranger doux.} & Citrus sinensis (L.) Pers.(= C. aurantium L.). & Rutaceae & Péricarpe dit écorce » ou zeste. & En l’étatEn poudre \\ \hline
\vocref{https://fr.wikipedia.org/wiki/Origan.}{Origan.} & Origanum vulgare L. & Lamiaceae & Feuille, sommité fleurie. & En l’étatEn poudre \\ \hline
\vocref{https://fr.wikipedia.org/wiki/Origan}{Origan marjolaine.} & Voir : Marjolaine. &  &  &  \\ \hline
\vocref{https://fr.wikipedia.org/wiki/Ortie}{Ortie blanche.} & Voir : Lamier blanc. &  &  &  \\ \hline
\vocref{https://fr.wikipedia.org/wiki/Ortie}{Ortie brûlante.} & Urtica urens L. & Urticaceae & Partie aérienne. & En l’état \\ \hline
\vocref{https://fr.wikipedia.org/wiki/Ortie}{Ortie dioïque.} & Urtica dioica L. & Urticaceae & Partie aérienne. & En l’état \\ \hline
\vocref{https://fr.wikipedia.org/wiki/Oseille}{Oseille de Guinée} & Voir : Karkadé. &  &  &  \\ \hline
\vocref{https://fr.wikipedia.org/wiki/Panax}{Panax de Chine} & Voir : Ginseng. &  &  &  \\ \hline
\vocref{https://fr.wikipedia.org/wiki/Papayer.}{Papayer.} & Carica papaya L. & Caricaceae & Suc du fruit, feuille. & En l’étatEn poudre (suc du fruit) \\ \hline
\vocref{https://fr.wikipedia.org/wiki/Passerose.}{Passerose.} & Voir : Rose trémière. &  &  &  \\ \hline
\vocref{https://fr.wikipedia.org/wiki/Paullinia.guarana.}{Paullinia.Guarana.} & Paullinia cupana Kunth.(= P. sorbilis Mart.). & Sapindaceae & Graine, extrait préparé avec la graine = guarana. & En l’étatEn poudre (extrait) \\ \hline
\vocref{https://fr.wikipedia.org/wiki/Pensée}{Pensée sauvage.Violette tricolore.} & Viola arvensis Murray,V. tricolor L. & Violaceae & Fleur, partie aérienne fleurie. & En l’état \\ \hline
\end{tabularx}
\newpage
\noindent\begin{tabularx}{\textwidth}{|X|X|X|X|X|}
\hline
\rowcolor{headerbg} \textcolor{white}{\textbf{Nom français}} & \textcolor{white}{\textbf{Nom latin}} & \textcolor{white}{\textbf{Famille}} & \textcolor{white}{\textbf{Parties utilisées}} & \textcolor{white}{\textbf{Forme de préparation}}  \\ \hline
\vocref{https://fr.wikipedia.org/wiki/Perce-pierre.}{Perce-pierre.} & Voir : Criste marine. &  &  &  \\ \hline
\vocref{https://fr.wikipedia.org/wiki/Piment}{Piment de Cayenne.Piment enragé.Piment (petit).} & Capsicum frutescens L. & Solanaceae & Fruit. & En l’étatEn poudre \\ \hline
\vocref{https://fr.wikipedia.org/wiki/Pin}{Pin sylvestre.} & Pinus sylvestris L. & Pinaceae & Bourgeon. & En l’état \\ \hline
\vocref{https://fr.wikipedia.org/wiki/Pissenlit.dent}{Pissenlit.Dent de lion.} & Taraxacum officinale Web. & Asteraceae & Feuille, partie aérienne. & En l’état \\ \hline
\vocref{https://fr.wikipedia.org/wiki/Pommier.}{Pommier.} & Malus sylvestris Mill.(= Pyrus malus L.). & Rosaceae & Fruit. & En l’état \\ \hline
\vocref{https://fr.wikipedia.org/wiki/Potiron.}{Potiron.} & Voir : Courge. &  &  &  \\ \hline
\vocref{https://fr.wikipedia.org/wiki/Prunier.}{Prunier.} & Prunus domestica L. & Rosaceae & Fruit. & En l’état \\ \hline
\vocref{https://fr.wikipedia.org/wiki/Queue}{Queue de cerise.} & Voir : Griottier. &  &  &  \\ \hline
\vocref{https://fr.wikipedia.org/wiki/Radis}{Radis noir.} & Raphanus sativus L. var. niger (Mill.) Kerner. & Brassicaceae & Racine. & En l’état \\ \hline
\vocref{https://fr.wikipedia.org/wiki/Raifort}{Raifort sauvage.} & Armoracia rusticana Gaertn., B. Mey. et Scherb.(= Cochlearia armoracia L.). & Brassicaceae & Racine. & En l’étatEn poudre \\ \hline
\vocref{https://fr.wikipedia.org/wiki/Réglisse.}{Réglisse.} & Glycyrrhiza glabra L. & Fabaceae & Partie souterraine. & En l’étatEn poudreExtrait sec aqueux \\ \hline
\vocref{https://fr.wikipedia.org/wiki/Reine-des-prés.ulmaire.}{Reine-des-prés.Ulmaire.} & Filipendula ulmaria (L.) Maxim.(= Spiraea ulmaria L.). & Rosaceae & Fleur, sommité fleurie. & En l’état \\ \hline
\vocref{https://fr.wikipedia.org/wiki/Romarin.}{Romarin.} & Rosmarinus officinalis L. & Lamiaceae & Feuille, sommité fleurie. & En l’étatEn poudre \\ \hline
\vocref{https://fr.wikipedia.org/wiki/Ronce.}{Ronce.} & Rubus sp. & Rosaceae & Feuille. & En l’état \\ \hline
\vocref{https://fr.wikipedia.org/wiki/Rose}{Rose trémière.Passerose.} & Alcea rosea L.(= Althaea rosea L.). & Malvaceae & Fleur. & En l’état \\ \hline
\end{tabularx}
\newpage
\noindent\begin{tabularx}{\textwidth}{|X|X|X|X|X|}
\hline
\rowcolor{headerbg} \textcolor{white}{\textbf{Nom français}} & \textcolor{white}{\textbf{Nom latin}} & \textcolor{white}{\textbf{Famille}} & \textcolor{white}{\textbf{Parties utilisées}} & \textcolor{white}{\textbf{Forme de préparation}}  \\ \hline
\vocref{https://fr.wikipedia.org/wiki/Rosier}{Rosier à roses pâles.} & Rosa centifolia L. & Rosaceae & Bouton floral, pétale. & En l’état \\ \hline
\vocref{https://fr.wikipedia.org/wiki/Rosier}{Rosier de Damas.} & Rosa damascena Mill. & Rosaceae & Bouton floral, pétale. & En l’état \\ \hline
\vocref{https://fr.wikipedia.org/wiki/Rosier}{Rosier de Provins.Rosier à roses rouges.} & Rosa gallica L. & Rosaceae & Bouton floral, pétale. & En l’état \\ \hline
\vocref{https://fr.wikipedia.org/wiki/Rosier}{Rosier sauvage.} & Voir : Eglantier. &  &  &  \\ \hline
\vocref{https://fr.wikipedia.org/wiki/Safran.}{Safran.} & Crocus sativus L. & Iridaceae & Stigmate. & En l’étatEn poudre \\ \hline
\vocref{https://fr.wikipedia.org/wiki/Sarriette}{Sarriette des jardins.} & Satureja hortensis L. & Lamiaceae & Feuille, sommité fleurie. & En l’étatEn poudre \\ \hline
\vocref{https://fr.wikipedia.org/wiki/Sarriette}{Sarriette des montagnes.} & Satureja montana L. & Lamiaceae & Feuille, sommité fleurie. & En l’étatEn poudre \\ \hline
\vocref{https://fr.wikipedia.org/wiki/Sauge}{Sauge d’Espagne.} & Salvia lavandulifolia Vahl. & Lamiaceae & Feuille, sommité fleurie. & En l’étatEn poudre \\ \hline
\vocref{https://fr.wikipedia.org/wiki/Sauge}{Sauge officinale.} & Salvia officinalis L. & Lamiaceae & Feuille. & En l’état \\ \hline
\vocref{https://fr.wikipedia.org/wiki/Sauge}{Sauge sclarée.Sclarée toute-bonne.} & Salvia sclarea L. & Lamiaceae & Feuille, sommité fleurie. & En l’étatEn poudre \\ \hline
\vocref{https://fr.wikipedia.org/wiki/Sauge}{Sauge trilobée.} & Salvia fruticosa Mill.(= S. triloba L. f.). & Lamiaceae & Feuille. & En l’étatEn poudre \\ \hline
\vocref{https://fr.wikipedia.org/wiki/Seigle.}{Seigle.} & Secale cereale L. & Poaceae & Fruit, son. & En l’étatEn poudre \\ \hline
\vocref{https://fr.wikipedia.org/wiki/Serpolet.thym}{Serpolet.Thym serpolet.} & Thymus serpyllum L. sensu latiore. & Lamiaceae & Feuille, sommité fleurie. & En l’étatEn poudre \\ \hline
\vocref{https://fr.wikipedia.org/wiki/Sterculia.gomme}{Sterculia.Gomme Karaya.Gomme M’Bep.Gomme de Sterculia.} & Sterculia urens Roxb., S. tomentosa Guill. et Perr. & Sterculiaceae & Exsudation gommeuse = gomme de Sterculia, gomme Karaya, gomme M’Bep. & En l’étatEn poudreExtrait sec aqueux \\ \hline
\vocref{https://fr.wikipedia.org/wiki/Sureau}{Sureau noir.} & Sambucus nigra L. & Caprifoliaceae & Fleur, fruit. & En l’état \\ \hline
\end{tabularx}
\newpage
\noindent\begin{tabularx}{\textwidth}{|X|X|X|X|X|}
\hline
\rowcolor{headerbg} \textcolor{white}{\textbf{Nom français}} & \textcolor{white}{\textbf{Nom latin}} & \textcolor{white}{\textbf{Famille}} & \textcolor{white}{\textbf{Parties utilisées}} & \textcolor{white}{\textbf{Forme de préparation}}  \\ \hline
\vocref{https://fr.wikipedia.org/wiki/Tamarinier}{Tamarinier de l’Inde.} & Tamarindus indica L. & Fabaceae & Pulpe de fruit. & En l’étatEn poudre \\ \hline
\vocref{https://fr.wikipedia.org/wiki/Temoe-lawacq.}{Temoe-lawacq.} & Curcuma xanthorrhiza Roxb. & Zingiberaceae & Rhizome. & En l’état \\ \hline
\vocref{https://fr.wikipedia.org/wiki/Thé}{Thé du Paraguay.} & Voir : Maté. &  &  &  \\ \hline
\vocref{https://fr.wikipedia.org/wiki/Théier.thé.}{Théier.Thé.} & Camellia sinensis (L.) Kuntze(= C. thea Link)(= Thea sinensis (L.) Kuntze). & Theaceae & Feuille. & En l’étatExtrait sec aqueux \\ \hline
\vocref{https://fr.wikipedia.org/wiki/Thym.}{Thym.} & Thymus vulgaris L.,T. zygis L. & Lamiaceae & Feuille, sommité fleurie. & En l’étatEn poudre \\ \hline
\vocref{https://fr.wikipedia.org/wiki/Thym}{Thym serpolet.} & Voir : Serpolet. &  &  &  \\ \hline
\vocref{https://fr.wikipedia.org/wiki/Tilleul.}{Tilleul.} & Tilia platyphyllos Scop., T. cordata Mill.(= T. ulmifolia Scop.) (= T. parvifolia Ehrh.ex Hoffm.) (= T. sylvestris Desf.),T. × vulgaris Heyne ou mélanges. & Tiliaceae & Aubier, inflorescence. & En l’état \\ \hline
\vocref{https://fr.wikipedia.org/wiki/Trèfle}{Trèfle d’eau.} & Voir : Ményanthe. &  &  &  \\ \hline
\vocref{https://fr.wikipedia.org/wiki/Ulmaire.}{Ulmaire.} & Voir : Reine-des-prés. &  &  &  \\ \hline
\vocref{https://fr.wikipedia.org/wiki/Verveine}{Verveine odorante.} & Aloysia citrodora Palau(= Aloysia triphylla (L’Hérit.) Britt.)(= Lippia citriodora H.B.K.). & Verbenaceae & Feuille. & En l’état \\ \hline
\vocref{https://fr.wikipedia.org/wiki/Vigne}{Vigne rouge.} & Vitis vinifera L. & Vitaceae & Feuille. & En l’état \\ \hline
\vocref{https://fr.wikipedia.org/wiki/Violette.}{Violette.} & Viola calcarata L.,V. lutea Huds.,V. odorata L. & Violaceae & Fleur. & En l’état \\ \hline
\vocref{https://fr.wikipedia.org/wiki/Violette}{Violette tricolore.} & Voir : Pensée sauvage. &  &  &  \\ \hline
\end{tabularx}
\newpage
\end{document}