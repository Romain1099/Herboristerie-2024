\documentclass[a4paper,11pt,fleqn]{article}

\usepackage[left=1cm,right=0.5cm,top=0.5cm,bottom=2cm]{geometry}

\usepackage{bfcours}
\usepackage{cite} % ou \usepackage{natbib} selon votre style de citation
\usepackage{fontawesome5}
\def\rdifficulty{1}
\setrdexo{%left skip=1cm,
display exotitle,
exo header = tcolorbox,
%display tags,
skin = bouyachakka,
lower ={box=crep},
display score,
display level,
save lower,
score=\points,
level=\rdifficulty,
overlay={\node[inner sep=0pt,
anchor=west,rotate=90, yshift=0.3cm]%,xshift=-3em], yshift=0.45cm
at (frame.south west) {\thetags[0]} ;}
]%obligatoire
}
\setrdcrep{seyes, correction=true, correction color=monrose, correction font = \large\bfseries}

\newcommand{\tikzinclude}[1]{%
    \stepcounter{tikzfigcounter}%
    \csname tikzfig#1\endcsname
}
\input{tex/figures}

\newcommand{\citer}[2]{
    \hfill\begin{minipage}[t]{0.4\textwidth}
        \begin{center}
            \frquote{\textit{#1}}\\
            \hfill #2
        \end{center}
    \end{minipage}
}


\hypersetup{
    pdfauthor={C.Ehrhard - R.Deschamps},
    pdftitle={},
    pdfsubject={},
    pdfkeywords={},
    pdfproducer={LuaLaTeX},
    pdfcreator={Boum Factory}
}

\newcommand{\monimage}[2]{
	\includegraphics[width=#1\textwidth]{Images/#2}
}
\newcommand{\includeplantenature}[4][0.25]{
    \begin{minipage}[t]{#1\textwidth}    
        \phantom{a}
        \begin{center}
            \includegraphics[width=0.8\linewidth]{Images/plantes_nature/#2}
            \captionof{figure}{#3}
            {\footnotesize Plante en milieu naturel\\#4}
        \end{center}
    \end{minipage}
}
\newcommand{\includeplanteseche}[4][0.25]{
    \begin{minipage}[t]{#1\textwidth}    
        \phantom{a}
        \begin{center}
            \includegraphics[width=0.8\linewidth]{Images/plantes_sechees/#2}
            \captionof{figure}{#3}
            {\footnotesize Plante émondée\\#4}
        \end{center}
    \end{minipage}
}
\newcommand{\includeprepa}[4][0.25]{
    \begin{minipage}[t]{#1\textwidth}    
        \phantom{a}
        \begin{center}
            \includegraphics[width=0.8\linewidth]{Images/preparations/#2}
            \captionof{figure}{#3}
            {\footnotesize Préparation\\#4}
        \end{center}
    \end{minipage}
}

\newcommand{\vocref}[2]{
    \href{#1}{\bfseries\color{monrose} #2\index{#2}}
}
\newcommand{\vocnoindexref}[2]{
    \href{#1}{\bfseries\color{monrose} #2}
}

\usepackage{imakeidx}
\makeindex[title=Index des termes, intoc]

\usepackage{datatool}

\definecolor{headerbg}{RGB}{0, 128, 0} % Couleur du fond de l'en-tête
\definecolor{rowbg1}{RGB}{245, 245, 245}
\definecolor{rowbg2}{RGB}{255, 255, 255}

% Charger les données depuis le fichier CSV
%\DTLloaddb{liste148}{Ressources/liste_148.csv}

\definecolor{highlight}{RGB}{255, 255, 0} % Couleur de surlignage
\newcommand{\voc}[1]{\textbf{\color{red!65!black}#1}\index{#1}}

\newcolumntype{L}[1]{>{\raggedright\arraybackslash}p{#1}} % Définir une colonne alignée à gauche avec largeur fixe

% Commande pour créer un tableau pour une plante
\newcommand{\plante}[5]{
\begin{tabular}{|L{3cm}|L{4cm}|L{2cm}|L{3cm}|L{3cm}|}
\hline
\rowcolor{headerbg}
\textcolor{white}{\textbf{Nom français}} & \textcolor{white}{\textbf{Nom latin}} & \textcolor{white}{\textbf{Famille}} & \textcolor{white}{\textbf{Parties utilisées}} & \textcolor{white}{\textbf{Forme de préparation}} \\
\hline
\rowcolor{rowbg1}
#1 & #2 & #3 & #4 & #5 \\
\hline
\end{tabular}
}

\renewcommand\boite[2]{
\begin{tcolorbox}[nobeforeafter,title=\bcfleur #1,halign title=flush left,fonttitle=\bfseries,colbacktitle=headerbg,coltitle=white,colback=white]%red!50!black
#2
\end{tcolorbox}
}

% Définition de la commande \boiteidentiteplante
\NewDocumentEnvironment{boiteidentiteplante}{o+b}{
    \begin{tcolorbox}[
        enhanced,
	breakable,
        before skip=2mm,after skip=2mm,
        colback=red!5,colframe=red!50,boxrule=0pt,
        attach boxed title to top left={xshift=1cm,yshift*=1mm-\tcboxedtitleheight},
        varwidth boxed title*=-3cm,
        boxed title style={frame code={
            \path[fill=blue!50]
            ([yshift=-1mm,xshift=-1mm]frame.north west)
            arc[start angle=0,end angle=180,radius=1mm]
            ([yshift=-1mm,xshift=1mm]frame.north east)
            arc[start angle=180,end angle=0,radius=1mm];
            \path[left color=blue!50,right color=blue!50]
            ([xshift=-2mm]frame.north west) -- ([xshift=2mm]frame.north east)
            [rounded corners=1mm]-- ([xshift=1mm,yshift=-1mm]frame.north east)
            -- (frame.south east) -- (frame.south west)
            -- ([xshift=-1mm,yshift=-1mm]frame.north west)
            [sharp corners]-- cycle;
        },interior engine=empty,
        },
        fonttitle=\bfseries,
        title={\large{Fiche d'identité}},
        coltitle =white,
        drop shadow,
        borderline west={0.05mm}{0pt}{blue!50},
        borderline south={0.05mm}{0pt}{blue!50!black},
        overlay={
            \draw[line width=0.5mm, rem!50] 
            ([xshift=0mm,yshift=-0.25mm]frame.south west)--([xshift=0mm]frame.north west); % Bordure gauche
            \draw[line width=0.5mm, rem!50] 
            ([yshift=0mm]frame.south west)--([yshift=0mm]frame.south east); % Bordure du bas
            \ifx#1\empty
            \else
            \node[anchor=north east, fill=white, draw=rem!50, rounded corners] at ([xshift=-4cm]frame.north east) {\begin{minipage}{0.5\textwidth} \centering \textbf{#1} \end{minipage}};
            \fi%
        }
    ]
    #2
    \end{tcolorbox}
}{}

\NewDocumentEnvironment{boiterecette}{o+b}{
    \begin{tcolorbox}[
        enhanced,
	    breakable,
        before skip=2mm,after skip=2mm,
        colback=red!5,colframe=green!30!gray,boxrule=0pt,
        attach boxed title to top left={xshift=1cm,yshift*=1mm-\tcboxedtitleheight},
        varwidth boxed title*=-3cm,
        boxed title style={frame code={
            \path[fill=green!30!gray]
            ([yshift=-1mm,xshift=-1mm]frame.north west)
            arc[start angle=0,end angle=180,radius=1mm]
            ([yshift=-1mm,xshift=1mm]frame.north east)
            arc[start angle=180,end angle=0,radius=1mm];
            \path[left color=green!30!gray,right color=green!30!gray]
            ([xshift=-2mm]frame.north west) -- ([xshift=2mm]frame.north east)
            [rounded corners=1mm]-- ([xshift=1mm,yshift=-1mm]frame.north east)
            -- (frame.south east) -- (frame.south west)
            -- ([xshift=-1mm,yshift=-1mm]frame.north west)
            [sharp corners]-- cycle;
        },interior engine=empty,
        },
        fonttitle=\bfseries,
        title={\large{Recette}},
        coltitle =white,
        drop shadow,
        borderline west={0.05mm}{0pt}{red!40},
        borderline south={0.05mm}{0pt}{red!40!black},
        overlay={
            \draw[line width=0.5mm, rem!50] 
            ([xshift=0mm,yshift=-0.25mm]frame.south west)--([xshift=0mm]frame.north west); % Bordure gauche
            \draw[line width=0.5mm, rem!50] 
            ([yshift=0mm]frame.south west)--([yshift=0mm]frame.south east); % Bordure du bas
            \ifx#1\empty
            \else
            \node[anchor=north east, fill=white, draw=rem!50, rounded corners] at ([xshift=-4cm]frame.north east) {\begin{minipage}{0.5\textwidth} \centering \textbf{#1} \end{minipage}};
            \fi%
        }
    ]
    #2
    \end{tcolorbox}
}{}

% Définition de la commande \ficheidentiteplante
\newcommand{\ficheidentiteplante}[9]{
    \subsubsection{#1}

    \ifx\cita\empty
    \else
    \cita 
    \fi%


    \begin{boiteidentiteplante}[#1]

        \begin{multicols}{2}
            	\boite{Type d'effet :}{#2}
		
		        \columnbreak

            	\boite{Effets recherchés :}{#3}
        \end{multicols}

        \begin{multicols}{2}
            \begin{bclogo}[couleur=blue!30,arrondi=0.1,logo=\bccalendrier]{Cueillette}
                #4
            \end{bclogo}
            \begin{bclogo}[couleur=blue!30,arrondi=0.1,logo=\bcoutil]{Utilisation privilégiée}
                #5
            \end{bclogo}

	        \columnbreak

	        \begin{center}\includeplantenature[0.4]{#7}{#8}{#9}\end{center}
        \end{multicols}

        #6

    \end{boiteidentiteplante}
}
\newcommand{\cita}{}
\newcommand{\ficheidentiteplantelong}[9]{
    \subsubsection{#1}

    \ifx\cita\empty
    \else
    \cita 
    \fi%


    \begin{boiteidentiteplante}[#1]

        \begin{multicols}{2}
            	\boite{Type d'effet :}{#2}
		
		        \columnbreak

            	\boite{Effets recherchés :}{#3}
        \end{multicols}

        \begin{multicols}{2}
            \begin{bclogo}[couleur=blue!30,arrondi=0.1,logo=\bccalendrier]{Cueillette}
                #4
            \end{bclogo}

	        \columnbreak

	        \begin{center}\includeplantenature[0.4]{#7}{#8}{#9}\end{center}
        \end{multicols}
	 \begin{bclogo}[couleur=blue!30,arrondi=0.1,logo=\bcoutil]{Utilisation privilégiée}
                \begin{multicols}{2}
			#5	
		\end{multicols}
          \end{bclogo}
        #6

    \end{boiteidentiteplante}
}
\newcommand{\ficherecette}[9]{
    \subsubsection{#1}
    %citation si citation
    \begin{boiterecette}[#1]
        \begin{multicols}{2}
            \begin{bclogo}[couleur=green!30!white,arrondi=0.1,logo=\bcoeil]{Ingrédients}
                \begin{itemize}[label=\mysquare]
                    #2
                \end{itemize}
            \end{bclogo}
            \begin{bclogo}[couleur=green!40!white,arrondi=0.1,logo=\bcoutil]{Matériel}
                \begin{itemize}[label=\mysquare]
                    #3
                \end{itemize}
            \end{bclogo}
        \end{multicols}

        \boite{Préparation :}{
            \begin{multicols}{2}
                \begin{enumerate}
                    #4
                \end{enumerate}
            \end{multicols}
        }

        \begin{multicols}{2}

            \begin{bclogo}[couleur=green!40!gray,arrondi=0.1,logo=\bcbook]{Conservation / Utilisation}
                #5
            \end{bclogo}

	        \columnbreak

	        \begin{center}\includeprepa[0.4]{#7}{#8}{#9}\end{center}
        \end{multicols}
        \begin{Remarque}
            #6
        \end{Remarque}
    \end{boiterecette}
}

% Définition de la commande \lien
\newcommand{\lien}[2]{\hyperref[#1]{\bfseries\color{red!50!yellow}#2}}

\newcommand{\Potins}[2]{
    \begin{bclogo}[couleur=blue!10, arrondi=0.1, logo=\bcbook, marge=10]{Potins des plantes - #1}
        #2% Votre contenu ici
    \end{bclogo}
}

\newcommand{\subtil}[2]{
    \begin{bclogo}[couleur=blue!10, arrondi=0.1, logo=\bcyin, marge=10]{Sur le plan subtile - #1}
        \begin{itemize}[label=\faPen]
            #2% Votre contenu ici
        \end{itemize}
    \end{bclogo}
}
\begin{document}
\setcounter{pagecounter}{0}
\setcounter{ExoMA}{0}
\setcounter{prof}{0}
%Pour les overlay
\def\points{\phantom{AAA}}
\def\difficulty{\phantom{AAA}}
\chapitre[
    $\mathcal{C}\mathcal{R}$% niveau % $\mathbf{6^{\text{ème}}}$
    ]{
    Herboristerie 2024% theme
    }{
    Association% type_etablissement
    }{
    \vocref{https://arsimed.net/}{Arsimed}% nom_etablissement
    }{
    \label{exemple}\tableofcontents \newpage% supplement
    }{
    Notes de stage :% type_document
    }

    \section{Introduction}

        
\subsection{Le métier d'herboriste}
Le métier d'herboriste n'est plus reconnu depuis 1941. Depuis, il y a un monopole des pharmaciens et des médecins sur le conseil des propriétés des plantes.
\begin{Remarque}
    Chez l'herboriste actuel(le), on trouve de l'ortie et de la camomille qui n'est pas français (le salaire des cueilleurs est trop élevé).
\end{Remarque}

\subsection{Les étapes de transformation d'une plante}

Il est important de connaître certaines informations avant d'utiliser une plante. \\
La \textbf{première question à se poser}, c'est \textit{que veut-on faire de la plante ?}, quels \textit{principes actifs} voulons-nous \textbf{extraire}.\\

La réponse à cette question permet de choisir l'un des \textbf{trois solvants} principalement utilisés pour effectuer des préparations.\\
\begin{itemize}[label=\faPen]
    \item L'eau
    \item L'alcool
    \item L'huile
\end{itemize}
Les utilisations de ces solvants seront décrites dans la section \textbf{Transformations} de ce document.\\
Dans les grandes lignes, chacun de ces solvant permet d'extraire exclusivement certaines molécules. En outre, chaque solvant a ses particularités de \textbf{préparation} et de \textbf{conservation}.


        \subsection*{Comment utiliser ce document ?}
            \addcontentsline{toc}{subsection}{Comment utiliser ce document ?}
            \subsection*{Interactivité}

Ce document dispose de fonctionnalités interactives. \\
Des \vocref{https://fr.wikipedia.org/wiki/Hyperlien}{hyperliens} balisés en {\color{monrose}rose} dirigent vers des \textbf{pages web} donnant plus de détails. \\
\textbf{Pour les ouvrir} il est conseillé de maintenir \frquote{ctrl} appuyé au clic pour \frquote{ouvrir dans un nouvel onglet}. 

Des \lien{exemple}{liens internes en orange} permettent de se diriger à l'intérieur de ce document. \\

La table des matières est interactive également. \\

Il est possible d'utiliser certains lecteurs pdf pour pouvoir naviguer entre les parties via un menu latéral. 

\subsection*{Participation - Modification}

Le code source du document est disponible sur \textbf{GitHub} : \vocref{https://github.com/Romain1099/Herboristerie-2024.git}{Herboristerie 2024}\\
Il est ainsi possible de télécharger le code source en local pour le \textbf{modifier} ou \textbf{participer} à son élaboration. \\
Pour ce faire, il sera nécessaire d'utiliser mon package personnel : \vocref{https://github.com/Romain1099/BFCours.git}{BFCours}


        \subsection*{Remerciements}
            \addcontentsline{toc}{subsection}{Remerciements}
            Groupe d'Herboristerie - Stages Arsimed - session 1

Formateur : Delphine - \frquote{Sacrées Plantes}

Participants : 
\begin{itemize}[label = \faPen]
    \item \'Emilie
    \item Alfred
    \item Pauline a.k.a \frquote{Pissenlit}
    \item Catherine
    \item Mirjam
    \item Cynthia
    \item Romain
\end{itemize}

    \newpage
    \section{Transformations}

        \subsection{Teintures mères}

\begin{Defi}[Teinture mère]

    Une \voc{teinture mère} désigne une préparation de plantes infusées dans de l'\voc{alcool}. \\
    L'utilisation professionnel de ce terme est \textit{réservé} aux pharmaciens. Nous utiliserons donc le terme commun d'\voc{alcoolature} dans la suite de ce document.\\

    On parle de \voc{teinture officinale} lorsque la préparation est réalisée à l'aide de \textbf{plantes sèches}.\\

    On parle de \voc{dynamiser} une solution lorsqu'on la \textbf{mélange}.\\

    La \voc{diffusion} des principes actifs s'effectue en \voc{dynamisant} le bocal \textbf{chaque jour}. 
\end{Defi}
\begin{Remarque}[]%\monimage{0.05}{avec_sourire.png}

    \begin{itemize}[label=\faPen]
        \item Plus le degré d'alcool est élevé, plus les principes actifs se diffuseront efficacement. 
        \item Permet d'effectuer une préparation même si on dispose d'une \textbf{faible quantité de plantes}. 
        \item On peut utiliser indiféremment des plantes \voc{fraîches} ou \voc{sèches}.
        \item Cela permet d'extraire des \voc{principes actifs} \voc{complexes} de la plante. Par exemple :
                \begin{multicols}{3}
                    \begin{itemize}
                        \item gomme et résine.
                        \item \vocref{https://fr.wikipedia.org/wiki/Alcalo\%C3\%AFde}{alcaloïdes} 
                        \item les principes \voc{volatiles}
                    \end{itemize}
                \end{multicols}
    \end{itemize}
\end{Remarque}
\begin{multicols}{2}
    \boite{Préparation :}{
        Utiliser un alcool assez \textbf{fort} ( type rhum, absinthe\ldots ). \\
        \textbf{Degré d'alcool souhaité} : entre $40\degres$ et $90\degres$. \\
        \begin{itemize}[label=\mysquare]
            \item Découper les plantes \textbf{séchées} à l'aide d'un \voc{sécateur} ou d'une \voc{paire de ciseaux}.\\Il est également possible de les broyer à l'aide d'un mortier.
            \item Les placer dans un bocal adapté à la taille de la cueillette, \textbf{à ras} et \textbf{sans tasser}.
            \item Couvrir d'\voc{alcool} en veillant à \textbf{éliminer} les \textbf{bulles d'air}.
            \item Refermer le bocal et conserver dans un environnement \textbf{propre} et \voc{à l'abri de la lumière}.
        \end{itemize}
    }

        
    \columnbreak 


    \boite{Macération :}{
        \begin{itemize}[label=\mysquare]
            \item \voc{Dynamiser} chaque jour pendant \textbf{28 jours} pour permettre la \voc{diffusion} des principes actifs. 
        \end{itemize}
    }\\
    \includeprepa[0.2]{couper_plantes.jpg}{Découper les plantes}{30/07/2024}
    \includeprepa[0.2]{alcoolature_coquelicot.jpg}{Remplir le bocal}{30/07/2024}


\end{multicols}
\begin{center}

\boite{Conservation :}{
    \begin{itemize}[label=\faPen]
        \item Se conserve \voc{à l'abri de l'humidité}.
        \item La teinture mère se conserve sur une période allant de 2 à 5 ans.% \monimage{0.1}{groupe/tetra.png}
    \end{itemize}
    
}
\end{center}
\boite{Conseils d'utilisation}{
    \begin{itemize}[label=\faPen]
        \item Une attention toute particulière est à porter à l'utilisation de teinture mère. 
        \item L'automédication n'est pas une pratique à prendre à la légère et il est conseillé de s'entretenir avec un professionnel. 
        \item Les dosages ne sont plus soumis à autant de contraintes qu'auparavant.\\
                Les entreprises pharmaceutiques sont libres de doser la quantité de plante dans leurs teintures mères sans l'indiquer. 
                Néanmoins, voici quelques bonnes pratiques concernant la posologie : \\
    \end{itemize}
    \boite{Posologie :}{
        \begin{itemize}[label=\faPen]
            \item Pour un alcool à $50\degres$, la posologie indiquée est $\mathbf{30}$\textbf{gouttes par jour}. 
            \item De façon générale, commencer avec une dose réduite puis augmenter progressivement. 
        \end{itemize}
    }
}


\newpage
\subsection{Solvant eau}
RàS mon capitaine !
\newpage
\subsection{Solvant vinaigre}

\begin{Defi}[Macération au vinaigre]
	
	\label{vinaigre}
    Une \voc{macération au vinaigre} consiste à faire \voc{macérer} une plante dans du \voc{vinaigre}.\\

	C'est l'effet du \textbf{mouvement des plantes} chaque jour qui permet de diffuser les \voc{principes actifs}.\\

    L'utilisation du \textbf{solvant vinaigre} permet un compromis entre les propriétés des \textbf{teintures mères} et des \textbf{infusions}.\\
	
	

\end{Defi}

\begin{multicols}{2}
    \boite{Préparation :}{
		Pour préparer une macération au vinaigre, on respecte en général un \voc{dosage} de $40$g de \textbf{plantes fraîches} pour $1$L de \textbf{vinaigre}.\\
		\begin{itemize}[label = \faPen]
            \item Dans un bocal \voc{hermétique} et \voc{désinfecté}, remplir de \voc{plantes fraîches} à \textbf{ras}.
            \item \textbf{Couvrir} le mélange de vinaigre.\\On pourra utiliser du vinaigre de \voc{cidre de pommes}.
            \item Noter sur une étiquette les informations essentielles de la préparation.\\Exemples : Nom des plantes, type de vinaigre, lieu de récolte, date... 
        \end{itemize}
	}

        
    \columnbreak 


    \boite{Macération :}{
        \begin{itemize}[label=\mysquare]
            \item \voc{Dynamiser} chaque jour pendant \textbf{28 jours} pour permettre la \voc{diffusion} des principes actifs. 
            \item Conserver \textbf{à l'abri de la lumière}.
        \end{itemize}
    }\\
    \hfill
	\includeprepa[0.35]{vinaigre_plantes_internet.jpg}{Vinaigre de plantes}{source : Internet}
\hfill



\end{multicols}

\begin{Remarque}
    Une macération au vinaigre permet d'extraire :
	\begin{multicols}{2}
        \begin{itemize}[label = \faPen]
		\item Certains \textbf{acides} comme par exemmple la \voc{vitamine C}.
		\item Des tanins.
		\item Des antioxydants.
		\item Bien que le vinaigre ne soit pas le meilleur solvant pour extraire les huiles essentielles, certains composants volatils ou semi-volatils peuvent être extraits en petite quantité.
		\item Les saponines peuvent être extraites ainsi.\\ Elles sont utiles pour récupérer les principes \voc{expectorants} et \voc{anti-inflammatoires}.
		\item Certains \voc{minéraux} notamment le \voc{calcium}, le \voc{magnésium} et le \voc{fer}.
        \end{itemize}
\end{Remarque}
\newpage
\label{huileux}
\subsection{Macérat huileux}

\begin{Defi}[Macérat huileux]

    Un \voc{macérat huileux} ou \voc{macérat solaire} désigne une infusion de plantes dans un \voc{corps gras}. \\
    Ici, c'est l'effet de la \textbf{chaleur} qui permet la \voc{diffusion} des principes actifs.\\

    Cette technique permet d'extraire les \voc{huiles essentielles}, les \voc{cires} et les \voc{résines}.
\end{Defi}

\begin{multicols}{2}
    \boite{Préparation}{
        \begin{enumerate}
            \item Découper les plantes \textbf{séchées}.
            \item Les placer dans un bocal \textbf{à ras} et \textbf{sans tasser}.
            \item Couvrir d'huile \textbf{végétale} en veillant à \textbf{éliminer} les \textbf{bulles d'air} à l'aide d'une petite spatule.
            \item Laisser le bocal en le \textbf{couvrant} avec une \voc{étamine} pendant 24h dans un environnement \textbf{propre} et \voc{ensoleillé} ( ou chaud ). 
        \end{enumerate}
    }

    \boite{Macération}{
        \begin{itemize}[label=\mysquare]
            \item Refermer le pot en éliminant la \voc{condensation}. 
            \item Conserver \voc{au soleil} ou \voc{au chaud} pendant $\mathbf{28}$\textbf{ jours}.
            \item Passé cette période : filtrer les plantes et verser l'huile dans un bocal.
        \end{itemize}
    }
\end{multicols}

\boite{Conservation}{
    \begin{itemize}[label=\faPen]
        \item Se conserve \voc{à l'abri de l'humidité}.
        \item Le macérat huileux peut se conserver sur une période allant de 6 mois à 2 ans.% \monimage{0.1}{groupe/tetra.png}
    \end{itemize}
    
}
\boite{Conseils d'utilisation}{
    Il est conseillé de \voc{filtrer} le mélange avant d'utiliser l'huile. \\
    Selon l'huile choisie, on pourra consommer le macérat lors des repas. \\

    On peut les utiliser pour des bougies, des huiles de massages, pour soigner (interne/externe) ou pour utilisation cosmétique.
}

\begin{Remarque}[]%\monimage{0.05}{avec_sourire.png}
    \begin{itemize}[label=\faPen]
        \item Les huiles les plus \voc{stables} sont l'\voc{huile d'olive} et l'\voc{huile de tournesol} et permettent une conservation sur \textbf{2 ans}. \\
        \item Les autres huiles végétales ont une durée de conservation de \textbf{6 mois}.
        \item Les plantes doivent être séchées au préalable afin qu'il n'y ait \voc{pas d'insecte} et \textbf{pas d'humidiité}.
    \end{itemize}
\end{Remarque}
\newpage

	\newpage
    \section{Botanique}

        \subsection{Introduction}
En France, la vente des plantes médicinales (inscrite à la \voc{pharmacopée}), est réservée aux pharmaciens, à l’exception de 148 espèces libérées et d’une centaine d’aromates et épices.


Il existe deux listes de plantes médicinales, la liste A et B. \\
La liste A comprend les plantes testées scientifiquement et bonnes par la santé .\\
la liste B comprend les plantes poison.\\
Nous parlerons que de la liste A, qui comprend 148 plantes autorisées à la production et à la vente. D'un point de vue légal, il n'y a que 7 plantes autorisées.\\

On utilise la \lien{biglist}{liste des 148} qui répertorie les informations de base sur les plantes librement accessibles. \\

Je me suis basé sur le site internet suivant pour la construire : \\
\begin{center}\href{https://www.passerelleco.info/spip.php?page=article\&id_article=407}{https://www.passerelleco.info/spip.php?page=article\&id\_article=407}\end{center}

\begin{Defi}[Sommité fleurie]
    \label{sommite}
    La \voc{sommité fleurie} d'une plante désigne la partie aérienne contenant l'\vocref{https://fr.wikipedia.org/wiki/Inflorescence}{inflorescence} sommitale ou apicale.\\
    Cette sommité fleurie se compose de la zone florale avec les fleurs, les feuilles et la tige.\\
    Elle inclut une grande partie de la tige florale, où débute la première fleur (ou fleuron).
\end{Defi}

\begin{Defi}[Partie aérienne]
    \label{aerienne}
    La \voc{partie aérienne} d'une plante désigne la structure émergée de la plante.\\
    Elle a pour fonction de soutenir et de structurer la plante, en soutenant ses autres organes végétaux aériens, tels que les feuilles et les fleurs.\\ Une autre de ses principales caractéristiques est qu'elle a un géotropisme négatif, ce qui signifie qu'elle pousse dans la direction opposée à la gravité.\\
    source : \href{https://www.projetecolo.com/anatomie-d-une-plante-les-parties-d-une-plante-avec-schema-31.html}{internet}.
\end{Defi}

\boite{Parties utilisées dans les plantes :}{
Il y a plusieurs parties de la plante qui peuvent être utilisées selon les plantes : 

\begin{multicols}{2}
    \begin{itemize}[label = \bcfleur]
        \item Les \vocref{https://fr.wikipedia.org/wiki/Bourgeon_(botanique)}{bourgeons}
        \item Les \vocref{https://fr.wikipedia.org/wiki/Fleur}{fleurs}
        \item Les \lien{sommite}{sommités fleuries}
        \item Les \vocref{https://fr.wikipedia.org/wiki/Racine_(botanique)}{racines}
        \item Les \vocref{https://fr.wikipedia.org/wiki/Pétale}{pétales}
        \item Les \vocref{https://fr.wikipedia.org/wiki/Racine_(botanique)}{parties souterraines}
        \item Les \vocref{https://fr.wikipedia.org/wiki/Feuille}{feuilles}
        \item La \vocref{https://fr.wikipedia.org/wiki/Sève}{sève}
        \item Les \vocref{https://fr.wikipedia.org/wiki/Rhizome}{rhizomes}
        \item Les \lien{aerienne}{parties aériennes}
        \item Les \vocref{https://fr.wikipedia.org/wiki/Graine}{graines} ou \vocref{https://fr.wikipedia.org/wiki/Semence_(agriculture)}{semences}
        \item Les plantes entières (l'ortie)
        \item Les \vocref{https://fr.wikipedia.org/wiki/Cône_(botanique)}{cônes}. 
    \end{itemize}
\end{multicols}
}

\boite{Nommer les plantes}{
    Il existe plusieurs manières de nommes les plantes : le \textbf{nom latin} mais aussi des \textbf{noms communs} qui donnent des informations sur l'usage de la plante.\\
    Par exemple, le plantain, appelé aussi l'herbe à cinq coutures, le lancéolé et le major.\\

    Pour une \textbf{utilisation médicinale}, il est important de noter le \textbf{nom latin} afin d'être certain des \textbf{propriétés de la plante}.
}

\boite{Propriétés des plantes}{
    Toutes les plantes ont des propriétés \cite{plantemed}. 
}

\boite{Les familles de plantes}{
    Toutes les familles des plantes se terminent par -acée (en français) ou par -aceae (en latin).\\
    On pourra trouver des compléments ici sur la \vocref{https://parlonssciences.ca/ressources-pedagogiques/documents-dinformation/la-taxonomie-vegetale}{taxonomie végétale} ou là pour les \vocref{https://verstige.fr/familles}{familles de plantes}.
    
 
}



\subsection{Asteraceae}

\ficheidentiteplante
{Achillée millefeuille}
{%effet général
    L'\vocref{https://fr.wikipedia.org/wiki/Achill\%C3\%A9e_millefeuille}{achillée millefeuille} est considérée comme la \voc{plante de la femme} de base.\\ 
    Elle est utilisée dans la pharmacopée pour son effet sur les \voc{saignements} et \voc{désinfectant}.\\
    On dit que c'est une plante \vocref{https://fr.wikipedia.org/wiki/Emm\%C3\%A9nagogue}{emménagogue}.\\

    La plante a également un effet sur la \textbf{digestion}.
}
{%utilisation privilégiée
    \begin{itemize}[label = \bcplume]
        \item Dans le cas d'un oedème
        \item \textbf{règles}
        \item \textbf{varices}
        \item Saignements ( même importants )
    \end{itemize}
}
{%infos cueillette
    ?
}
{%sous quelle forme utiliser
    \begin{itemize}[label = \bccrayon]
        \item Plante sèche
        \item \voc{Macérat huileux}
        \item \voc{Teinture mère}
    \end{itemize}
}
{%remarques
    \begin{itemize}[label = \bcplume]
        \item C'est une plante plutôt \voc{amère}, \voc{astreingeante}.
        \item Se combine avec des \textbf{feuilles de framboises} pour faciliter la \voc{reminéralisation} après un saignement.
        \item C'est une plante \vocref{https://fr.wikipedia.org/wiki/Liste_de_plantes_tinctoriales}{teinctoriale} associée à la couleur \textbf{vert kaki}.\\Elle est utilisée pour tremper les vêtements militaires ce qui leur donne cette couleur. 
        \item[\bcattention] En cas d'urgence, \textbf{mâcher} la plante de sorte à constituer une \textbf{pâte}.\\ \textbf{Appliquer directement sur la plaie}.
    \end{itemize}
}
{%image
    achillee_millefeuille_wiki.jpeg
}
{%titre photo
    Achillée millefeuille
}
{%description photo : "lieu - date"
    source : Wikipedia 
}

{Armoise annuelle}
{%effet général
    L'\vocref{https://fr.wikipedia.org/wiki/Armoise_annuelle}{Armoise annuelle} .\\

    
}
{%utilisation privilégiée
    \begin{itemize}[label = \bcplume]
        \item \textbf{Antipaludique}, 
        \item \textbf{stimulant digestif}
    \end{itemize}
}
{%infos cueillette
    \begin{itemize}[label = \bcplume]
        \item \textbf{Feuilles}, et \textbf{tiges}.
        \item fin d'été.
    \end{itemize}
}
{%sous quelle forme utiliser
    \begin{itemize}[label = \bccrayon]
        \item \voc{Purin} : conseillé en cas de piéride du chou. Efficace en pulvérisation contre les limaces. Mettez à macérer 1kg de plante fraîche dans 10L d'eau pendant 24 heures.
        \item \voc{Tisanes} : 15 à 30g par litre d'eau, 3 à 5 tasses par jour.
        \item \voc{Teinture mère} : 3 à 5 gouttes par jour.
        \item \voc{Elixir floraux} : pour aider les hypersensibles à s'enraciner. Cela apaise harmonise et rééquilibre la sensibilité.
    \end{itemize}
}
{%Usage interne
\begin{itemize}[label = \bccrayon]
    \item l'armoise est surtout connue pour être \voc{emménagogue} pour régulariser ou provoquer les règles, contre les douleurs menstuelles et lors de la ménopause.
    \item Elle est églament efficace pendant l'accouchement surtout en cas de rétention de placenta.
    \item On l'utilise aussi pour les digestion lente, l'inappétence, les crampes d'estomac, les parasites intestinaux, l'anémie, l'épilepsie, les troubles nerveux.
\end{itemize}
}
{%Usage externe
\begin{itemize}[label = \bccrayon]
    \item Les bains d'armoises sont conseillés contre la goutte et les rhumatismes.
    \item les cendres de la plante arrêtent les saignements de nez.
    \item Elle est souvent utilisée comme une aternative au tabac
\end{itemize}
}
{%remarques
    \begin{itemize}[label = \bcplume]
        \item Cette plante est fortement déconseillée en cas de grossesse et d'allaitement.
        \item Elle est toxique en cas d'emploi prolongé
        \item Ne pas confondre avec l'ambroisie qui n'a pas le dessous des feuilles argentée
    \end{itemize}
}
{%image
    armoise annuelle.jpeg
}
{%titre photo
    Armoise annuelle
}
{%description photo : "lieu - date"
    source : Wikipedia 
}


\subsection{Bracicaceae}
\input{tex/botanique/bracicaceae}

\subsection{Hypericaceae}

\ficheidentiteplante
{Millepertuis}
{%effet général
    Le \voc{millepertuis} en latin \textit{hypericum perforatum} est aussi appelé le \voc{soleil intérieur}.\\

    C'est une plante qui a un très fort pouvoir de \voc{guérison} et de \voc{protection}.\\
    Elle est \textbf{riche en mélatonine} et a une action \textbf{régulatrice du sommeil}.
}
{%utilisation privilégiée
    \begin{itemize}[label=\bcoutil]
        \item Contre les dépressions
        \item Contre les \textbf{maux d'hiver} : \\
        \item Dépressions 
        \item Stress
        \item Système nerveux fragilisé
    \end{itemize}
}
{%infos cueillette
    \begin{itemize}[label=\faPen]
        \item Penser à se protéger du soleil.
        \item Vérifier qu'il s'agit d'\textit{hypericum perforatum} en \\
                \begin{itemize}[label = \bcoeil]
                    \item Observant des \textbf{petits trous} sous ses feuilles.
                    \item \textbf{\'Ecrasant la fleur} pour voir s'échapper un \textbf{liquide rouge}. 
                \end{itemize}
    \end{itemize}
}
{%sous quelle forme utiliser
    \begin{itemize}[label = \bcplume]
        \item On utilise principalement les \voc{sommités fleuries}.
        \item On peut l'utiliser en fumigation $\longrightarrow$ vertus protectrices et purificatrices
        \item Peut être utilisé comme plante \voc{teinctoriale} pour colorer en \textbf{rouge}.
    \end{itemize}
}
{%remarques
    
    \begin{minipage}[t]{0.35\textwidth}
        \boite{Utilisation interne :}{
            On peut faire une \lien{infusion}{infusion} ou une \lien{decoction}{decoction} en utilisant $15$ à $20$ grammes de plantes séchées \textbf{par litre d'eau}.\\
            $2$ à $4$ tasses par jour.\\

            \begin{itemize}[label=\faPen]
                \item Affections pulmonaire chronique
                \item Troubles hépatiques et circulatoires
                \item Asthme
                \item Cystite
                \item Herpès
            \end{itemize}
        }
    \end{minipage}
    \hfill
    \begin{minipage}[t]{0.65\textwidth}
        \boite{Utilisation externe :}{
            \setlength{\columnseprule}{1pt}
            \begin{multicols}{2}
                \textbf{En macérat huileux :}
                \begin{itemize}[label=\faPen]
                    \item Douleurs rhumatismales
                    \item Sciatiques
                    \item Tendinites, torticoli, luxation
                    \item Ulcères
                    \item Goutte
                    \item Plaies
                    \item Brûlures
                    \item Hématomes
                    \item Coupures
                    \item Contusions
                \end{itemize}
                
                \columnbreak


                \textbf{En pommade : }
                \begin{itemize}[label=\faPen]
                    \item Douleurs nerveuses
                    \item Engorgement mammaire
                \end{itemize}
                \vspace{0.5cm}
                \textbf{En infusion :} comme \voc{lotion} pour les peaux grasses et \voc{acnéeiques}.
            \end{multicols}
        }
    \end{minipage}
}
{%image
    millepertuis (1).jpeg
}
{%titre photo
    Millepertuis
}
{%description photo : "lieu - date"
    Jardin de  - 30/07/2024 
}

\begin{Remarque}
    \begin{itemize}[label=\faPen]
        \item Le millepertuis induit un effet \voc{photosensible} pouvant occasionner des \textbf{brulures}.
    \end{itemize}
\end{Remarque}

\Potins{Millepertuis}{
    L’herbe de la Saint-Jean: la plante du solstice d’été    
    Le millepertuis avait une grande renommée dans tout les pays celtes. \\
    Elle apportait paix et prospérité au foyer, santé aux animaux et récoltes abondantes. \\
    C’était une plante de guérison, que les gens portaient sur eux à la veille du solstice. \\
    
    
    C’est une plante “solaire”, considérée comme “positive” et récoltée durant les fêtes païennes du solstice avec 
    l’armoise ou l’achillée. \\
    Elle était récoltée de nuit, certains disent même à minuit, afin de conserver la rosée sur les feuilles et les 
    fleurs. Cette eau était considérée comme sacrée et constituait une “eau de longue vie” augmentant les pouvoirs de 
    la plante. \\
    Alexander Carmichael dans son Carmina Gadelica nous donne une incantation à réciter avant la récolte:
    \begin{center}
        \textit{
            Herbe de la Saint-Jean, Herbe de la Saint-Jean\\
            Je te cueillerai avec ma main droite\\
            Je te préserverai avec ma main gauche\\
            Celui qui te trouve dans l’enclos du bétail\\
            Ne sera jamais sans bétail.
        }
    \end{center}
    Les plantes récolées autour du solstice, puis à la Saint-Jean (la fête a été déplacée au 24 juin par les chrétiens) 
    avaient la réputation de chasser les démons et de faire perdre leur dangerosité aux plantes toxiques. 
}

\subsection{Lamiaceae}
\ficheidentiteplante
{Sauge}
{%effet général
    La sauge est considérée comme une \voc{plante de la femme}.\\ Elle est utilisée dans la pharmacopée pour permettre un \voc{rééquilibre hormonal}.
}
{%utilisation privilégiée
    Lors d'épisodes de déséquilibre hormonal chez la femme, l'utilisation peut réduire entre autres la \voc{transpiration}. 
}
{%infos cueillette
    ?
}
{%sous quelle forme utiliser
    \begin{itemize}[label = \bcplume]
        \item Bain de bouche
        \item Fumigation \textbf{après séchage}.
    \end{itemize}
}
{%remarques
    ?
}
{%image
    sauge_de_jerusalem.jpg
}
{%titre photo
    Sauge de Jérusalem
}
{%description photo : "lieu - date"
    Jardin des thermes de \frquote{Cassinomagus} - 04/08/2024 
}


\subsection{Rosaceae}
\ficheidentiteplante
{Aubépine}
{%effet général
    L'\voc{aubépine} 
}
{%utilisation privilégiée
    
}
{%infos cueillette
    ?
}
{%sous quelle forme utiliser
    \begin{itemize}[label = \bcplume]
        \item On utilise principalement son fruit la \vocref{https://fr.wikipedia.org/wiki/Cenelle}{cénelle}.
    \end{itemize}
}
{%remarques
    \begin{Remarque}
        La \textbf{fleur} de l'aubépine n'est pas autorisée à être utilisée en tisane.\\

        L'aubépine est un cardiotonique, mais qui soigne aussi le cœur émotionnel, elle dé-serre le cœur au niveau \voc{subtil}.
    \end{Remarque}
}
{%image
    downloads/aubepine.jpg
}
{%titre photo
    Aubépine monogyne
}
{%description photo : "lieu - date"
    source : wikipedia
}


\subsection{Solanaceae}
\input{tex/botanique/solanaceae}

\subsection{Urticaceae}
\label{ortie}
\renewcommand{\cita}{
    \phantom{a}\citer{La vérité était que la vie nous avait jetés aux orties, l'un et l'autre\ldots\\ et c'est toujours ce qu'on appelle une rencontre.}{\href{https://fr.wikipedia.org/wiki/Romain_Gary}{Romain Gary}}

}
\ficheidentiteplante
{Ortie}
{%effet général
    L'\voc{ortie}, en latin \textit{Urtica dioica}, est une des plantes médicinales les plus \textbf{efficaces} et les plus \textbf{communes} que nous cotoyons.

    De manière générale, l'ortie a des effets reminéralisants, régénère le sang, stimule les \voc{fonctions digestives} par exemple en diminuant la \voc{glycémie}.\\ 
    Elle est \voc{fortifiante}, améliore la \voc{concentration} et aide à réduire l'\voc{anxiété}.\\
}
{%utilisation privilégiée
    L'ortie serait une \voc{plante adaptogène}. \\ Elle aidera le corps à réagir à de nombreuses situations de \voc{stress}.\\
    On l'utilise pour \voc{nettoyer}, \voc{récurrer}, \voc{lustrer} et \voc{patiner} grâce à sa teneur en \voc{silice}.\\

    Elle stimule la production de \voc{globules rouges} et aide à \voc{éliminer les bactéries} et en cas de maladie \voc{virale}.\\


    On peut utiliser la racine ou la plante entière pour obtenir des \textbf{teintes jaunes}.
}
{%infos cueillette
    \begin{itemize}[label=\faPen]
        \item On préférera utiliser des \textbf{jeunes pousses} car en vieillissant, l'ortie se charge en \voc{minéraux} pouvant provoquer des \voc{blocages rénaux}.
        \item On utilise la \textbf{plante entière}
        \item Cueillir au \textbf{printemps}
    \end{itemize}
}
{%sous quelle forme utiliser
    \begin{itemize}[label = \bcplume]
        \item En cure au \textbf{printemps} $\longrightarrow$ $2$L par jour
        \item En décoction pour un usage externe
        \item En \voc{teinture mère} pour un usage aussi bien interne que externe.
    \end{itemize}
}
{%remarques
\begin{multicols}{2}

    \boite{Usage interne :}{
        Doser $20$g de plante par litre d'eau et consommer jusqu'à $1$L par jour de cette tisane. \\
        On peut l'utiliser en cas de :
        \begin{itemize}[label = \bctrefle]
            \item \voc{Diabète}
            \item \voc{Anémie}
            \item \voc{Hépatite}
            \item \voc{Hémorragie utérine}
            \item \voc{Goutte}
            \item \voc{Rhumatismes}, arthrose
            \item \voc{Diarrhées}
            \item Troubles du foie et de la rate
            \item \voc{Ulcères} d'estomac ou intestinaux
        \end{itemize}
        \textbf{En teinture mère : }
                \begin{itemize}[label = \faPen]
                    \item Allergies : 20 gouttes $3$ fois par jour pendant 3 semaines.
                \end{itemize}
    }

    \columnbreak

    \boite{Usage externe :}{
        \begin{itemize}[label = \bcoeil]
            \item En décoction :
                \begin{itemize}[label = \faPen]
                    \item Angine ( gargarismes )
                    \item Affection de la peau ( eczéma, acnée, herpès )
                    \item nettoyage du cuir chevelu
                \end{itemize}
            \item En teinture mère : 
                \begin{itemize}[label = \faPen]
                    \item Appliquer des frictions ou cataplasmes sur les \voc{rhumatismes}, les \voc{entorses}, les \voc{sciatiques}
                    \item Nettoyage du cuir chevelu
                    \item En gargarismes dilué dans de l'eau pour les gingivites, infections, aphtes, angine.
                \end{itemize}
        \end{itemize}
    }

\end{multicols}
\boite{Au jardin :}{
        \begin{multicols}{2}
            Le \voc{purin} d'orties est un \voc{engrais} efficace
            \begin{itemize}[label = \faPen]
                \item Riche en \voc{azote}
                \item en éléments organiques
                \item en \voc{minéraux}
                \item en \voc{oligoéléments}
            \end{itemize}
            Il est \textbf{préventif} contre le \voc{mildiou} ou la \voc{rouille} ou l'\voc{oïdium}.\\
            On peut également l'utiliser comme \voc{insecticide}
            Voir \cite{laporte2023} page 103 pour plus de détails.
        \end{multicols}
    }

}
{%image
    ortie.jpg
}
{%titre photo
    Ortie
}
{%description photo : "lieu - date"
    Logis médiéval de Tessé - 29/07/2024 
}
\begin{Remarque}

    \bcattention Attention à la consommation des \textbf{graines} qui peuvent être toxiques au delà d'une certaine quantité.\\

    \bcattention La plante ne doit pas être consommée en cas d'oedème par rétention due à une insuffisance cardiaque ou rénale.\\

    \bcattention L'ortie peut \textbf{influencer} des traitements en cours aussi bien en les accélérant qu'en les ralentissant.\\
    Il est nécessaire de bien se renseigner lorsqu'on l'utilise en parallèle d'autres traitements.

    \bcattention \'Eviter l'ortie crue pendant la \textbf{grossesse}.
\end{Remarque}
\renewcommand{\cita}{}

\Potins{Ortie}{
    Dans la tradition celtique
    Plante de feu, l’ortie symbolise la force, l’énergie et le courage. Malgré son aspect rugueux et irritant, c’est une plante 
    d’une grande générosité, offrant la nourriture aux hommes et aux animaux, des vêtements, des remèdes.\\
    Elle est vénérée par les peuples celtes, symbolisant la foi et la persévérance qui nous sont nécessaires pour aller au- delà des 
    apparences, nous donnant alors accès à des trésors cachés.\\
    L’ortie s’est progressivement installée dans les clairières, permettant grâce à sa fibre extrêmement solide, de fabriquer des 
    cordes, des nasses et des tissus.\\

    Depuis très longtemps, la fibre d’ortie est utilisée pour faire du tissu (proche du chanvre). Les momies égyptiennes étaient entourées de bandelettes tissées de fibre d’ortie, et c’est sans doute grâce aux qualités de cette matière que les momies se sont conservées si longtemps. On y reviendra peut-être mais, actuellement, le travail de la fibre d’ortie étant difficile à mécaniser, personne
    n’y a plus recours.\\

    L’ortie serait une plante adaptogène. Autrement dit, elle aide le corps à s’adapter aux différents stress environnementaux et 
    psychologiques. Elle est reminéralisante.\\
    On l’utilise pour nettoyer, récurer, lustrer et patiner grâce à sa teneur en silice.\\
    On utilise la racine ou la plante entière pour obtenir des teintes jaunes. \\

    L’ortie pousse autour des maisons dans les endroits où l’homme a vécu. C’est un signe de bonne santé du sol, elle régularise le 
    fer et l’azote. Elle semble transformer les ondes négatives et assainir les terres.\\
    Elle aide à protéger les tomates et les pommes de terre contre le mildiou.\\

    Le voisinage de l’ortie augmente la teneur en huile essentielle des plantes médicinales. 
    Le suc de ces plantes voisines s’altère moins vite.\\
    Elle favorise la transformation des déchets en humus lorsqu’elle est ajoutée au compost (avant la floraison). 
    En couverture, elle protège et nourrit le sol.
}

voir \cite{laporte2023} pp 52-53
	\newpage
    \section{Recettes}

        \label{baume}
\subsection{Baumes}

    \begin{Defi}[Baume]
        Pour obtenir un \voc{baume}, on utilise le mélange d'un \voc{corps gras} avec de la \voc{cire d'abeille} utilisée comme agent texturant.\\

        Généralement, la masse de cire d'abeille utilisée correspond à $\dfrac{3}{10}$ de la masse de corps gras. 
    \end{Defi}


    \newcommand{\macerat}{\voc{macérat huileux}}
\newcommand{\macerats}{\voc{macérats huileux}}

    \ficherecette
    {%titre recette
        Baume à lèvres - anti-inflammatoire
    }
    {%liste d'ingrédients commençant par \item directement.
        \item $50$g de \voc{macérat huileux} de \voc{calendula}.
        \item $10$g de \voc{macérat huileux} d'\voc{achillée}\\\textbf{millefeuille}.
        \item $18$g de \voc{cire d'abeille}
        \item $6$g de \voc{miel}.
        \item $3$ pulvérisations d'\voc{eau florale de rose}.
        \item $1$ goutte de \voc{propolis}.
    }
    {%liste du matériel commençant par \item directement.
        \item $1$ \frquote{\voc{cul de poule}} propre. 
        \item $1$ fouet. 
        \item $1$ pot \textbf{désinfecté}, \voc{étanche} et \voc{sec}.
        \item $1$ spatule. 
        \item $1$ casserole et de l'eau pour le bain marie. 
        \item Des plaques chauffantes. 
    }
    {%Métnodes et conseils de conservation
        \begin{itemize}[label=\faPen]
            \item Conservation courte $\approx 6$ mois.
            \item Dans un endroit sec, de préférence à l'abri de la lumière. 
        \end{itemize}
    }
    {%Métnodes et conseils d'utilisation    
        \begin{itemize}[label=\faPen]
            \item En application locale sur la zone irritée.
            \item La préparation est commestible.
            \item Utiliser un \textbf{ustensile propre} lors de l'utilisation pour prolonger la durée de conservation. 
        \end{itemize}
    }
    {%Remarques
        Le dosage de cire d'abeille correspond à $\dfrac{3}{10}$ de la masse de \macerat.\\
        Le dosage de miel correspond à $\dfrac{1}{10}$ de la masse de \macerat.\\
        La \textbf{texture} doit correspondre à celle du \frquote{baume du tigre}.
    }
    {%chemin de l'illustration dans le dossier 'préparations'
        baume.jpg
    }
    {%Titre donné à l'illustration dans le document latex
        Baume au Calendula
    }
    {%Légende donnée à l'illustration dans le document latex
        31/07/2024
    }

    \newcommandx{\teinture}{\voc{teinture mère}}

\ficherecette
{%titre recette
    Baume pour la lignée féminine
}
{%liste d'ingrédients commençant par \item directement.
    \item $25$g de \voc{millepertuis}.
    \item $35$g d'\voc{achillée millefeuille}.
    \item $11$ gouttes de \teinture d'aubépine.
    \item $6$ pulvérisations d'\voc{eau de rose}.
    \item $8$g de \voc{cire d'abeille}.
}
{%liste du matériel commençant par \item directement.
    \item $1$ \frquote{\voc{cul de poule}} propre. 
    \item $1$ fouet et $1$ spatule. 
    \item $1$ pot \textbf{désinfecté}, \voc{étanche} et \voc{sec}.
    \item $1$ casserole et de l'eau pour le bain marie. 
    \item Des plaques chauffantes.  
}
{%Métnodes de préparation
    \item Verser les macérats huileux dans un cul de poule. 
    \item Ajouter la cire d'abeille \textbf{émiétée} pour faciliter son incorporation durant la \textbf{chauffe}.
    \item Faire chauffer au \textbf{bain-marie} en \textbf{remuant}.
    \item Une fois la cire d'abeille incorporée, sortir du bain-marie et ajouter les ingrédients restants. 
    \item Continuer de mélanger jusqu'à l'apparition d'une \textbf{mayonnaise} sur les rebords de la préparation. \\
            Cela caractérise un refroidissement suffisant.
    \item Verser dans un bocal hermétique, sec, et propre. 
    \item Laisser sécher pendant \textbf{24h} puis refermer le bocal. 
}
{%Métnodes et conseils d'utilisation    
    \textbf{Conseils de conservation :}
    \begin{itemize}[label=\faPen]
        \item Conservation courte $\approx 6$ mois.
        \item Dans un endroit sec, de préférence à l'abri de la lumière. 
    \end{itemize}
    \textbf{Conseils d'utilisation :}
    \begin{itemize}[label=\faPen]
        \item En application sur le \textbf{point de chakra} du \voc{plexus solaire}.
        \item Application possible sur le \textbf{premier point de chakra}.
        \item Utiliser un \textbf{ustensile propre} lors de l'utilisation pour prolonger la durée de conservation. 
    \end{itemize}
}
{%Remarques
    Le dosage de cire d'abeille correspond à $\dfrac{2}{10}$ de la masse de \macerat.\\
    Le dosage de miel correspond à $\dfrac{1}{10}$ de la masse de \macerat.\\
    La \textbf{texture} doit être un peu plus fluide que celle du \frquote{baume du tigre}.
}
{%chemin de l'illustration dans le dossier 'préparations'
    baume/baume_dessus_2.jpg    
}
{%Titre donné à l'illustration dans le document latex
    Baume pour la lignée féminine
}
{%Légende donnée à l'illustration dans le document latex
    01/08/2024
}

\subsection{Cuisine}

\label{onguent}
\subsection{Onguents}

    \begin{Defi}[Onguent]
        Pour obtenir un \voc{onguent}, on utilise le mélange d'un \voc{corps gras} avec de la \voc{cire d'abeille} utilisée comme agent texturant dans une proportion moindre part rapport au baume.\\

        Généralement, la masse de cire d'abeille utilisée correspond à $\dfrac{1}{10}$ de la masse de corps gras. 
    \end{Defi}

    \label{soleilcp}
\ficherecette
{%titre recette
    Crème après soleil
}
{%liste d'ingrédients commençant par \item directement.
    \item $25\%$ de \voc{macérat huileux} de \voc{calendula}
    \item $35\%$ de \voc{macérat huileux} de \voc{paquerette}
    \item $6\%$ de \voc{cire d'abeille}
    \item $6$g de \voc{miel}.
    \item $8$ pulvérisations d'\voc{eau florale de rose}.
    \item $1$ goutte de \voc{propolis}.
}
{%liste du matériel commençant par \item directement.
    \item $1$ \frquote{\voc{cul de poule}} propre. 
    \item $1$ fouet. 
    \item $1$ pot \textbf{désinfecté}, \voc{étanche} et \voc{sec}.
    \item $1$ spatule. 
    \item $1$ casserole et de l'eau pour le bain marie. 
    \item Des plaques chauffantes. 
}
{%Métnodes et conseils de conservation
    \begin{itemize}[label=\faPen]
        \item Conservation courte $\approx 6$ mois.
        \item Dans un endroit sec, de préférence à l'abri de la lumière. 
    \end{itemize}
}
{%Métnodes et conseils d'utilisation    
    \begin{itemize}[label=\faPen]
        \item En application locale sur la zone irritée.
        \item Utiliser un \textbf{ustensile propre} lors de l'utilisation pour prolonger la durée de conservation. 
    \end{itemize}
}
{%Remarques
    \textit{A priori} il n'y a pas de contre-indication pour être à nouveau au soleil après application contrairement à une préparation contenant du millepertuis.
}
{%chemin de l'illustration dans le dossier 'préparations'
    baume/pommades.jpg
}
{%Titre donné à l'illustration dans le document latex
    Crème au après solaire
}
{%Légende donnée à l'illustration dans le document latex
    Calendula - Paquerettes - 31/07/2024
}

    \label{hemorroides}
\ficherecette
{%titre recette
    Crème anti-hémorroïdes
}
{%liste d'ingrédients commençant par \item directement.
    \item $60$g de \voc{macérat huileux} de \voc{achillée millefeuille}
    \item $6$ gouttes de \voc{teinture mère} d'\voc{achillée millefeuille}
    \item $6 \text{ à }8$g de \voc{cire d'abeille}
    \item $6$g de \voc{miel}.
    \item $3$ pulv. d'\voc{eau florale de rose}.
}
{%liste du matériel commençant par \item directement.
    \item $1$ \frquote{\voc{cul de poule}} propre. 
    \item $1$ fouet. 
    \item $1$ pot \textbf{désinfecté}, \voc{étanche} et \voc{sec}.
    \item $1$ spatule. 
    \item $1$ casserole et de l'eau pour le bain marie. 
    \item Des plaques chauffantes. 
}
{%Métnodes de préparation
    \item Verser les macérats huileux dans un cul de poule. 
    \item Ajouter la cire d'abeille \textbf{émiétée} pour faciliter son incorporation durant la \textbf{chauffe}.
    \item Faire chauffer au \textbf{bain-marie} en \textbf{remuant}.
    \item Une fois la cire d'abeille incorporée, sortir du bain-marie et ajouter les ingrédients restants. 
    \item Continuer de mélanger jusqu'à l'apparition d'une \textbf{mayonnaise} sur les rebords de la préparation. \\
            Cela caractérise un refroidissement suffisant.
    \item Verser dans un bocal hermétique, sec, et propre. 
    \item Laisser sécher pendant \textbf{24h} puis refermer le bocal. 

}
{%Métnodes et conseils d'utilisation    
    \textbf{Conseils de conservation :}

    \begin{itemize}[label=\faPen]
        \item Conservation longue $\approx 2$ ans.
        \item Dans un endroit sec, de préférence à l'abri de la lumière. 
    \end{itemize}
    \textbf{Conseils d'utilisation :}

    \begin{itemize}[label=\faPen]
        %\item \voc{Diluer} dans de l'eau en respectant la \voc{posologie} ( voir \lien{teinture}{teintures mères} ).
        \item En application locale sur la zone irritée.
    \end{itemize}
}
{%Remarques
    Pour la teinture mère, \voc{diluer} dans de l'eau en respectant la \voc{posologie} ( voir \lien{teinture}{teintures mères} )
}
{%chemin de l'illustration dans le dossier 'préparations'
    downloads/tm_achillee.jpg
}
{%Titre donné à l'illustration dans le document latex
   Teinture mère d'Achillée millefeuille
}
{%Légende donnée à l'illustration dans le document latex
    source : internet
}

\subsection{Huiles}

\subsection{Tisanes}

\subsection{Vinaigres}

    \ficherecette
{%titre recette
    Vinaigre d'origan
}
{%liste d'ingrédients commençant par \item directement.
    \item 40g d'origan
    \item 1L de \textbf{vinaigre de cidre}
}
{%liste du matériel commençant par \item directement.
    \item Un bocal adapté à la taille de la cueillette.
    \item Une spatule pour tasser les plantes.

}
{%Métnodes et conseils de conservation
    Conserver pendant $28$ jour à l'abri du soleil. \\
    \voc{Dynamiser} chaque jour pour permettre la \voc{diffusion} des minéraux dans le vinaigre.
}
{%Métnodes et conseils d'utilisation    
    \textbf{En cure :}\\
    $1$ cuillère à soupe diluée dans de l'eau chaude.\\
    La préparation est à consommer \textbf{chaque matin} durant \textbf{trois semaines}.\\

    \textbf{Après une cure :}\\
    Arrêter de consommer pendant \textbf{une semaine}. \\
    Reprendre ensuite en \textbf{changeant de plante}.
}
{%Remarques
    Voir les détails de préparation dans la partie \lien{vinaigre}{préparation des vinaigres}.\\

    Il est utile de bien \voc{sécher les plantes}.
}
{%chemin de l'illustration dans le dossier 'préparations'
    vinaigre_plantes_internet.jpg
}
{%Titre donné à l'illustration dans le document latex
    Vinaigre de plantes
}
{%Légende donnée à l'illustration dans le document latex
    source : internet
}

	\newpage
    \section{Utilisations médicinales}

        

\subsection{Ballonnements / digestion}
    \subsubsection{Digestion difficile}

On pourra utiliser plusieurs plantes et recettes : \\

\begin{itemize}[label = \faPen]
    \item Macérat de genévrier : circulation lymphatique, travaille sur la digestion
\end{itemize}

\subsection{Brulûres}
    Dans le cas de brulures, on peut utiliser un \voc{macérat} de \lien{Millepertuis}{millepertuis}.



\subsection{Coupures}
    \voc{alcool}

\subsection{Douleurs de règles}

\subsection{Fragilité}
    On peut utiliser un macérat d'\lien{Ortie}{ortie}.\\
Permet de se minéraliser et prendre des \voc{vitamines}.


\subsection{Insomnie}
    \boite{Infusion de houblon contre l'insomnie :}{
    Pour lutter contre l'insomnie, on pourra utiliser une \voc{infusion} de cônes de \lien{houblon}{houblon}.\\
    Infuser $10$g de cônes pour $1$L d'eau.\\
    Consommer 1 à 2 tasses au coucher.
}
\subsection{Irritations}
    \subsubsection{Maux de gorge}
On pourra utiliser des \lien{huileux}{macérats huileux} avec des plantes comme :
\begin{itemize}[label=\bcfleur]
    \item La \vocref{https://fr.wikipedia.org/wiki/Ronce_commune}{ronce} : maux de gorges et extinction de voix
    \item L'huile d'\vocref{https://fr.wikipedia.org/wiki/Hysope}{hysope} : feuilles et fleurs contre l'extinction de voix
\end{itemize}

    \subsubsection{Hémorroïdes}

Utiliser une \lien{hemorroides}{crème anti-hémorroïdes} à l'achillée millefeuille.

\subsection{Migraines}


\subsection{Saignements}

Dans le cas d'un saignement, réaliser une pommade à base d'\voc{achillée millefeuille}. \\



    \newpage
    
    \section{Liste des 148}
        
\begin{Defi}[Liste des 148]
    La \voc{liste des 148} désigne la liste des plantes inscrites à la \vocref{https://fr.wikipedia.org/wiki/Pharmacop\%C3\%A9e}{pharmacopée}.
\end{Defi}

Dans le tableau ci-dessous, on peut trouver les informations sur les plantes inscrites dans cette liste : \\
\noindent\begin{tabularx}{\textwidth}{|X|X|X|X|X|}
\hline
\rowcolor{headerbg} \textcolor{white}{\textbf{Nom français}} & \textcolor{white}{\textbf{Nom latin}} & \textcolor{white}{\textbf{Famille}} & \textcolor{white}{\textbf{Parties utilisées}} & \textcolor{white}{\textbf{Forme de préparation}}  \\ \hline
\vocnoindexref{https://fr.wikipedia.org/wiki/Acacia}{Acacia à gomme} & Acacia senegal (L.) Willd. et autres espèces d’acacias d’origine africaine. & Fabaceae & Exsudation gommeuse = gomme arabique. & En l’état - En poudre - Extrait sec aqueux \\ \hline
\vocnoindexref{https://fr.wikipedia.org/wiki/Ache}{Ache des marais} & Apium graveolens L. & Apiaceae & Souche radicante. & En l’état - En poudre \\ \hline
\vocnoindexref{https://fr.wikipedia.org/wiki/Achillée}{Achillée millefeuille.Millefeuille} & Achillea millefolium L. & Asteraceae & Sommité fleurie. & En l’état \\ \hline
\vocnoindexref{https://fr.wikipedia.org/wiki/Agar-agar}{Agar-agar} & Gelidium sp., Euchema sp., Gracilaria sp. & Rhodophyceae & Mucilage = gélose. & En l’état - En poudre \\ \hline
\vocnoindexref{https://fr.wikipedia.org/wiki/Ail}{Ail} & Allium sativum L. & Liliaceae & Bulbe. & En l’état - En poudre \\ \hline
\vocnoindexref{https://fr.wikipedia.org/wiki/Airelle}{Airelle myrtille} & Voir : Myrtille. &  &  &  \\ \hline
\vocnoindexref{https://fr.wikipedia.org/wiki/Ajowan}{Ajowan} & Carum copticum Benth. et Hook. f.(= Psychotis ajowan DC.). & Apiaceae & Fruit. & En l’état - En poudre \\ \hline
\vocnoindexref{https://fr.wikipedia.org/wiki/Alchémille}{Alchémille} & Alchemilla vulgaris L. (sensu latiore). & Rosaceae & Partie aérienne. & En l’état \\ \hline
\vocnoindexref{https://fr.wikipedia.org/wiki/Alkékenge}{Alkékenge.Coqueret} & Physalis alkekengi L. & Solanaceae & Fruit. & En l’état \\ \hline
\vocnoindexref{https://fr.wikipedia.org/wiki/Alliaire}{Alliaire} & Sisymbrium alliaria Scop. & Brassicaceae & Plante entière. & En l’état - En poudre \\ \hline
\vocnoindexref{https://fr.wikipedia.org/wiki/Aloès}{Aloès des Barbades} & Aloe barbadensis Mill.(= Aloe vera L.). & Liliaceae & Mucilage. & En l’état - En poudre \\ \hline
\vocnoindexref{https://fr.wikipedia.org/wiki/Amandier}{Amandier doux} & Prunus dulcis (Mill.) D. Webb var. dulcis. & Rosaceae & Graine, graine mondée. & En l’état - En poudre \\ \hline
\vocnoindexref{https://fr.wikipedia.org/wiki/Ambrette}{Ambrette} & Hibiscus abelmoschus L. & Malvaceae & Graine. & En l’état - En poudre \\ \hline
\vocnoindexref{https://fr.wikipedia.org/wiki/Aneth}{Aneth} & Anethum graveolens L.(= Peucedanum graveolens Benth. et Hook.). & Apiaceae & Fruit. & En l’état - En poudre \\ \hline
\end{tabularx}
\newpage
\noindent\begin{tabularx}{\textwidth}{|X|X|X|X|X|}
\hline
\rowcolor{headerbg} \textcolor{white}{\textbf{Nom français}} & \textcolor{white}{\textbf{Nom latin}} & \textcolor{white}{\textbf{Famille}} & \textcolor{white}{\textbf{Parties utilisées}} & \textcolor{white}{\textbf{Forme de préparation}}  \\ \hline
\vocnoindexref{https://fr.wikipedia.org/wiki/Aneth}{Aneth fenouil} & Voir : Fenouil doux. &  &  &  \\ \hline
\vocnoindexref{https://fr.wikipedia.org/wiki/Angélique}{Angélique.Angélique officinale} & Angelica archangelica L.(= Archangelica officinalis Hoffm.). & Apiaceae & Fruit. & En l’état - En poudre \\ \hline
\vocnoindexref{https://fr.wikipedia.org/wiki/Anis}{Anis.Anis vert} & Pimpinella anisum L. & Apiaceae & Fruit. & En l’état - En poudre \\ \hline
\vocnoindexref{https://fr.wikipedia.org/wiki/Anis}{Anis étoilé} & Voir : Badianier de Chine. &  &  &  \\ \hline
\vocnoindexref{https://fr.wikipedia.org/wiki/Ascophyllum}{Ascophyllum} & Ascophyllum nodosum Le Jol. & Phaeophyceae & Thalle. & En l’état - En poudre - Extrait sec aqueux \\ \hline
\vocnoindexref{https://fr.wikipedia.org/wiki/Aspérule}{Aspérule odorante} & Galium odoratum (L.) Scop.(= Asperula odorata L.). & Rubiaceae & Partie aérienne fleurie. & En l’état \\ \hline
\vocnoindexref{https://fr.wikipedia.org/wiki/Lavandula_latifolia}{Aspic.Lavande aspic} & Lavandula latifolia (L. f.) Medik. & Lamiaceae & Sommité fleurie. & En l’état \\ \hline
\vocnoindexref{https://fr.wikipedia.org/wiki/Astragale_(flore)}{Astragale à gomme} & Astragalus gummifer (Labill.) et certaines espèces du genre Astragalus d’Asie occidentale. & Fabaceae & Exsudation gommeuse = gomme adragante. & En l’état - En poudre - Extrait sec aqueux \\ \hline
\vocnoindexref{https://fr.wikipedia.org/wiki/Aubépine}{Aubépine} & Crataegus laevigata (Poir.) DC.,C. monogyna Jacq. (Lindm.)(= C. oxyacanthoïdes Thuill.). & Rosaceae & Fruit. & En l’état \\ \hline
\vocnoindexref{https://fr.wikipedia.org/wiki/Aunée}{Aunée.Aunée officinale} & Inula helenium L. & Asteraceae & Partie souterraine. & En l’état - En poudre \\ \hline
\vocnoindexref{https://fr.wikipedia.org/wiki/Avoine}{Avoine} & Avena sativa L. & Poaceae & Fruit. & En l’état - En poudre \\ \hline
\end{tabularx}
\newpage
\noindent\begin{tabularx}{\textwidth}{|X|X|X|X|X|}
\hline
\rowcolor{headerbg} \textcolor{white}{\textbf{Nom français}} & \textcolor{white}{\textbf{Nom latin}} & \textcolor{white}{\textbf{Famille}} & \textcolor{white}{\textbf{Parties utilisées}} & \textcolor{white}{\textbf{Forme de préparation}}  \\ \hline
\vocnoindexref{https://fr.wikipedia.org/wiki/Balsamite}{Balsamite odorante.Menthe coq} & Balsamita major Desf.(= Chrysanthemum balsamita [L.] Baill.). & Asteraceae & Feuille, sommité fleurie. & En l’état \\ \hline
\vocnoindexref{https://fr.wikipedia.org/wiki/Bardane}{Bardane (grande)} & Arctium lappa L.(= A. majus [Gaertn.] Bernh.)(= Lappa major Gaertn.). & Asteraceae & Feuille, racine. & En l’état \\ \hline
\vocnoindexref{https://fr.wikipedia.org/wiki/Basilic}{Basilic.Basilic doux} & Ocimum basilicum L. & Lamiaceae & Feuille. & En l’état - En poudre \\ \hline
\vocnoindexref{https://fr.wikipedia.org/wiki/Baumier}{Baumier de Copahu.Baume de Copahu} & Copaifera officinalis L.,C. guyanensis Desf.,C. lansdorfii Desf. & Fabaceae & Oléo-résine dite baume de copahu » . & En l’état \\ \hline
\vocnoindexref{https://fr.wikipedia.org/wiki/Bétoine}{Bétoine} & Stachys officinalis (L.) Trevis.(= Betonica officinalis L.). & Lamiaceae & Feuille. & En l’état \\ \hline
\vocnoindexref{https://fr.wikipedia.org/wiki/Bigaradier}{Bigaradier} & Voir : Oranger amer. &  &  &  \\ \hline
\vocnoindexref{https://fr.wikipedia.org/wiki/Blé}{Blé} & Triticum aestivum L. et cultivars(= T. vulgare Host)(= T. sativum Lam.). & Poaceae & Son. & En l’état - En poudre \\ \hline
\vocnoindexref{https://fr.wikipedia.org/wiki/Bouillon}{Bouillon blanc} & Verbascum thapsus L.,V. densiflorum Bertol.(= V. thapsiforme Schrad.),V. phlomoides L. & Scrophulariaceae & Corolle mondée. & En l’état \\ \hline
\vocnoindexref{https://fr.wikipedia.org/wiki/Bourrache}{Bourrache} & Borago officinalis L. & Boraginaceae & Fleur. & En l’état \\ \hline
\vocnoindexref{https://fr.wikipedia.org/wiki/Bruyère}{Bruyère cendrée} & Erica cinerea L. & Ericaceae & Fleur. & En l’état \\ \hline
\vocnoindexref{https://fr.wikipedia.org/wiki/Camomille}{Camomille allemande} & Voir : Matricaire. &  &  &  \\ \hline
\vocnoindexref{https://fr.wikipedia.org/wiki/Camomille}{Camomille romaine} & Chamaemelum nobile (L.) All.(= Anthemis nobilis L.). & Asteraceae & Capitule. & En l’état \\ \hline
\vocnoindexref{https://fr.wikipedia.org/wiki/Camomille}{Camomille vulgaire} & Voir : Matricaire. &  &  &  \\ \hline
\vocnoindexref{https://fr.wikipedia.org/wiki/Canéficier}{Canéficier} & Cassia fistula L. & Fabaceae & Pulpe de fruit. & En l’état \\ \hline
\end{tabularx}
\newpage
\noindent\begin{tabularx}{\textwidth}{|X|X|X|X|X|}
\hline
\rowcolor{headerbg} \textcolor{white}{\textbf{Nom français}} & \textcolor{white}{\textbf{Nom latin}} & \textcolor{white}{\textbf{Famille}} & \textcolor{white}{\textbf{Parties utilisées}} & \textcolor{white}{\textbf{Forme de préparation}}  \\ \hline
\vocnoindexref{https://fr.wikipedia.org/wiki/Cannelier}{Cannelier de Ceylan.Cannelle de Ceylan} & Cinnamomum zeylanicum Nees. & Lauraceae & Ecorce de tige raclée = cannelle de Ceylan. & En l’état - En poudre \\ \hline
\vocnoindexref{https://fr.wikipedia.org/wiki/Cannelier}{Cannelier de Chine.Cannelle de Chine} & Cinnamomum aromaticum Nees,C. cassia Nees ex Blume. & Lauraceae & Ecorce de tige = cannelle de Chine. & En l’état - En poudre \\ \hline
\vocnoindexref{https://fr.wikipedia.org/wiki/Capucine}{Capucine} & Tropaeolum majus L. & Tropaeolaceae & Feuille. & En l’état \\ \hline
\vocnoindexref{https://fr.wikipedia.org/wiki/Cardamome}{Cardamome} & Elettaria cardamomum (L.) Maton. & Zingiberaceae & Fruit. & En l’état - En poudre \\ \hline
\vocnoindexref{https://fr.wikipedia.org/wiki/Caroubier}{Caroubier.Gomme caroube} & Ceratonia siliqua L. & Fabaceae & Graine mondée = gomme caroube. & En l’état - En poudre \\ \hline
\vocnoindexref{https://fr.wikipedia.org/wiki/Carragaheen}{Carragaheen.Mousse d’Irlande} & Chondrus crispus Lingby. & Gigartinaceae & Thalle. & En l’état \\ \hline
\vocnoindexref{https://fr.wikipedia.org/wiki/Carthame}{Carthame} & Carthamus tinctorius L. & Asteraceae & Fleur. & En l’état \\ \hline
\vocnoindexref{https://fr.wikipedia.org/wiki/Carvi}{Carvi.Cumin des prés} & Carum carvi L. & Apiaceae & Fruit. & En l’état - En poudre \\ \hline
\vocnoindexref{https://fr.wikipedia.org/wiki/Cassissier}{Cassissier.Groseiller noir} & Ribes nigrum L. & Grossulariaceae & Feuille, fruit. & En l’état \\ \hline
\vocnoindexref{https://fr.wikipedia.org/wiki/Centaurée}{Centaurée (petite)} & Centaurium erythraea Raf.(= Erythraea centaurium [L.] Persoon)(= C. minus Moench)(= C. umbellatum Gilib.). & Gentianaceae & Sommité fleurie. & En l’état \\ \hline
\vocnoindexref{https://fr.wikipedia.org/wiki/Cerisier}{Cerisier griottier} & Voir : Griottier. &  &  &  \\ \hline
\vocnoindexref{https://fr.wikipedia.org/wiki/Chicorée}{Chicorée} & Cichorium intybus L. & Asteraceae & Feuille, racine. & En l’état \\ \hline
\vocnoindexref{https://fr.wikipedia.org/wiki/Chiendent}{Chiendent (gros).Chiendent pied de poule} & Cynodon dactylon (L.) Pers. & Poaceae & Rhizome. & En l’état \\ \hline
\vocnoindexref{https://fr.wikipedia.org/wiki/Chiendent}{Chiendent.Chiendent (petit)} & Elytrigia repens [L.] Desv. ex Nevski(= Agropyron repens [L.] Beauv.)(= Elymus repens [L.] Goudl.). & Poaceae & Rhizome. & En l’état \\ \hline
\vocnoindexref{https://fr.wikipedia.org/wiki/Citronnelles}{Citronnelles} & Cymbopogon sp. & Poaceae & Feuille. & En l’état - En poudre \\ \hline
\end{tabularx}
\newpage
\noindent\begin{tabularx}{\textwidth}{|X|X|X|X|X|}
\hline
\rowcolor{headerbg} \textcolor{white}{\textbf{Nom français}} & \textcolor{white}{\textbf{Nom latin}} & \textcolor{white}{\textbf{Famille}} & \textcolor{white}{\textbf{Parties utilisées}} & \textcolor{white}{\textbf{Forme de préparation}}  \\ \hline
\vocnoindexref{https://fr.wikipedia.org/wiki/Citrouille}{Citrouille} & Voir : Courge citrouille. &  &  &  \\ \hline
\vocnoindexref{https://fr.wikipedia.org/wiki/Clou}{Clou de girofle} & Voir : Giroflier. &  &  &  \\ \hline
\vocnoindexref{https://fr.wikipedia.org/wiki/Cochléaire}{Cochléaire} & Cochlearia officinalis L. & Brassicaceae & Feuille. & En l’état \\ \hline
\vocnoindexref{https://fr.wikipedia.org/wiki/Colatier}{Colatier} & Voir : Kolatier. &  &  &  \\ \hline
\vocnoindexref{https://fr.wikipedia.org/wiki/Coquelicot}{Coquelicot} & Papaver rhoeas L.,P. dubium L. & Papaveraceae & Pétale. & En l’état \\ \hline
\vocnoindexref{https://fr.wikipedia.org/wiki/Coqueret}{Coqueret} & Voir : Alkékenge. &  &  &  \\ \hline
\vocnoindexref{https://fr.wikipedia.org/wiki/Coriandre}{Coriandre} & Coriandrum sativum L. & Apiaceae & Fruit. & En l’état - En poudre \\ \hline
\vocnoindexref{https://fr.wikipedia.org/wiki/Courge}{Courge citrouille.Citrouille} & Cucurbita pepo L.. & Cucurbitaceae & Graine. & En l’état \\ \hline
\vocnoindexref{https://fr.wikipedia.org/wiki/Courge}{Courge.Potiron} & Cucurbita maxima Lam. & Cucurbitaceae & Graine. & En l’état \\ \hline
\vocnoindexref{https://fr.wikipedia.org/wiki/Criste}{Criste marine.Perce-pierre} & Crithmum maritimum L.. & Apiaceae & Partie aérienne. & En l’état \\ \hline
\vocnoindexref{https://fr.wikipedia.org/wiki/Cumin}{Cumin des prés} & Voir : Carvi. &  &  &  \\ \hline
\vocnoindexref{https://fr.wikipedia.org/wiki/Curcuma}{Curcuma long} & Curcuma domestica Vahl(= C. longa L.). & Zingiberaceae & Rhizome. & En l’état - En poudre \\ \hline
\vocnoindexref{https://fr.wikipedia.org/wiki/Cyamopsis}{Cyamopsis.Gomme guar.Guar} & Cyamopsis tetragonolobus (L.) Taub. & Fabaceae & Graine mondée = gomme guar. & En l’état - En poudre - Extrait sec aqueux \\ \hline
\vocnoindexref{https://fr.wikipedia.org/wiki/Eglantier}{Eglantier.Cynorrhodon.Rosier sauvage} & Rosa canina L., R. pendulina L. et autres espèces de Rosa. & Rosaceae & Pseudo-fruit = cynorrhodon. & En l’état \\ \hline
\vocnoindexref{https://fr.wikipedia.org/wiki/Eleuthérocoque}{Eleuthérocoque} & Eleutherococcus senticosus Maxim. & Araliaceae & Partie souterraine. & En l’état \\ \hline
\end{tabularx}
\newpage
\noindent\begin{tabularx}{\textwidth}{|X|X|X|X|X|}
\hline
\rowcolor{headerbg} \textcolor{white}{\textbf{Nom français}} & \textcolor{white}{\textbf{Nom latin}} & \textcolor{white}{\textbf{Famille}} & \textcolor{white}{\textbf{Parties utilisées}} & \textcolor{white}{\textbf{Forme de préparation}}  \\ \hline
\vocnoindexref{https://fr.wikipedia.org/wiki/Estragon}{Estragon} & Artemisia dracunculus L. & Asteraceae & Partie aérienne. & En l’état - En poudre \\ \hline
\vocnoindexref{https://fr.wikipedia.org/wiki/Eucalyptus}{Eucalyptus.Eucalyptus globuleux} & Eucalyptus globulus Labill. & Myrtaceae & Feuille. & En l’état \\ \hline
\vocnoindexref{https://fr.wikipedia.org/wiki/Fenouil}{Fenouil amer} & Foeniculum vulgare Mill. var. vulgare. & Apiaceae & Fruit. & En l’état - En poudre \\ \hline
\vocnoindexref{https://fr.wikipedia.org/wiki/Fenouil}{Fenouil doux.Aneth fenouil} & Foeniculum vulgare Mill. var. dulcis. & Apiaceae & Fruit. & En l’état - En poudre \\ \hline
\vocnoindexref{https://fr.wikipedia.org/wiki/Fenugrec}{Fenugrec} & Trigonella foenum-graecum L. & Fabaceae & Graine. & En l’état - En poudre \\ \hline
\vocnoindexref{https://fr.wikipedia.org/wiki/Févier}{Févier} & Voir : Gléditschia. &  &  &  \\ \hline
\vocnoindexref{https://fr.wikipedia.org/wiki/Figuier}{Figuier} & Ficus carica L. & Moraceae & Pseudo-fruit. & En l’état \\ \hline
\vocnoindexref{https://fr.wikipedia.org/wiki/Frêne}{Frêne} & Fraxinus excelsior L.,F. oxyphylla M. Bieb. & Oleaceae & Feuille. & En l’état \\ \hline
\vocnoindexref{https://fr.wikipedia.org/wiki/Frêne}{Frêne à manne} & Fraxinus ornus L. & Oleaceae & Suc épaissi dit manne ». & En l’état - En poudre \\ \hline
\vocnoindexref{https://fr.wikipedia.org/wiki/Fucus}{Fucus} & Fucus serratus L.,F. vesiculosus L. & Fucaceae & Thalle. & En l’état - En poudre \\ \hline
\end{tabularx}
\newpage
\noindent\begin{tabularx}{\textwidth}{|X|X|X|X|X|}
\hline
\rowcolor{headerbg} \textcolor{white}{\textbf{Nom français}} & \textcolor{white}{\textbf{Nom latin}} & \textcolor{white}{\textbf{Famille}} & \textcolor{white}{\textbf{Parties utilisées}} & \textcolor{white}{\textbf{Forme de préparation}}  \\ \hline
\vocnoindexref{https://fr.wikipedia.org/wiki/Galanga}{Galanga (petit)} & Alpinia officinarum Hance. & Zingiberaceae & Rhizome. & En l’état - En poudre \\ \hline
\vocnoindexref{https://fr.wikipedia.org/wiki/Genévrier}{Genévrier.Genièvre} & Juniperus communis L. & Cupressaceae & Cône femelle dit baie de genièvre ». & En l’état \\ \hline
\vocnoindexref{https://fr.wikipedia.org/wiki/Gentiane}{Gentiane.Gentiane jaune} & Gentiana lutea L. & Gentianaceae & Partie souterraine. & En l’état - En poudre \\ \hline
\vocnoindexref{https://fr.wikipedia.org/wiki/Gingembre}{Gingembre} & Zingiber officinale Roscoe. & Zingiberaceae & Rhizome. & En l’état - En poudre \\ \hline
\vocnoindexref{https://fr.wikipedia.org/wiki/Ginseng}{Ginseng.Panax de Chine} & Panax ginseng C.A. Meyer(= Aralia quinquefolia Decne. et Planch.). & Araliaceae & Partie souterraine. & En l’état - En poudre - Extrait sec aqueux \\ \hline
\vocnoindexref{https://fr.wikipedia.org/wiki/Giroflier}{Giroflier} & Syzygium aromaticum (L.) Merr. et Perry(= Eugenia caryophyllus (Sprengel) Bull. et Harr.). & Myrtaceae & Bouton floral = clou de girofle. & En l’état - En poudre \\ \hline
\vocnoindexref{https://fr.wikipedia.org/wiki/Gléditschia}{Gléditschia.Févier} & Gleditschia triacanthos L.,G. ferox Desf. & Fabaceae & Graine. & En l’état - En poudre - Extrait sec aqueux \\ \hline
\vocnoindexref{https://fr.wikipedia.org/wiki/Gomme}{Gomme adragante} & Voir : Astragale à gomme. &  &  &  \\ \hline
\vocnoindexref{https://fr.wikipedia.org/wiki/Gomme}{Gomme arabique} & Voir : Acacia à gomme. &  &  &  \\ \hline
\vocnoindexref{https://fr.wikipedia.org/wiki/Gomme}{Gomme caroube} & Voir : Caroubier. &  &  &  \\ \hline
\vocnoindexref{https://fr.wikipedia.org/wiki/Gomme}{Gomme de sterculia} & Voir : Sterculia. &  &  &  \\ \hline
\vocnoindexref{https://fr.wikipedia.org/wiki/Gomme}{Gomme guar} & Voir : Cyamopsis. &  &  &  \\ \hline
\vocnoindexref{https://fr.wikipedia.org/wiki/Gomme}{Gomme Karaya} & Voir : Sterculia. &  &  &  \\ \hline
\vocnoindexref{https://fr.wikipedia.org/wiki/Gomme}{Gomme M’Bep} & Voir : Sterculia. &  &  &  \\ \hline
\end{tabularx}
\newpage
\noindent\begin{tabularx}{\textwidth}{|X|X|X|X|X|}
\hline
\rowcolor{headerbg} \textcolor{white}{\textbf{Nom français}} & \textcolor{white}{\textbf{Nom latin}} & \textcolor{white}{\textbf{Famille}} & \textcolor{white}{\textbf{Parties utilisées}} & \textcolor{white}{\textbf{Forme de préparation}}  \\ \hline
\vocnoindexref{https://fr.wikipedia.org/wiki/Griottier.cerisier}{Griottier.Cerisier griottier.Queue de cerise} & Prunus cerasus L.,P. avium (L.) L. & Rosaceae & Pédoncule du fruit = queue de cerise. & En l’état \\ \hline
\vocnoindexref{https://fr.wikipedia.org/wiki/Groseiller}{Groseiller noir} & Voir : Cassissier. &  &  &  \\ \hline
\vocnoindexref{https://fr.wikipedia.org/wiki/Guar}{Guar} & Voir : Cyamopsis. &  &  &  \\ \hline
\vocnoindexref{https://fr.wikipedia.org/wiki/Guarana}{Guarana} & Voir : Paullinia. &  &  &  \\ \hline
\vocnoindexref{https://fr.wikipedia.org/wiki/Guimauve}{Guimauve} & Althaea officinalis L. & Malvaceae & Feuille, fleur, racine. & En l’état - En poudre (racine) \\ \hline
\vocnoindexref{https://fr.wikipedia.org/wiki/Hibiscus}{Hibiscus} & Voir : Karkadé. &  &  &  \\ \hline
\vocnoindexref{https://fr.wikipedia.org/wiki/Houblon}{Houblon} & Humulus lupulus L. & Cannabaceae & Inflorescence femelle dite cône de houblon ». & En l’état \\ \hline
\vocnoindexref{https://fr.wikipedia.org/wiki/Jujubier}{Jujubier} & Ziziphus jujuba Mill.(= Z. sativa Gaertn.)(= Z. vulgaris Lam.)(= Rhamnus zizyphus L.). & Rhamnaceae & Fruit privé de graines. & En l’état \\ \hline
\vocnoindexref{https://fr.wikipedia.org/wiki/Karkadé.oseille}{Karkadé.Oseille de Guinée.Hibiscus} & Hibiscus sabdariffa L. & Malvaceae & Calice et calicule. & En l’état \\ \hline
\vocnoindexref{https://fr.wikipedia.org/wiki/Kolatier}{Kolatier.Colatier.Kola} & Cola acuminata (P. Beauv.) Schott et Endl.(= Sterculia acuminata P. Beauv.),C. nitida (Vent.) Schott et Endl.(= C. vera K. Schum.) et variétés. & Sterculiaceae & Amande dite noix de kola ». & En l’état - En poudre \\ \hline
\vocnoindexref{https://fr.wikipedia.org/wiki/Lamier}{Lamier blanc.Ortie blanche} & Lamium album L. & Lamiaceae & Corolle mondée, sommité fleurie. & En l’état \\ \hline
\vocnoindexref{https://fr.wikipedia.org/wiki/Laminaire}{Laminaire} & Laminaria digitata J.P. Lamour.,L. hyperborea (Gunnerus) Foslie,L. cloustonii Le Jol. & Laminariaceae & Stipe, thalle. & En l’état - Extrait sec aqueux (thalle) \\ \hline
\vocnoindexref{https://fr.wikipedia.org/wiki/Laurier}{Laurier commun.Laurier sauce} & Laurus nobilis L. & Lauraceae & Feuille. & En l’état - En poudre \\ \hline
\vocnoindexref{https://fr.wikipedia.org/wiki/Lavande}{Lavande.Lavande vraie} & Lavandula angustifolia Mill.(= L. vera DC.). & Lamiaceae & Fleur, sommité fleurie. & En l’état \\ \hline
\vocnoindexref{https://fr.wikipedia.org/wiki/Lavande}{Lavande aspic} & Voir : Aspic. &  &  &  \\ \hline
\end{tabularx}
\newpage
\noindent\begin{tabularx}{\textwidth}{|X|X|X|X|X|}
\hline
\rowcolor{headerbg} \textcolor{white}{\textbf{Nom français}} & \textcolor{white}{\textbf{Nom latin}} & \textcolor{white}{\textbf{Famille}} & \textcolor{white}{\textbf{Parties utilisées}} & \textcolor{white}{\textbf{Forme de préparation}}  \\ \hline
\vocnoindexref{https://fr.wikipedia.org/wiki/Lavande}{Lavande stoechas} & Lavandula stoechas L. & Lamiaceae & Fleur, sommité fleurie. & En l’état \\ \hline
\vocnoindexref{https://fr.wikipedia.org/wiki/Lavande}{Lavande vraie} & Voir : Lavande. &  &  &  \\ \hline
\vocnoindexref{https://fr.wikipedia.org/wiki/Lavandin}{Lavandin Grosso »} & Lavandula × intermedia Emeric ex Loisel. & Lamiaceae & Fleur, sommité fleurie. & En l’état \\ \hline
\vocnoindexref{https://fr.wikipedia.org/wiki/Lemongrass}{Lemongrass de l’Amérique centrale} & Cymbopogon citratus (DC.) Stapf. & Poaceae & Feuille. & En l’état - En poudre \\ \hline
\vocnoindexref{https://fr.wikipedia.org/wiki/Lemongrass}{Lemongrass de l’Inde} & Cymbopogon flexuosus (Nees ex Steud.) J.F. Wats. & Poaceae & Feuille. & En l’état - En poudre \\ \hline
\vocnoindexref{https://fr.wikipedia.org/wiki/Lichen}{Lichen d’Islande} & Cetraria islandica (L.) Ach. sensu latiore. & Parmeliaceae & Thalle. & En l’état \\ \hline
\vocnoindexref{https://fr.wikipedia.org/wiki/Lierre}{Lierre terrestre} & Glechoma hederacea L.(= Nepeta glechoma Benth.). & Lamiaceae & Partie aérienne fleurie. & En l’état \\ \hline
\vocnoindexref{https://fr.wikipedia.org/wiki/Lin}{Lin} & Linum usitatissimum L. & Linaceae & Graine. & En l’état - En poudre \\ \hline
\vocnoindexref{https://fr.wikipedia.org/wiki/Livèche}{Livèche} & Levisticum officinale Koch. & Apiaceae & Feuille, fruit, partie souterraine. & En l’état - En poudre \\ \hline
\end{tabularx}
\newpage
\noindent\begin{tabularx}{\textwidth}{|X|X|X|X|X|}
\hline
\rowcolor{headerbg} \textcolor{white}{\textbf{Nom français}} & \textcolor{white}{\textbf{Nom latin}} & \textcolor{white}{\textbf{Famille}} & \textcolor{white}{\textbf{Parties utilisées}} & \textcolor{white}{\textbf{Forme de préparation}}  \\ \hline
\vocnoindexref{https://fr.wikipedia.org/wiki/Marjolaine.origan}{Marjolaine.Origan marjolaine} & Origanum majorana L.(= Majorana hortensis Moench). & Lamiaceae & Feuille, sommité fleurie. & En l’état - En poudre \\ \hline
\vocnoindexref{https://fr.wikipedia.org/wiki/Maté}{Maté.Thé du Paraguay} & Ilex paraguariensis St.-Hil.(= I. paraguayensis Lamb.). & Aquifoliaceae & Feuille. & En l’état - Extrait sec aqueux \\ \hline
\vocnoindexref{https://fr.wikipedia.org/wiki/Matricaire.camomille}{Matricaire.Camomille allemande.Camomille vulgaire} & Matricaria recutita L.(= Chamomilla recutita [L.] Rausch.)(= M. chamomilla L.). & Asteraceae & Capitule. & En l’état \\ \hline
\vocnoindexref{https://fr.wikipedia.org/wiki/Mauve}{Mauve} & Malva sylvestris L. & Malvaceae & Feuille, fleur. & En l’état \\ \hline
\vocnoindexref{https://fr.wikipedia.org/wiki/Mélisse}{Mélisse} & Melissa officinalis L. & Lamiaceae & Feuille, sommité fleurie. & En l’état \\ \hline
\vocnoindexref{https://fr.wikipedia.org/wiki/Menthe}{Menthe coq} & Voir : Balsamite odorante. &  &  &  \\ \hline
\vocnoindexref{https://fr.wikipedia.org/wiki/Menthe}{Menthe poivrée} & Mentha × piperita L. & Lamiaceae & Feuille, sommité fleurie. & En l’état \\ \hline
\vocnoindexref{https://fr.wikipedia.org/wiki/Menthe}{Menthe verte} & Mentha spicata L. (= M. viridis L.). & Lamiaceae & Feuille, sommité fleurie. & En l’état \\ \hline
\vocnoindexref{https://fr.wikipedia.org/wiki/Ményanthe}{Ményanthe.Trèfle d’eau} & Menyanthes trifoliata L. & Menyanthaceae & Feuille. & En l’état \\ \hline
\vocnoindexref{https://fr.wikipedia.org/wiki/Millefeuille}{Millefeuille} & Voir : Achillée millefeuille. &  &  &  \\ \hline
\vocnoindexref{https://fr.wikipedia.org/wiki/Mousse}{Mousse d’Irlande} & Voir : Carragaheen. &  &  &  \\ \hline
\vocnoindexref{https://fr.wikipedia.org/wiki/Moutarde}{Moutarde junciforme} & Brassica juncea (L.) Czern. & Brassicaceae & Graine. & En l’état - En poudre \\ \hline
\vocnoindexref{https://fr.wikipedia.org/wiki/Muscadier}{Muscadier aromatique.Macis.Muscade} & Myristica fragrans Houtt.(= M. moschata Thunb.). & Myristicaceae & Graine dite muscade » ou noix de muscade », arille dite macis ». & En l’état - En poudre (graine) \\ \hline
\vocnoindexref{https://fr.wikipedia.org/wiki/Myrte}{Myrte} & Myrtus communis L. & Myrtaceae & Feuille. & En l’état \\ \hline
\end{tabularx}
\newpage
\noindent\begin{tabularx}{\textwidth}{|X|X|X|X|X|}
\hline
\rowcolor{headerbg} \textcolor{white}{\textbf{Nom français}} & \textcolor{white}{\textbf{Nom latin}} & \textcolor{white}{\textbf{Famille}} & \textcolor{white}{\textbf{Parties utilisées}} & \textcolor{white}{\textbf{Forme de préparation}}  \\ \hline
\vocnoindexref{https://fr.wikipedia.org/wiki/Myrtille}{Myrtille.Airelle myrtille} & Vaccinium myrtillus L. & Ericaceae & Feuille, fruit. & En l’état \\ \hline
\vocnoindexref{https://fr.wikipedia.org/wiki/Olivier}{Olivier} & Olea europaea L. & Oleaceae & Feuille. & En l’état \\ \hline
\vocnoindexref{https://fr.wikipedia.org/wiki/Oranger}{Oranger amer.Bigaradier} & Citrus aurantium L.(= C. bigaradia Duch.)(= C. vulgaris Risso). & Rutaceae & Feuille, fleur, péricarpe dit écorce » ou zeste. & En l’état - En poudre (péricarpe) \\ \hline
\vocnoindexref{https://fr.wikipedia.org/wiki/Oranger}{Oranger doux} & Citrus sinensis (L.) Pers.(= C. aurantium L.). & Rutaceae & Péricarpe dit écorce » ou zeste. & En l’état - En poudre \\ \hline
\vocnoindexref{https://fr.wikipedia.org/wiki/Origan}{Origan} & Origanum vulgare L. & Lamiaceae & Feuille, sommité fleurie. & En l’état - En poudre \\ \hline
\vocnoindexref{https://fr.wikipedia.org/wiki/Origan}{Origan marjolaine} & Voir : Marjolaine. &  &  &  \\ \hline
\vocnoindexref{https://fr.wikipedia.org/wiki/Ortie}{Ortie blanche} & Voir : Lamier blanc. &  &  &  \\ \hline
\vocnoindexref{https://fr.wikipedia.org/wiki/Ortie}{Ortie brûlante} & Urtica urens L. & Urticaceae & Partie aérienne. & En l’état \\ \hline
\vocnoindexref{https://fr.wikipedia.org/wiki/Ortie}{Ortie dioïque} & Urtica dioica L. & Urticaceae & Partie aérienne. & En l’état \\ \hline
\vocnoindexref{https://fr.wikipedia.org/wiki/Oseille}{Oseille de Guinée} & Voir : Karkadé. &  &  &  \\ \hline
\vocnoindexref{https://fr.wikipedia.org/wiki/Panax}{Panax de Chine} & Voir : Ginseng. &  &  &  \\ \hline
\vocnoindexref{https://fr.wikipedia.org/wiki/Papayer}{Papayer} & Carica papaya L. & Caricaceae & Suc du fruit, feuille. & En l’état - En poudre (suc du fruit) \\ \hline
\vocnoindexref{https://fr.wikipedia.org/wiki/Passerose}{Passerose} & Voir : Rose trémière. &  &  &  \\ \hline
\vocnoindexref{https://fr.wikipedia.org/wiki/Paullinia}{Paullinia.Guarana} & Paullinia cupana Kunth.(= P. sorbilis Mart.). & Sapindaceae & Graine, extrait préparé avec la graine = guarana. & En l’état - En poudre (extrait) \\ \hline
\vocnoindexref{https://fr.wikipedia.org/wiki/Pensée}{Pensée sauvage.Violette tricolore} & Viola arvensis Murray,V. tricolor L. & Violaceae & Fleur, partie aérienne fleurie. & En l’état \\ \hline
\end{tabularx}
\newpage
\noindent\begin{tabularx}{\textwidth}{|X|X|X|X|X|}
\hline
\rowcolor{headerbg} \textcolor{white}{\textbf{Nom français}} & \textcolor{white}{\textbf{Nom latin}} & \textcolor{white}{\textbf{Famille}} & \textcolor{white}{\textbf{Parties utilisées}} & \textcolor{white}{\textbf{Forme de préparation}}  \\ \hline
\vocnoindexref{https://fr.wikipedia.org/wiki/Perce-pierre}{Perce-pierre} & Voir : Criste marine. &  &  &  \\ \hline
\vocnoindexref{https://fr.wikipedia.org/wiki/Piment}{Piment de Cayenne.Piment enragé.Piment (petit)} & Capsicum frutescens L. & Solanaceae & Fruit. & En l’état - En poudre \\ \hline
\vocnoindexref{https://fr.wikipedia.org/wiki/Pin}{Pin sylvestre} & Pinus sylvestris L. & Pinaceae & Bourgeon. & En l’état \\ \hline
\vocnoindexref{https://fr.wikipedia.org/wiki/Pissenlit.dent}{Pissenlit.Dent de lion} & Taraxacum officinale Web. & Asteraceae & Feuille, partie aérienne. & En l’état \\ \hline
\vocnoindexref{https://fr.wikipedia.org/wiki/Pommier}{Pommier} & Malus sylvestris Mill.(= Pyrus malus L.). & Rosaceae & Fruit. & En l’état \\ \hline
\vocnoindexref{https://fr.wikipedia.org/wiki/Potiron}{Potiron} & Voir : Courge. &  &  &  \\ \hline
\vocnoindexref{https://fr.wikipedia.org/wiki/Prunier}{Prunier} & Prunus domestica L. & Rosaceae & Fruit. & En l’état \\ \hline
\vocnoindexref{https://fr.wikipedia.org/wiki/Queue}{Queue de cerise} & Voir : Griottier. &  &  &  \\ \hline
\vocnoindexref{https://fr.wikipedia.org/wiki/Radis}{Radis noir} & Raphanus sativus L. var. niger (Mill.) Kerner. & Brassicaceae & Racine. & En l’état \\ \hline
\vocnoindexref{https://fr.wikipedia.org/wiki/Raifort}{Raifort sauvage} & Armoracia rusticana Gaertn., B. Mey. et Scherb.(= Cochlearia armoracia L.). & Brassicaceae & Racine. & En l’état - En poudre \\ \hline
\vocnoindexref{https://fr.wikipedia.org/wiki/Réglisse}{Réglisse} & Glycyrrhiza glabra L. & Fabaceae & Partie souterraine. & En l’état - En poudre - Extrait sec aqueux \\ \hline
\vocnoindexref{https://fr.wikipedia.org/wiki/Reine-des-prés}{Reine-des-prés.Ulmaire} & Filipendula ulmaria (L.) Maxim.(= Spiraea ulmaria L.). & Rosaceae & Fleur, sommité fleurie. & En l’état \\ \hline
\vocnoindexref{https://fr.wikipedia.org/wiki/Romarin}{Romarin} & Rosmarinus officinalis L. & Lamiaceae & Feuille, sommité fleurie. & En l’état - En poudre \\ \hline
\vocnoindexref{https://fr.wikipedia.org/wiki/Ronce}{Ronce} & Rubus sp. & Rosaceae & Feuille. & En l’état \\ \hline
\vocnoindexref{https://fr.wikipedia.org/wiki/Rose}{Rose trémière.Passerose} & Alcea rosea L.(= Althaea rosea L.). & Malvaceae & Fleur. & En l’état \\ \hline
\end{tabularx}
\newpage
\noindent\begin{tabularx}{\textwidth}{|X|X|X|X|X|}
\hline
\rowcolor{headerbg} \textcolor{white}{\textbf{Nom français}} & \textcolor{white}{\textbf{Nom latin}} & \textcolor{white}{\textbf{Famille}} & \textcolor{white}{\textbf{Parties utilisées}} & \textcolor{white}{\textbf{Forme de préparation}}  \\ \hline
\vocnoindexref{https://fr.wikipedia.org/wiki/Rosier}{Rosier à roses pâles} & Rosa centifolia L. & Rosaceae & Bouton floral, pétale. & En l’état \\ \hline
\vocnoindexref{https://fr.wikipedia.org/wiki/Rosier}{Rosier de Damas} & Rosa damascena Mill. & Rosaceae & Bouton floral, pétale. & En l’état \\ \hline
\vocnoindexref{https://fr.wikipedia.org/wiki/Rosier}{Rosier de Provins.Rosier à roses rouges} & Rosa gallica L. & Rosaceae & Bouton floral, pétale. & En l’état \\ \hline
\vocnoindexref{https://fr.wikipedia.org/wiki/Rosier}{Rosier sauvage} & Voir : Eglantier. &  &  &  \\ \hline
\vocnoindexref{https://fr.wikipedia.org/wiki/Safran}{Safran} & Crocus sativus L. & Iridaceae & Stigmate. & En l’état - En poudre \\ \hline
\vocnoindexref{https://fr.wikipedia.org/wiki/Sarriette}{Sarriette des jardins} & Satureja hortensis L. & Lamiaceae & Feuille, sommité fleurie. & En l’état - En poudre \\ \hline
\vocnoindexref{https://fr.wikipedia.org/wiki/Sarriette}{Sarriette des montagnes} & Satureja montana L. & Lamiaceae & Feuille, sommité fleurie. & En l’état - En poudre \\ \hline
\vocnoindexref{https://fr.wikipedia.org/wiki/Sauge}{Sauge d’Espagne} & Salvia lavandulifolia Vahl. & Lamiaceae & Feuille, sommité fleurie. & En l’état - En poudre \\ \hline
\vocnoindexref{https://fr.wikipedia.org/wiki/Sauge}{Sauge officinale} & Salvia officinalis L. & Lamiaceae & Feuille. & En l’état \\ \hline
\vocnoindexref{https://fr.wikipedia.org/wiki/Sauge}{Sauge sclarée.Sclarée toute-bonne} & Salvia sclarea L. & Lamiaceae & Feuille, sommité fleurie. & En l’état - En poudre \\ \hline
\vocnoindexref{https://fr.wikipedia.org/wiki/Sauge}{Sauge trilobée} & Salvia fruticosa Mill.(= S. triloba L. f.). & Lamiaceae & Feuille. & En l’état - En poudre \\ \hline
\vocnoindexref{https://fr.wikipedia.org/wiki/Seigle}{Seigle} & Secale cereale L. & Poaceae & Fruit, son. & En l’état - En poudre \\ \hline
\vocnoindexref{https://fr.wikipedia.org/wiki/Serpolet}{Serpolet.Thym serpolet} & Thymus serpyllum L. sensu latiore. & Lamiaceae & Feuille, sommité fleurie. & En l’état - En poudre \\ \hline
\vocnoindexref{https://fr.wikipedia.org/wiki/Sterculia}{Sterculia.Gomme Karaya.Gomme M’Bep.Gomme de Sterculia} & Sterculia urens Roxb., S. tomentosa Guill. et Perr. & Sterculiaceae & Exsudation gommeuse = gomme de Sterculia, gomme Karaya, gomme M’Bep. & En l’état - En poudre - Extrait sec aqueux \\ \hline
\vocnoindexref{https://fr.wikipedia.org/wiki/Sureau}{Sureau noir} & Sambucus nigra L. & Caprifoliaceae & Fleur, fruit. & En l’état \\ \hline
\end{tabularx}
\newpage
\noindent\begin{tabularx}{\textwidth}{|X|X|X|X|X|}
\hline
\rowcolor{headerbg} \textcolor{white}{\textbf{Nom français}} & \textcolor{white}{\textbf{Nom latin}} & \textcolor{white}{\textbf{Famille}} & \textcolor{white}{\textbf{Parties utilisées}} & \textcolor{white}{\textbf{Forme de préparation}}  \\ \hline
\vocnoindexref{https://fr.wikipedia.org/wiki/Tamarinier}{Tamarinier de l’Inde} & Tamarindus indica L. & Fabaceae & Pulpe de fruit. & En l’état - En poudre \\ \hline
\vocnoindexref{https://fr.wikipedia.org/wiki/Temoe-lawacq}{Temoe-lawacq} & Curcuma xanthorrhiza Roxb. & Zingiberaceae & Rhizome. & En l’état \\ \hline
\vocnoindexref{https://fr.wikipedia.org/wiki/Thé}{Thé du Paraguay} & Voir : Maté. &  &  &  \\ \hline
\vocnoindexref{https://fr.wikipedia.org/wiki/Théier}{Théier.Thé} & Camellia sinensis (L.) Kuntze(= C. thea Link)(= Thea sinensis (L.) Kuntze). & Theaceae & Feuille. & En l’état - Extrait sec aqueux \\ \hline
\vocnoindexref{https://fr.wikipedia.org/wiki/Thym}{Thym} & Thymus vulgaris L.,T. zygis L. & Lamiaceae & Feuille, sommité fleurie. & En l’état - En poudre \\ \hline
\vocnoindexref{https://fr.wikipedia.org/wiki/Thym}{Thym serpolet} & Voir : Serpolet. &  &  &  \\ \hline
\vocnoindexref{https://fr.wikipedia.org/wiki/Tilleul}{Tilleul} & Tilia platyphyllos Scop., T. cordata Mill.(= T. ulmifolia Scop.) (= T. parvifolia Ehrh.ex Hoffm.) (= T. sylvestris Desf.),T. × vulgaris Heyne ou mélanges. & Tiliaceae & Aubier, inflorescence. & En l’état \\ \hline
\vocnoindexref{https://fr.wikipedia.org/wiki/Trèfle}{Trèfle d’eau} & Voir : Ményanthe. &  &  &  \\ \hline
\vocnoindexref{https://fr.wikipedia.org/wiki/Ulmaire}{Ulmaire} & Voir : Reine-des-prés. &  &  &  \\ \hline
\vocnoindexref{https://fr.wikipedia.org/wiki/Verveine}{Verveine odorante} & Aloysia citrodora Palau(= Aloysia triphylla (L’Hérit.) Britt.)(= Lippia citriodora H.B.K.). & Verbenaceae & Feuille. & En l’état \\ \hline
\vocnoindexref{https://fr.wikipedia.org/wiki/Vigne}{Vigne rouge} & Vitis vinifera L. & Vitaceae & Feuille. & En l’état \\ \hline
\vocnoindexref{https://fr.wikipedia.org/wiki/Violette}{Violette} & Viola calcarata L.,V. lutea Huds.,V. odorata L. & Violaceae & Fleur. & En l’état \\ \hline
\vocnoindexref{https://fr.wikipedia.org/wiki/Violette}{Violette tricolore} & Voir : Pensée sauvage. &  &  &  \\ \hline
\end{tabularx}

        
    % Insérer l'index
    \printindex

    \newpage
    \section*{Références}
    \addcontentsline{toc}{section}{Références}
	    
\renewcommand{\refname}{}
\begin{thebibliography}{9}

    \bibitem{laporte2023}
    Florence Laporte,
    \textit{Les plantes des druides},
    Rustica editions, 2023,
    Edelvives Espagne,
    Frédérique Chavances,
    ISBN: 978-2-8153-1020-8.
    
\end{thebibliography}
\end{document}