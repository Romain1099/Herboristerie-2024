\documentclass[a4paper,11pt,fleqn]{article}

\usepackage[left=1cm,right=0.5cm,top=0.5cm,bottom=2cm]{geometry}

\usepackage{bfcours}
\usepackage{cite} % ou \usepackage{natbib} selon votre style de citation

\def\rdifficulty{1}
\setrdexo{%left skip=1cm,
display exotitle,
exo header = tcolorbox,
%display tags,
skin = bouyachakka,
lower ={box=crep},
display score,
display level,
save lower,
score=\points,
level=\rdifficulty,
overlay={\node[inner sep=0pt,
anchor=west,rotate=90, yshift=0.3cm]%,xshift=-3em], yshift=0.45cm
at (frame.south west) {\thetags[0]} ;}
]%obligatoire
}
\setrdcrep{seyes, correction=true, correction color=monrose, correction font = \large\bfseries}

\newcommand{\tikzinclude}[1]{%
    \stepcounter{tikzfigcounter}%
    \csname tikzfig#1\endcsname
}
\input{tex/figures}

\hypersetup{
    pdfauthor={C.Ehrhard - R.Deschamps},
    pdftitle={},
    pdfsubject={},
    pdfkeywords={},
    pdfproducer={LuaLaTeX},
    pdfcreator={Boum Factory}
}

\newcommand{\monimage}[2]{
	\includegraphics[width=#1\textwidth]{Images/#2}
}
\newcommand{\includeplantenature}[4][0.25]{
    \begin{minipage}[t]{#1\textwidth}    
        \phantom{a}
        \begin{center}
            \includegraphics[width=0.8\linewidth]{Images/plantes_nature/#2}
            \captionof{figure}{#3}
            {\footnotesize Plante en milieu naturel\\#4}
        \end{center}
    \end{minipage}
}
\newcommand{\includeplanteseche}[4][0.25]{
    \begin{minipage}[t]{#1\textwidth}    
        \phantom{a}
        \begin{center}
            \includegraphics[width=0.8\linewidth]{Images/plantes_sechees/#2}
            \captionof{figure}{#3}
            {\footnotesize Plante émondée\\#4}
        \end{center}
    \end{minipage}
}
\newcommand{\includeprepa}[4][0.25]{
    \begin{minipage}[t]{#1\textwidth}    
        \phantom{a}
        \begin{center}
            \includegraphics[width=0.8\linewidth]{Images/preparations/#2}
            \captionof{figure}{#3}
            {\footnotesize Préparation\\#4}
        \end{center}
    \end{minipage}
}

\newcommand{\vocref}[2]{
    \href{#1}{\bfseries\color{monrose} #2\index{#2}}
}
\newcommand{\vocnoindexref}[2]{
    \href{#1}{\bfseries\color{monrose} #2}
}

\usepackage{imakeidx}
\makeindex[title=Index des termes, intoc]

\usepackage{datatool}

\definecolor{headerbg}{RGB}{0, 128, 0} % Couleur du fond de l'en-tête
\definecolor{rowbg1}{RGB}{245, 245, 245}
\definecolor{rowbg2}{RGB}{255, 255, 255}

% Charger les données depuis le fichier CSV
%\DTLloaddb{liste148}{Ressources/liste_148.csv}

\definecolor{highlight}{RGB}{255, 255, 0} % Couleur de surlignage
\newcommand{\voc}[1]{\textbf{\color{headerbg!75!black}#1}\index{#1}}

\newcolumntype{L}[1]{>{\raggedright\arraybackslash}p{#1}} % Définir une colonne alignée à gauche avec largeur fixe

% Commande pour créer un tableau pour une plante
\newcommand{\plante}[5]{
\begin{tabular}{|L{3cm}|L{4cm}|L{2cm}|L{3cm}|L{3cm}|}
\hline
\rowcolor{headerbg}
\textcolor{white}{\textbf{Nom français}} & \textcolor{white}{\textbf{Nom latin}} & \textcolor{white}{\textbf{Famille}} & \textcolor{white}{\textbf{Parties utilisées}} & \textcolor{white}{\textbf{Forme de préparation}} \\
\hline
\rowcolor{rowbg1}
#1 & #2 & #3 & #4 & #5 \\
\hline
\end{tabular}
}

\renewcommand\boite[2]{
\begin{tcolorbox}[nobeforeafter,title=\bcfleur #1,halign title=flush left,fonttitle=\bfseries,colbacktitle=headerbg,coltitle=white,colback=white]%red!50!black
#2
\end{tcolorbox}
}



\begin{document}

\setcounter{pagecounter}{0}
\setcounter{ExoMA}{0}
\setcounter{prof}{0}
%Pour les overlay
\def\points{\phantom{AAA}}
\def\difficulty{\phantom{AAA}}
\chapitre[
    $\mathcal{C}\mathcal{R}$% niveau % $\mathbf{6^{\text{ème}}}$
    ]{
    Herboristerie 2024% theme
    }{
    Association% type_etablissement
    }{
    Arsimed% nom_etablissement
    }{
    \tableofcontents \newpage% supplement
    }{
    Notes de stage :% type_document
    }

    \section*{Remerciements}
    \addcontentsline{toc}{section}{Remerciements}
        Groupe d'Herboristerie - Stages Arsimed - session 1

Formateur : Delphine - \frquote{Sacrées Plantes}

Participants : 
\begin{multicols}{3}
    \begin{itemize}[label = \faPen]
        \item \'Emilie et Alfred : merci pour vos notes qui ont pu ou seront incorporées au document.
        \item Pauline a.k.a \frquote{Pissenlit}
        \item Catherine
        \item Miryam : merci pour tes notes qui ont pu être incorporées au document.
        \item Cynthia dont les recherches et les apports à ce document on été nombreuses.
        \item Romain
    \end{itemize}
\end{multicols}



    \newpage
    \section{Transformations}

        \subsection{Teintures mères}

\begin{Defi}[Teinture mère]

    Une \voc{teinture mère} désigne une préparation de plantes infusées dans de l'\voc{alcool}. \\
    L'utilisation professionnel de ce terme est \textit{réservé} aux pharmaciens. Nous utiliserons donc le terme commun d'\voc{alcoolature} dans la suite de ce document.\\

    On parle de \voc{teinture officinale} lorsque la préparation est réalisée à l'aide de \textbf{plantes sèches}.\\

    On parle de \voc{dynamiser} une solution lorsqu'on la \textbf{mélange}.\\

    La \voc{diffusion} des principes actifs s'effectue en \voc{dynamisant} le bocal \textbf{chaque jour}. 
\end{Defi}
\begin{Remarque}[]%\monimage{0.05}{avec_sourire.png}

    \begin{itemize}[label=\faPen]
        \item Plus le degré d'alcool est élevé, plus les principes actifs se diffuseront efficacement. 
        \item Permet d'effectuer une préparation même si on dispose d'une \textbf{faible quantité de plantes}. 
        \item On peut utiliser indiféremment des plantes \voc{fraîches} ou \voc{sèches}.
        \item Cela permet d'extraire des \voc{principes actifs} \voc{complexes} de la plante. Par exemple :
                \begin{multicols}{3}
                    \begin{itemize}
                        \item gomme et résine.
                        \item \vocref{https://fr.wikipedia.org/wiki/Alcalo\%C3\%AFde}{alcaloïdes} 
                        \item les principes \voc{volatiles}
                    \end{itemize}
                \end{multicols}
    \end{itemize}
\end{Remarque}
\begin{multicols}{2}
    \boite{Préparation :}{
        Utiliser un alcool assez \textbf{fort} ( type rhum, absinthe\ldots ). \\
        \textbf{Degré d'alcool souhaité} : entre $40\degres$ et $90\degres$. \\
        \begin{itemize}[label=\mysquare]
            \item Découper les plantes \textbf{séchées} à l'aide d'un \voc{sécateur} ou d'une \voc{paire de ciseaux}.\\Il est également possible de les broyer à l'aide d'un mortier.
            \item Les placer dans un bocal adapté à la taille de la cueillette, \textbf{à ras} et \textbf{sans tasser}.
            \item Couvrir d'\voc{alcool} en veillant à \textbf{éliminer} les \textbf{bulles d'air}.
            \item Refermer le bocal et conserver dans un environnement \textbf{propre} et \voc{à l'abri de la lumière}.
        \end{itemize}
    }

        
    \columnbreak 


    \boite{Macération :}{
        \begin{itemize}[label=\mysquare]
            \item \voc{Dynamiser} chaque jour pendant \textbf{28 jours} pour permettre la \voc{diffusion} des principes actifs. 
        \end{itemize}
    }\\
    \includeprepa[0.2]{couper_plantes.jpg}{Découper les plantes}{30/07/2024}
    \includeprepa[0.2]{alcoolature_coquelicot.jpg}{Remplir le bocal}{30/07/2024}


\end{multicols}
\begin{center}

\boite{Conservation :}{
    \begin{itemize}[label=\faPen]
        \item Se conserve \voc{à l'abri de l'humidité}.
        \item La teinture mère se conserve sur une période allant de 2 à 5 ans.% \monimage{0.1}{groupe/tetra.png}
    \end{itemize}
    
}
\end{center}
\boite{Conseils d'utilisation}{
    \begin{itemize}[label=\faPen]
        \item Une attention toute particulière est à porter à l'utilisation de teinture mère. 
        \item L'automédication n'est pas une pratique à prendre à la légère et il est conseillé de s'entretenir avec un professionnel. 
        \item Les dosages ne sont plus soumis à autant de contraintes qu'auparavant.\\
                Les entreprises pharmaceutiques sont libres de doser la quantité de plante dans leurs teintures mères sans l'indiquer. 
                Néanmoins, voici quelques bonnes pratiques concernant la posologie : \\
    \end{itemize}
    \boite{Posologie :}{
        \begin{itemize}[label=\faPen]
            \item Pour un alcool à $50\degres$, la posologie indiquée est $\mathbf{30}$\textbf{gouttes par jour}. 
            \item De façon générale, commencer avec une dose réduite puis augmenter progressivement. 
        \end{itemize}
    }
}


\newpage
\subsection{Solvant eau}

\label{infusion}
\begin{Defi}[Infusion]
	
	Une \voc{infusion} consiste à faire \voc{macérer} une plante dans de l'\voc{eau chaude}.\\

	C'est l'effet de la \textbf{chaleur} qui permet de diffuser les \voc{principes actifs}.
	
	\boite{Préparation :}{
		Pour préparer une infusion, de \textbf{recouvrir} les plantes d'une eau à $75\degres$.\\
		
		Laisser \textbf{infuser} $15$ minutes.
	}
\end{Defi}



\begin{Remarque}
	\begin{itemize}[label = \faPen]
		\item \textbf{Couvrir} le mélange pour garder les \voc{principes volatiles}.
		\item L'eau ne doit pas être bouillante $\longrightarrow$ casse les molécules.
	\end{itemize}
\end{Remarque}

\label{decoction}
\begin{Defi}[Décoction]
	
	Une \voc{décotion} consiste à faire \voc{macérer} une plante dans de l'\voc{eau chaude}.\\

	A la \textbf{différence} de l'infusion, la \textbf{décoction} demande un \textbf{départ à froid}. \\
	
	Ce procédé d'extraction est intéressant lorsque la plante est \textbf{dure} ( par exemple de l'\textbf{écorce}, les \textbf{baies}... ).

	\boite{Préparation :}{
		\begin{itemize}
			\item[\bcoutil] Pour préparer une décoction, \textbf{recouvrir} les plantes d'une eau à $\approx20\degres$.\\

			\item[\bchorloge] Porter à \voc{ébullition} pendant $5\text{ à }30$ minutes.

			\item[\bcoutil] Filtrer et laisser reposer selon l'utilisation future du mélange.
		\end{itemize}
	}
\end{Defi}

\boite{Intérets de l'utilisation du solvant \frquote{eau} :}{
	Un \textbf{intéret pratique} de l'utilisation de l'eau est la \textbf{rapidité} d'accès au produit fini.\\

	De plus, il faut boire presque $2$ litres d'eau par jour.\\
	Les tisanes sont donc un excellent moyen d'éviter de se déshydrater en plus des effets apportés par les plantes. 
	\begin{multicols}{2}
	\textbf{Principes actifs récupérés : }
	\begin{itemize}[label = \faPen]
		\item \voc{Vitamines}
		\item \voc{Minéraux}
		\item \voc{Sucres} ( saccharides )
		\item \voc{Principes amers} $\longrightarrow$ digestion
	\end{itemize}

	\textbf{Modes d'utilisation : }
	\begin{itemize}[label = \bcoutil]
		\item \voc{Cataplasme}
		\item \voc{Bain de plantes}
		\item \voc{Compresses}
		\item \voc{Inhalation}
	\end{itemize}

	\columnbreak

	\includeprepa[0.4]{presentation_infusion_sacrees_plantes_6.jpg}{Infusion}{31/07/2024}
\end{multicols}
}

\begin{Remarque}[Mémoire de l'eau]
	Les travaux sur l'\voc{homéopathie} et la \voc{mémoire de l'eau}, pourront intéresser le lecteur curieux. 
	Les teintures mères sont les souches en homéopathie.\\
	Elles sont ensuite diluées dans de l'eau.
	\textbf{Exemple} : 2CH correspond à $2$ dillutions successives à $1\%$. 

\end{Remarque}
\newpage
\subsection{Solvant vinaigre}

Texte sur le vinaigre
\newpage
\subsection{Macérat huileux}

\begin{Defi}[Macérat huileux]

    Un \voc{macérat huileux} ou \voc{macérat solaire} désigne une infusion de plantes dans un \voc{corps gras}. \\
    Ici, c'est l'effet de la \textbf{chaleur} qui permet la \voc{diffusion} des principes actifs.
\end{Defi}

\begin{multicols}{2}
    \boite{Préparation}{
        \begin{itemize}[label=\mysquare]
            \item Découper les plantes \textbf{séchées}.
            \item Les placer dans un bocal \textbf{à ras} et \textbf{sans tasser}.
            \item Couvrir d'huile \textbf{végétale} en veillant à \textbf{éliminer} les \textbf{bulles d'air}.
            \item Laisser le bocal ouvert 24h dans un environnement \textbf{propre} et \voc{ensoleillé} ( ou chaud ). 
        \end{itemize}
    }

    \boite{Macération}{
        \begin{itemize}[label=\mysquare]
            \item Refermer le pot en éliminant la \voc{condensation}. 
            \item Conserver \voc{au soleil} ou \voc{au chaud} pendant $\mathbf{28}$\textbf{ jours}.
        \end{itemize}
    }
\end{multicols}

\boite{Conservation}{
    \begin{itemize}[label=\faPen]
        \item Se conserve \voc{à l'abri de l'humidité}.
        \item Le macérat huileux peut se conserver sur une période allant de 6 mois à 2 ans.% \monimage{0.1}{groupe/tetra.png}
    \end{itemize}
    
}
\boite{Conseils d'utilisation}{
    Il est conseillé de \voc{filtrer} le mélange avant d'utiliser l'huile. \\
    Selon l'huile choisie, on pourra consommer le macérat lors des repas. 
}

\begin{Remarque}[]%\monimage{0.05}{avec_sourire.png}
    Les huiles les plus \voc{stables} sont l'\voc{huile d'olive} et l'\voc{huile de tournesol} et permettent une conservation sur \textbf{2 ans}. \\
    Les autres huiles végétales ont une durée de conservation de \textbf{6 mois}.
\end{Remarque}
\newpage
\subsection{Extraction des huiles essentielles}

Voir le tutoriel d'\vocref{https://fr.wikihow.com/extraire-des-huiles-essentielles}{extraction des huiles essentielles}.
\newpage

	\newpage
    \section{Botanique}

        

\subsection{Introduction}
En France, la vente des plantes médicinales (inscrite à la \voc{pharmacopée}), est réservée aux pharmaciens, à l’exception de 148 espèces libérées et d’une centaine d’aromates et épices.

On utilise la \voc{liste des 148} qui répertorie les informations de base sur les plantes librement accessibles. 

Site internet de référence : \\
\begin{center}\href{https://www.passerelleco.info/spip.php?page=article\&id_article=407}{https://www.passerelleco.info/spip.php?page=article\&id\_article=407}\end{center}

\subsection{Plantes adaptogènes}
\subsection{Plantes adaptogènes}

\begin{Defi}[Plante adaptogène]

    Le concept de \vocref{https://fr.wikipedia.org/wiki/Adaptog\%C3\%A8ne}{plantes adaptogènes} nous vient du Dr. \voc{Nicolai Lazarev}, un toxicologue russe, 
    qui cherchait à définir le type d'action de plantes comme le ginseng en 1947.\\

    De façon générale, une plante \voc{adaptogène} permet d'aider à \voc{gérer un stress} employé içi au sens large.\\

    Pour être considérée comme \voc{adaptogène}, une plante doit satisfaire un certain nombre de critères :
    \begin{enumerate}
        \item Être \textbf{non toxique}
        \item Déclencher une réponse \textbf{non-spécifique} du corps
        \item Déclencher une \textbf{action régulatrice} sur les \voc{processus physiologiques}, peu importe le sens du déséquilibre.
    \end{enumerate} 
    On peut les considérer comme des plantes intelligentes : peuvent équilibrer le niveau hormonal et protéger tout le corps.\\

    On considère qu'au bout de sept jours, une plante adaptogène doit faire de l'effet. 

    
\end{Defi}

\begin{Remarque}
    Equivalent aux \frquote{toniques supérieures} en médecine chinoise, qui est globale et holistique.\\
    Ces plantes \textbf{stimulent} le système nerveux / immunitaire et endocrinien.\\
    Ces plantes ont généralement un effet antioxydant, hépatoprotecteur, cardioprotecteur.\\
    En général, elle soutient les fonctions surrénales, ce qui contre les effets néfastes du stress.\\
    Elle active les cellules du corps pour accéder à plus d'énergie, elle débarrasse les cellules de leurs déchets métaboliques toxiques, 
    elle fournit un effet \vocref{https://fr.wikipedia.org/wiki/Anabolisme}{anabolique} 
    et aide le corps à utiliser mieux l'oxygène et accélère la régulation des biorythmes.\\

\end{Remarque}

\begin{Exemple}[Plantes adaptogènes]

    Les plantes adaptogènes les plus connues et utilisées sont :
    \begin{itemize}
        \item La \vocref{https://fr.wikipedia.org/wiki/Rhodiola_rosea}{Rhodiola} ou \voc{orpin rose}
        \item L'\vocref{https://fr.wikipedia.org/wiki/\%C3\%89leuth\%C3\%A9rocoque}{éleuthérocoque}
        \item Le \vocref{https://fr.wikipedia.org/wiki/Ginseng}{Ginseng rouge}
        \item La \vocref{https://fr.wikipedia.org/wiki/Schisandra}{Schisandra de Chine}
    \end{itemize}
    Les champignons adaptogènes les plus connus et utilisés sont : 
    \begin{itemize}[label = \bcfleur]
        \item Le \vocref{https://fr.wikipedia.org/wiki/Inonotus_obliquus}{Chaga} – Lonotus obliquus
        \item Le \vocref{https://fr.wikipedia.org/wiki/Cordyceps}{Cordyceps} – Cordyceps sinensis, vient du Tibet
        \item L'\vocref{https://fr.wikipedia.org/wiki/Ashwagandha}{Ashwagandha} – Vithania Somnifera
        \item Le \vocref{https://fr.wikipedia.org/wiki/Ganoderme_luisant}{Reishi}
    \end{itemize}
\end{Exemple}

\newpage
\subsection{Asteraceae}
\subsubsection{Achillée millefeuille}

\boite{Extrait de la liste des 148 :}{
    \plante{Achillée millefeuille.}{Achillea millefolium L.}{Asteraceae}{Sommité fleurie.}{En l’état}
}
\newpage
\subsection{Bracicaceae}
\input{tex/botanique/bracicaceae.tex}
\newpage
\subsection{Cannabaceae}

\label{houblon}
\ficheidentiteplante
{Houblon}
{%effet général
    L'\vocref{https://fr.wikipedia.org/wiki/Houblon}{houblon} ou \voc{humulus lupulus} .
}
{%utilisation privilégiée
    \begin{itemize}[label = \bcplume]
        \item
    \end{itemize}
}
{%infos cueillette
    \begin{itemize}[label = \bcplume]
        \item On récupère les \voc{cônes}.
        \item 
    \end{itemize}
}
{%sous quelle forme utiliser
    \begin{itemize}[label = \bccrayon]
        
    \end{itemize}
}
{%supplément
    
}
{%image
    downloads/houblon_bouba.jpg
}
{%titre photo
    Houblon
}
{%description photo : "lieu - date"
    source : wikipédia
}


\newpage
\newpage
\subsection{Caprifoliaceae}

\subsubsection{Sureau}
\Potins{Sureau}{
    La mère sureau ou grand-mère sureau ou fée du sureau serait une présence féminine, gardienne du sureau et de la mère Terre. 
    Cette mère sureau serait peut-être la femme du dieu Pan, dieu de la nature, de la forêt et des animaux.\\

    \textbf{Dans la tradition celtique :}
    Le sureau est le treizième arbre du calendrier des arbres celtiques, placé en fin d’année. \\
    Il est l’un des derniers à perdre ses feuilles à l’automne, et le premier à les sortir au printemps. \\
    Il représente le lien entre la mort et la vie. On a retrouvé beaucoup de graines lors de fouilles archéologiques. \\
    Le sureau était déjà utilisé à une époque très lointaine pour le culte des morts.\\

    Le sureau a la capacité de se régénérer très facilement. Il suffit de planter une baguette de sureau dans la terre et il y a 
    de fortes chances qu’elle se réimplante et redevienne un arbuste l’année suivante. \\
    Pour les Celtes, il représente la vitalité, la vie éternelle, la mort et la renaissance. \\
    Il facilite le passage entre les mondes.\\

    Les Celtes s’en servaient pour confectionner des flûtes magiques pour faciliter les conversations avec les morts.\\
    Le sureau était considéré comme un arbre sacré dont le bois creux abritait des divinités, des esprits de la forêt, des fées. \\
    Si jamais l’on osait couper cet arbre malgré l’interdiction, cela portait malheur.\\ 
    Il est le lien avec les esprits de la nature.\\

    Le bois creux du sureau permettait de fabriquer des baguettes magiques, dans lesquelles on pouvait glisser des objets aux 
    pouvoirs guérisseurs.\\
    C’est une plante de protection que l’on porte sur soi ou que l’on suspend au-dessus des portes et des fenêtres. 
    On dit que le sureau n’est jamais frappé par la foudre.\\

    Mettre des feuilles fraîches de sureau dans les terriers permet d’en éloigner les rongeurs. 
    Pour éloigner les mouches, on suspendra dans la maison des feuilles fraîches de sureau.\\

    Des fleurs de sureau séchées, étalées dans le fond d’un cageot de pommes, permettront de conserver ces fruits plus longtemps 
    et leur donneront en plus un léger goût d’ananas.\\

    Le sureau influence le sommeil et les rêves.
    Il est possible d’utiliser les baies de sureau pour se teindre les cheveux. \\
    Cela permet en plus de les nourrir et de leur redonner de la force.
}

\newpage
\subsection{Herbaceae}


\ficheidentiteplante
{Plantain lancéolé}
{%effet général
    L'\vocref{https://fr.wikipedia.org/wiki/Plantago_lanceolata}{Plantain lancéolé}, en latin \textit{lantago lanceolata}.
}
{%utilisation privilégiée
    \begin{itemize}[label = \bcplume]
        \item En premier lieu contre les maladies des organes respiratoires.
        \item Effet astringent
        \item Effet cicatrisant
        \item S'utilise contre les inflammation et les \voc{hémorroïdes}
    \end{itemize}
}
{%infos cueillette
    \begin{itemize}[label = \bcplume]
        \item La plante entière est à utiliser ( y compris les racines ).
        \item La floraison a lieux d'\textbf{avril} à \textbf{octobre} sur le littoral méditerranéen.
    \end{itemize}
}
{%sous quelle forme utiliser
    \begin{itemize}[label = \bccrayon]
        \item En \lien{infusion}{infusion} : $1$ c.a.c de feuilles avec $0{,}25$cL d'eau.
        \item En \voc{cataplasme} de feuilles broyées.
        \item En \voc{sirop}.
        \item Contre les piqûres (moustiques, guêpes, orties…) et les démangeaisons.\\
                Frotter une ou plusieurs feuilles sur l’endroit de la piqûre jusqu’à en extraire le suc.
        \item Les feuilles fraîches, riches en mucilages, peuvent être utilisées en cataplasme pour arrêter les \voc{saignements} ou soigner les \voc{ampoules}.
    \end{itemize}
}
{%supplément
        \boite{Usage interne :}{
            \begin{itemize}[label = \bctrefle]
                \item Toute la plante est \voc{comestible}.
                \item Les feuilles tendres ont un goût de \textbf{champignon} se mangent \textbf{crues} en salade.\\
                \item Les feuilles plus âgées peuvent être mangées en soupe.
            \end{itemize}
        }

}
{%image
    plantain_lanceole.jpeg
}
{%titre photo
    Plantain lancéolé
}
{%description photo : "lieu - date"
    Jardin des thermes de \frquote{Cassinomagus} - 04/08/2024
}

\newpage
\subsection{Hypericaceae}

\ficheidentiteplante
{Millepertuis}
{%effet général
    Le \voc{millepertuis} en latin \textit{hypericum perforatum} est aussi appelé le \voc{soleil intérieur}.\\

    C'est une plante qui a un très fort pouvoir de \voc{guérison} et de \voc{protection}.\\
    Elle est \textbf{riche en mélatonine} et a une action \textbf{régulatrice du sommeil}.
}
{%utilisation privilégiée
    \begin{itemize}[label=\bcoutil]
        \item Contre les dépressions
        \item Contre les \textbf{maux d'hiver} : \\
        \item Dépressions 
        \item Stress
        \item Système nerveux fragilisé
    \end{itemize}
}
{%infos cueillette
    \begin{itemize}[label=\faPen]
        \item Penser à se protéger du soleil.
        \item Vérifier qu'il s'agit d'\textit{hypericum perforatum} en \\
                \begin{itemize}[label = \bcoeil]
                    \item Observant des \textbf{petits trous} sous ses feuilles.
                    \item \textbf{\'Ecrasant la fleur} pour voir s'échapper un \textbf{liquide rouge}. 
                \end{itemize}
    \end{itemize}
}
{%sous quelle forme utiliser
    \begin{itemize}[label = \bcplume]
        \item On utilise principalement les \voc{sommités fleuries}.
        \item On peut l'utiliser en fumigation $\longrightarrow$ vertus protectrices et purificatrices
        \item Peut être utilisé comme plante \voc{teinctoriale} pour colorer en \textbf{rouge}.
    \end{itemize}
}
{%remarques
    
    \begin{minipage}[t]{0.35\textwidth}
        \boite{Utilisation interne :}{
            On peut faire une \lien{infusion}{infusion} ou une \lien{decoction}{decoction} en utilisant $15$ à $20$ grammes de plantes séchées \textbf{par litre d'eau}.\\
            $2$ à $4$ tasses par jour.\\

            \begin{itemize}[label=\faPen]
                \item Affections pulmonaire chronique
                \item Troubles hépatiques et circulatoires
                \item Asthme
                \item Cystite
                \item Herpès
            \end{itemize}
        }
    \end{minipage}
    \hfill
    \begin{minipage}[t]{0.65\textwidth}
        \boite{Utilisation externe :}{
            \setlength{\columnseprule}{1pt}
            \begin{multicols}{2}
                \textbf{En macérat huileux :}
                \begin{itemize}[label=\faPen]
                    \item Douleurs rhumatismales
                    \item Sciatiques
                    \item Tendinites, torticoli, luxation
                    \item Ulcères
                    \item Goutte
                    \item Plaies
                    \item Brûlures
                    \item Hématomes
                    \item Coupures
                    \item Contusions
                \end{itemize}
                
                \columnbreak


                \textbf{En pommade : }
                \begin{itemize}[label=\faPen]
                    \item Douleurs nerveuses
                    \item Engorgement mammaire
                \end{itemize}
                \vspace{0.5cm}
                \textbf{En infusion :} comme \voc{lotion} pour les peaux grasses et \voc{acnéeiques}.
            \end{multicols}
        }
    \end{minipage}
}
{%image
    millepertuis (1).jpeg
}
{%titre photo
    Millepertuis
}
{%description photo : "lieu - date"
    Jardin de  - 30/07/2024 
}

\begin{Remarque}
    \begin{itemize}[label=\faPen]
        \item Le millepertuis induit un effet \voc{photosensible} pouvant occasionner des \textbf{brulures}.
    \end{itemize}
\end{Remarque}

\Potins{Millepertuis}{
    L’herbe de la Saint-Jean: la plante du solstice d’été    
    Le millepertuis avait une grande renommée dans tout les pays celtes. \\
    Elle apportait paix et prospérité au foyer, santé aux animaux et récoltes abondantes. \\
    C’était une plante de guérison, que les gens portaient sur eux à la veille du solstice. \\
    
    
    C’est une plante “solaire”, considérée comme “positive” et récoltée durant les fêtes païennes du solstice avec 
    l’armoise ou l’achillée. \\
    Elle était récoltée de nuit, certains disent même à minuit, afin de conserver la rosée sur les feuilles et les 
    fleurs. Cette eau était considérée comme sacrée et constituait une “eau de longue vie” augmentant les pouvoirs de 
    la plante. \\
    Alexander Carmichael dans son Carmina Gadelica nous donne une incantation à réciter avant la récolte:
    \begin{center}
        \textit{
            Herbe de la Saint-Jean, Herbe de la Saint-Jean\\
            Je te cueillerai avec ma main droite\\
            Je te préserverai avec ma main gauche\\
            Celui qui te trouve dans l’enclos du bétail\\
            Ne sera jamais sans bétail.
        }
    \end{center}
    Les plantes récolées autour du solstice, puis à la Saint-Jean (la fête a été déplacée au 24 juin par les chrétiens) 
    avaient la réputation de chasser les démons et de faire perdre leur dangerosité aux plantes toxiques. 
}
\newpage
\subsection{Lamiaceae}


\input{tex/botanique/lamiaceae/sauge.tex}

\ficheidentiteplantelong
{Romarin}
{%effet général
    Le \voc{romarin}, en latin \textit{Salvia rosmarinus} est également appelée \frquote{herbe aux couronnes} ou \frquote{encensier} en raison de son \textit{odeur camphrée}.\\

    C'est une plante \voc{condimentaire} \voc{aromatique} et \vocref{https://fr.wikipedia.org/wiki/Flore_mellifère}{mellifère} qui est originaire du bassin méditerranéen.\\
    

}
{%Effets recherchés
Fraîche ou séchée, cette herbe condimentaire fait partie de la cuisine méditerranéenne, et une variété se cultive dans les jardins.  \\
C'est également un produit fréquemment utilisé en parfumerie. \\
Enfin, diverses vertus phytothérapeutiques ont été étudiées.
}
{%infos cueillette
    La floraison commence dès le mois de \textbf{février}, parfois en janvier, et se poursuit jusqu'en \textbf{avril-mai}.
    Il est reconnaissable en toute saison à ses feuilles persistantes \vocref{https://fr.wikipedia.org/wiki/Pétiole}{sans pétiole}, coriaces, beaucoup plus longues que larges, 
    aux \textbf{bords légèrement enroulés}, \textbf{vert sombre} luisant sur le dessus, \textbf{blanchâtres}w en dessous.\\

    Il se multiplie facilement au printemps ou à l'automne par \vocref{https://fr.wikipedia.org/wiki/Bouturage}{bouturage} ou \vocref{https://fr.wikipedia.org/wiki/Marcottage}{marcottage} ; 
    plus difficilement en été par semis car sa germination est lente.
}
{%utilisation privilégiée
\textbf{En teinture mère :}
    \begin{itemize}[label = \faPen]
        \item Effets sur le système nerveux
        \item Effet anti-dépressif
    \end{itemize}
    \textbf{En macérat huileux :}
    \begin{itemize}[label = \faPen]
        \item Activer la digestion $\longrightarrow$ effet \voc{hépatoprotecteur}
        \item Effets sur la circulation sanguine
        \item Effet anti-coagulant
    \end{itemize}
    \textbf{En extrait aqueux :}
    \begin{itemize}[label = \faPen]
        \item Effet anti-spasmodique
    \end{itemize}
    \textbf{En huile essentielle :}
    \begin{itemize}[label = \faPen]
        \item Effet anti-bactérien
    \end{itemize}
}
{%remarques
    \begin{Remarque}
        L'huile essentielle de romarin peut avoir des effets neurotoxiques, déclencher \voc{convulsions} et \voc{crises d’épilepsie}.\\
        Par voie orale, et à part l'utilisation en cuisine, il est déconseillé aux femmes enceintes ou allaitantes.
    \end{Remarque}
}
{%image
    romarin.jpg
}
{%titre photo
    Romarin
}
{%description photo : "lieu - date"
    Jardin des thermes de \frquote{Cassinomagus} - 04/08/2024 
}





\newpage
\subsection{Rosaceae}
\ficheidentiteplante
{Aubépine}
{%effet général
    L'\voc{aubépine}, en latin \textit{Crataegus}, est aussi appelée \voc{épine blanche} en raison de ses fleurs blanches.\\
    La signature de ses branches et de ses épines symbolise le \textbf{coeur qui saigne} et tout ce qui est lié à notre \voc{affectif}.

}
{%utilisation privilégiée
    C'est un arbre qui possède des propriétés \voc{apaisantes} et \textbf{régulatrices} de la \voc{tension}.\\

    Elle est recommandée pour lutter contre les \voc{angoisses} et les \voc{insomnies}.
}
{%infos cueillette
    On la trouve dans les haies, les bordures de forêts et de bois.\\
    C'est un arbre \textbf{épineux} mais peu volumineux. \\
    Ses fleurs sont \textbf{blanches} avec un \vocref{https://fr.wikipedia.org/wiki/Pistil}{pistil} rose.\\

    On récolte l'\voc{écorce}, les \voc{baies}, les \voc{feuilles} mais surtout les \voc{fleurs} en \textbf{mai - juin} au début de la floraison.
}
{%sous quelle forme utiliser
    
    \begin{itemize}[label = \bcplume]
        \item L'écorce
        \item Les fruits ( \voc{cénelles} )
        \item Les fleurs
    \end{itemize}
}
{%remarques
    \begin{multicols}{2}

        \boite{Usage interne :}{
            Une cure d'\textbf{infusion} d'aubépine : 
            \begin{itemize}[label = \bctrefle]
                \item On utilise principalement son fruit la \vocref{https://fr.wikipedia.org/wiki/Cenelle}{cénelle} pour ses \voc{antioxydants}.
                \item Angoisses, insomnies
                \item angine
                \item troubles nerveux
                \item mauvaise circulation du sang
            \end{itemize}

            Suivre la cure pendant 3 mois en prenant $1$ cuillère à café par tasse d'eau chaude, $3$ à $4$ fois par jour. \\

            Une \textbf{décoction} à l'aubépine a un effet \voc{anti-diarrhéique} et permet d'expulser les calculs des reins.
        }

        \columnbreak

        \boite{Usage externe :}{
            \begin{itemize}[label = \bccrayon]
                \item Permet de réhydrater les peaux sèches :\\
                        En \textbf{lotion} ou en \lien{decoction}{décoction}, éventuellement versée dans un bain.
            \end{itemize}
        }

    \end{multicols}    

}
{%image
    downloads/aubepine.jpg
}
{%titre photo
    Aubépine monogyne
}
{%description photo : "lieu - date"
    source : wikipedia
}
\begin{Remarque}
    La \textbf{fleur} de l'aubépine n'est pas autorisée à être utilisée en tisane.\\

    La fleur de l'aubépine est un \voc{cardiotonique}, mais qui soigne aussi le cœur émotionnel, elle dé-serre le cœur au niveau \voc{subtil}.
\end{Remarque}

\Potins{Aubépine}{
    Dans la tradition celtique
    Pour les Celtes, c’est un arbre sacré qui a le pouvoir d’éloigner la foudre. La principale de ses vertus est la protection. \\

    La foudre ne tombe que rarement sur un buisson d’aubépine et les oiseaux sont nombreux à y faire leur nid. 
    Les buissons d’aubépine leur assurent en effet, grâce à leurs épines, une protection efficace contre les prédateurs. 
    De par ces observations, les Celtes en ont fait un arbre protecteur dont on suspendait les rameaux aux berceaux des nouveau-nés, 
    ainsi qu’aux entourages des portes et des fenêtres, afin de se protéger des mauvais sorts et des maladies.\\
    On attendait de rentrer dans la partie lumineuse de l’année pour se marier, à partir du 1er mai. 
    L’aubépine, en fleurs au mois de mai, était symbole de chasteté et de pureté, mais également de bonheur, prospérité et 
    fidélité conjugale. C’était l’arbre des mariages. \\
    Dans de nombreuses régions, on tressait autrefois des couronnes d’aubépine en offrande aux fées et aux anges qui venaient 
    danser la nuit autour des buissons en fleurs.\\ 

    L’aubépine était l’arbre sacré de la fête de Beltaine (1er mai),
    elle avait le pouvoir de nous faire entrer dans le monde magique des esprits de la nature.\\

    Autrefois, le bois servait pour la fabrication de manches d’outils et de cannes.
    On plantait de l’aubépine aux abords des maisons car on pensait que sa proximité permettait de conserver la viande, 
    empêchait de faire tourner le lait et faisait fuir les serpents.\\
    Les fruits de l’aubépine sont comestibles en confitures ou compotes. Il est préférable d’attendre les premières gelées pour 
    les cueillir.\\

    C’est un régulateur du cœur et de la tension.
    On l’a surnommée \frquote{bonnet de nuit} en raison de ses propriétés à provoquer le
    sommeil.
}
\newpage
\label{reinepres}
\ficheidentiteplante
{Reine des prés}
{%effet général
    La \vocref{https://fr.wikipedia.org/wiki/Filipendula_ulmaria}{reine des prés}, en latin \textit{Filipendula ulmaria} est également nommée
    \textbf{spirée ulmaire} car ses fruits sont tordus en forme de spirale.\\


    Elle est utilisée dans la pharmacopée en \textit{infusion} contre les \textbf{reflux} ou les \textbf{aigreurs d'estomac}.\\

    Ses \textbf{feuilles} sont utilisées pour teindre la laine en \textbf{jaune}.\\

    Elle était utilisée comme alternative au tabac.
}
{%utilisation privilégiée
    On cherche à extraire l'\voc{acide salicylique} ( aspirine ) de cette plante. \\

    La reine des prés a un effet \voc{anti-inflammatoire} et \voc{diurétique}.\\

    Son utilisation est susceptible de faire \voc{baisser la fièvre}.
}
{%infos cueillette

    Toutes les parties de la plante peuvent être utilisées.\\
    On récolte les \voc{sommités fleuries} de \textbf{juin à aout} avant leur complet épanouissement. \\

    La reine-des-prés pousse dans les lieux humides, en bordure de ruisseaux ou de rivières.\\
    Comme l'indique son nom, c'est une très grande plante : on la voit de loin !
}
{%sous quelle forme utiliser
    \begin{itemize}[label = \bcplume]
        \item Principalement en infusion :\\
                Réaliser une \textbf{cure} de 3 à 5 tasses par jour. \\
                Dosage : $20$ à $30$g de plantes par litre d'eau.
    \end{itemize}
}
{%remarques
    \begin{multicols}{2}

        \boite{Usage interne :}{
            Une \textbf{infusion} de reine des prés est un remède pour les affections suivantes : 
            \begin{itemize}[label = \bctrefle]
                \item Goutte
                \item Rétention de liquides
                \item Arthrites et douleurs articulaires
                \item Cellulites
                \item Certaines affections des voies urinaires
                \item Obésité graisseuse
            \end{itemize}
        }

        \columnbreak

        \boite{Usage externe :}{
            \begin{itemize}[label = \bccrayon]
                \item Application directe de la feuille sur une plaie pour en accélérer la \voc{cicatrisation}.
            \end{itemize}
        }

    \end{multicols}
}
{%image
    downloads/reine_des_pres.jpg
}
{%titre photo
    Reine des prés en fleurs
}
{%description photo : "lieu - date"
    source : wikipedia
}

\begin{Remarque}
    La reine des prés est \textbf{contre-indiquée} pour les personnes qui ne \textbf{supportent pas l'aspirine}.\\

    \textbf{Associations :}\\
    Pour accentuer les effets de la reine-des-prés, on pourra l'associer à la \voc{menthe} au \voc{cassis}, au \voc{frêne} et au \voc{romarin}.
\end{Remarque}

\newpage
\subsection{Solanaceae}
\input{tex/botanique/solanaceae.tex}
\newpage
\subsection{Urticaceae}
\label{ortie}
\ficheidentiteplante
{Ortie}
{%effet général
    L'\voc{ortie} 
}
{%utilisation privilégiée
    ?
}
{%infos cueillette
    ?
}
{%sous quelle forme utiliser
    \begin{itemize}[label = \bcplume]
        \item 
    \end{itemize}
}
{%remarques
    ?
}
{%image
    sauge_de_jerusalem.jpg
}
{%titre photo
    Sauge de Jérusalem
}
{%description photo : "lieu - date"
    Jardin des thermes de \frquote{Cassinomagus} - 04/08/2024 
}


	\newpage
    \section{Recettes}

        \subsection{Baumes}

    \begin{Defi}[Baume]
        Pour obtenir un \voc{baume}, on utilise le mélange d'un \voc{corps gras} avec de la \voc{cire d'abeille} utilisée comme agent texturant.\\

        Généralement, la masse de cire d'abeille utilisée correspond à $\dfrac{3}{10}$ de la masse de corps gras. 
    \end{Defi}

    \newcommand{\macerat}{\voc{macérat huileux}}
\newcommand{\macerats}{\voc{macérats huileux}}

\ficherecette
{%titre recette
    Baume à lèvres - anti-inflammatoire
}
{%liste d'ingrédients commençant par \item directement.
    \item $50$g de \voc{macérat huileux} de \voc{calendula}.
    \item $10$g de \voc{macérat huileux} d'\voc{achillée}\\\textbf{millefeuille}.
    \item $18$g de \voc{cire d'abeille}
    \item $6$g de \voc{miel}.
    \item $3$ pulvérisations d'\voc{eau florale de rose}.
    \item $1$ goutte de \voc{propolis}.
}
{%liste du matériel commençant par \item directement.
    \item $1$ \frquote{\voc{cul de poule}} propre. 
    \item $1$ fouet. 
    \item $1$ pot \textbf{désinfecté}, \voc{étanche} et \voc{sec}.
    \item $1$ spatule. 
    \item $1$ casserole et de l'eau pour le bain marie. 
    \item Des plaques chauffantes. 
}
{%Métnodes et conseils de conservation
    \begin{itemize}[label=\faPen]
        \item Conservation courte $\approx 6$ mois.
        \item Dans un endroit sec, de préférence à l'abri de la lumière. 
    \end{itemize}
}
{%Métnodes et conseils d'utilisation    
    \begin{itemize}[label=\faPen]
        \item En application locale sur la zone irritée.
        \item La préparation est commestible.
        \item Utiliser un \textbf{ustensile propre} lors de l'utilisation pour prolonger la durée de conservation. 
    \end{itemize}
}
{%Remarques
    Le dosage de cire d'abeille correspond à $\dfrac{3}{10}$ de la masse de \macerat.\\
    Le dosage de miel correspond à $\dfrac{1}{10}$ de la masse de \macerat.\\
    La \textbf{texture} doit correspondre à celle du \frquote{baume du tigre}.
}
{%chemin de l'illustration dans le dossier 'préparations'
    baume/baume.jpg
}
{%Titre donné à l'illustration dans le document latex
    Baume au Calendula
}
{%Légende donnée à l'illustration dans le document latex
    31/07/2024
}

    \newcommandx{\teinture}{\voc{teinture mère}}

\ficherecette
{%titre recette
    Baume pour la lignée féminine
}
{%liste d'ingrédients commençant par \item directement.
    \item $25$g de \voc{millepertuis}.
    \item $35$g d'\voc{achillée millefeuille}.
    \item $11$ gouttes de \teinture d'aubépine.
    \item $6$ pulvérisations d'\voc{eau de rose}.
    \item $8$g de \voc{cire d'abeille}.
}
{%liste du matériel commençant par \item directement.
    \item $1$ \frquote{\voc{cul de poule}} propre. 
    \item $1$ fouet et $1$ spatule. 
    \item $1$ pot \textbf{désinfecté}, \voc{étanche} et \voc{sec}.
    \item $1$ casserole et de l'eau pour le bain marie. 
    \item Des plaques chauffantes.  
}
{%Métnodes de préparation
    \item Verser les macérats huileux dans un cul de poule. 
    \item Ajouter la cire d'abeille \textbf{émiétée} pour faciliter son incorporation durant la \textbf{chauffe}.
    \item Faire chauffer au \textbf{bain-marie} en \textbf{remuant}.
    \item Une fois la cire d'abeille incorporée, sortir du bain-marie et ajouter les ingrédients restants. 
    \item Continuer de mélanger jusqu'à l'apparition d'une \textbf{mayonnaise} sur les rebords de la préparation. \\
            Cela caractérise un refroidissement suffisant.
    \item Verser dans un bocal hermétique, sec, et propre. 
    \item Laisser sécher pendant \textbf{24h} puis refermer le bocal. 
}
{%Métnodes et conseils d'utilisation    
    \textbf{Conseils de conservation :}
    \begin{itemize}[label=\faPen]
        \item Conservation courte $\approx 6$ mois.
        \item Dans un endroit sec, de préférence à l'abri de la lumière. 
    \end{itemize}
    \textbf{Conseils d'utilisation :}
    \begin{itemize}[label=\faPen]
        \item En application sur le \textbf{point de chakra} du \voc{plexus solaire}.
        \item Application possible sur le \textbf{premier point de chakra}.
        \item Utiliser un \textbf{ustensile propre} lors de l'utilisation pour prolonger la durée de conservation. 
    \end{itemize}
}
{%Remarques
    Le dosage de cire d'abeille correspond à $\dfrac{2}{10}$ de la masse de \macerat.\\
    Le dosage de miel correspond à $\dfrac{1}{10}$ de la masse de \macerat.\\
    La \textbf{texture} doit être un peu plus fluide que celle du \frquote{baume du tigre}.
}
{%chemin de l'illustration dans le dossier 'préparations'
    baume/baume_dessus_2.jpg    
}
{%Titre donné à l'illustration dans le document latex
    Baume pour la lignée féminine
}
{%Légende donnée à l'illustration dans le document latex
    01/08/2024
}

\subsection{Cuisine}


\subsection{Onguents}

    \begin{Defi}[Onguent]
        Pour obtenir un \voc{onguent}, on utilise le mélange d'un \voc{corps gras} avec de la \voc{cire d'abeille} utilisée comme agent texturant dans une proportion moindre part rapport au baume.\\

        Généralement, la masse de cire d'abeille utilisée correspond à $\dfrac{1}{10}$ de la masse de corps gras. 
    \end{Defi}

\subsection{Huiles}

\subsection{Tisanes}

\subsection{Vinaigres}

    \ficherecette
{%titre recette
    Vinaigre d'origan
}
{%liste d'ingrédients commençant par \item directement.
    \item 40g d'origan
    \item 1L de \textbf{vinaigre de cidre}
}
{%liste du matériel commençant par \item directement.
    \item Un bocal adapté à la taille de la cueillette.
    \item Une spatule pour tasser les plantes.

}
{%Métnodes et conseils de conservation
    Conserver pendant $28$ jour à l'abri du soleil. \\
    \voc{Dynamiser} chaque jour pour permettre la \voc{diffusion} des minéraux dans le vinaigre.
}
{%Métnodes et conseils d'utilisation    
    \textbf{En cure :}\\
    $1$ cuillère à soupe diluée dans de l'eau chaude.\\
    La préparation est à consommer \textbf{chaque matin} durant \textbf{trois semaines}.\\

    \textbf{Après une cure :}\\
    Arrêter de consommer pendant \textbf{une semaine}. \\
    Reprendre ensuite en \textbf{changeant de plante}.
}
{%Remarques
    Voir les détails de préparation dans la partie \lien{vinaigre}{préparation des vinaigres}.\\

    Il est utile de bien \voc{sécher les plantes}.
}
{%chemin de l'illustration dans le dossier 'préparations'
    vinaigre_plantes_internet.jpg
}
{%Titre donné à l'illustration dans le document latex
    Vinaigre de plantes
}
{%Légende donnée à l'illustration dans le document latex
    source : internet
}

	\newpage
    \section{Utilisations médicinales}

        
\subsection{Saignements}

Dans le cas d'un saignement, réaliser une pommade à base d'\voc{achillée millefeuille}. \\

\subsection{Migraines}

\subsection{Douleurs de règles}

\subsection{Brulûres}

\subsection{Irritations}

\subsection{Coupures}

\subsection{Ballonnements}

\input{tex/medecine/coupures.tex}


    \newpage
    
    \section{Liste des 148}
        
\begin{Defi}[Liste des 148]
    La \voc{liste des 148} désigne la liste des plantes inscrites à la \vocref{https://fr.wikipedia.org/wiki/Pharmacop\%C3\%A9e}{pharmacopée}.
\end{Defi}

Dans le tableau ci-dessous, on peut trouver les informations sur les plantes inscrites dans cette liste : \\
\noindent\begin{tabularx}{\textwidth}{|X|X|X|X|X|}
\hline
\rowcolor{headerbg} \textcolor{white}{\textbf{Nom français}} & \textcolor{white}{\textbf{Nom latin}} & \textcolor{white}{\textbf{Famille}} & \textcolor{white}{\textbf{Parties utilisées}} & \textcolor{white}{\textbf{Forme de préparation}}  \\ \hline
\vocnoindexref{https://fr.wikipedia.org/wiki/Acacia}{Acacia à gomme.} & Acacia senegal (L.) Willd. et autres espèces d’acacias d’origine africaine. & Fabaceae & Exsudation gommeuse = gomme arabique. & En l’état - En poudre - Extrait sec aqueux \\ \hline
\vocnoindexref{https://fr.wikipedia.org/wiki/Ache}{Ache des marais.} & Apium graveolens L. & Apiaceae & Souche radicante. & En l’état - En poudre \\ \hline
\vocnoindexref{https://fr.wikipedia.org/wiki/Achillée}{Achillée millefeuille.Millefeuille.} & Achillea millefolium L. & Asteraceae & Sommité fleurie. & En l’état \\ \hline
\vocnoindexref{https://fr.wikipedia.org/wiki/Agar-agar.}{Agar-agar.} & Gelidium sp., Euchema sp., Gracilaria sp. & Rhodophyceae & Mucilage = gélose. & En l’état - En poudre \\ \hline
\vocnoindexref{https://fr.wikipedia.org/wiki/Ail.}{Ail.} & Allium sativum L. & Liliaceae & Bulbe. & En l’état - En poudre \\ \hline
\vocnoindexref{https://fr.wikipedia.org/wiki/Airelle}{Airelle myrtille.} & Voir : Myrtille. &  &  &  \\ \hline
\vocnoindexref{https://fr.wikipedia.org/wiki/Ajowan.}{Ajowan.} & Carum copticum Benth. et Hook. f.(= Psychotis ajowan DC.). & Apiaceae & Fruit. & En l’état - En poudre \\ \hline
\vocnoindexref{https://fr.wikipedia.org/wiki/Alchémille}{Alchémille.} & Alchemilla vulgaris L. (sensu latiore). & Rosaceae & Partie aérienne. & En l’état \\ \hline
\vocnoindexref{https://fr.wikipedia.org/wiki/Alkékenge}{Alkékenge.Coqueret.} & Physalis alkekengi L. & Solanaceae & Fruit. & En l’état \\ \hline
\vocnoindexref{https://fr.wikipedia.org/wiki/Alliaire.}{Alliaire.} & Sisymbrium alliaria Scop. & Brassicaceae & Plante entière. & En l’état - En poudre \\ \hline
\vocnoindexref{https://fr.wikipedia.org/wiki/Aloès}{Aloès des Barbades.} & Aloe barbadensis Mill.(= Aloe vera L.). & Liliaceae & Mucilage. & En l’état - En poudre \\ \hline
\vocnoindexref{https://fr.wikipedia.org/wiki/Amandier}{Amandier doux.} & Prunus dulcis (Mill.) D. Webb var. dulcis. & Rosaceae & Graine, graine mondée. & En l’état - En poudre \\ \hline
\vocnoindexref{https://fr.wikipedia.org/wiki/Ambrette.}{Ambrette.} & Hibiscus abelmoschus L. & Malvaceae & Graine. & En l’état - En poudre \\ \hline
\vocnoindexref{https://fr.wikipedia.org/wiki/Aneth.}{Aneth.} & Anethum graveolens L.(= Peucedanum graveolens Benth. et Hook.). & Apiaceae & Fruit. & En l’état - En poudre \\ \hline
\end{tabularx}
\newpage
\noindent\begin{tabularx}{\textwidth}{|X|X|X|X|X|}
\hline
\rowcolor{headerbg} \textcolor{white}{\textbf{Nom français}} & \textcolor{white}{\textbf{Nom latin}} & \textcolor{white}{\textbf{Famille}} & \textcolor{white}{\textbf{Parties utilisées}} & \textcolor{white}{\textbf{Forme de préparation}}  \\ \hline
\vocnoindexref{https://fr.wikipedia.org/wiki/Aneth}{Aneth fenouil.} & Voir : Fenouil doux. &  &  &  \\ \hline
\vocnoindexref{https://fr.wikipedia.org/wiki/Angélique.angélique}{Angélique.Angélique officinale.} & Angelica archangelica L.(= Archangelica officinalis Hoffm.). & Apiaceae & Fruit. & En l’état - En poudre \\ \hline
\vocnoindexref{https://fr.wikipedia.org/wiki/Anis.anis}{Anis.Anis vert.} & Pimpinella anisum L. & Apiaceae & Fruit. & En l’état - En poudre \\ \hline
\vocnoindexref{https://fr.wikipedia.org/wiki/Anis}{Anis étoilé.} & Voir : Badianier de Chine. &  &  &  \\ \hline
\vocnoindexref{https://fr.wikipedia.org/wiki/Ascophyllum.}{Ascophyllum.} & Ascophyllum nodosum Le Jol. & Phaeophyceae & Thalle. & En l’état - En poudre - Extrait sec aqueux \\ \hline
\vocnoindexref{https://fr.wikipedia.org/wiki/Aspérule}{Aspérule odorante.} & Galium odoratum (L.) Scop.(= Asperula odorata L.). & Rubiaceae & Partie aérienne fleurie. & En l’état \\ \hline
\vocnoindexref{https://fr.wikipedia.org/wiki/Aspic.lavande}{Aspic.Lavande aspic.} & Lavandula latifolia (L. f.) Medik. & Lamiaceae & Sommité fleurie. & En l’état \\ \hline
\vocnoindexref{https://fr.wikipedia.org/wiki/Astragale}{Astragale à gomme.Gomme adragante.} & Astragalus gummifer (Labill.) et certaines espèces du genre Astragalus d’Asie occidentale. & Fabaceae & Exsudation gommeuse = gomme adragante. & En l’état - En poudre - Extrait sec aqueux \\ \hline
\vocnoindexref{https://fr.wikipedia.org/wiki/Aubépine.epine}{Aubépine.Epine blanche.} & Crataegus laevigata (Poir.) DC.,C. monogyna Jacq. (Lindm.)(= C. oxyacanthoïdes Thuill.). & Rosaceae & Fruit. & En l’état \\ \hline
\vocnoindexref{https://fr.wikipedia.org/wiki/Aunée.aunée}{Aunée.Aunée officinale.} & Inula helenium L. & Asteraceae & Partie souterraine. & En l’état - En poudre \\ \hline
\vocnoindexref{https://fr.wikipedia.org/wiki/Avoine.}{Avoine.} & Avena sativa L. & Poaceae & Fruit. & En l’état - En poudre \\ \hline
\end{tabularx}
\newpage
\noindent\begin{tabularx}{\textwidth}{|X|X|X|X|X|}
\hline
\rowcolor{headerbg} \textcolor{white}{\textbf{Nom français}} & \textcolor{white}{\textbf{Nom latin}} & \textcolor{white}{\textbf{Famille}} & \textcolor{white}{\textbf{Parties utilisées}} & \textcolor{white}{\textbf{Forme de préparation}}  \\ \hline
\vocnoindexref{https://fr.wikipedia.org/wiki/Balsamite}{Balsamite odorante.Menthe coq.} & Balsamita major Desf.(= Chrysanthemum balsamita [L.] Baill.). & Asteraceae & Feuille, sommité fleurie. & En l’état \\ \hline
\vocnoindexref{https://fr.wikipedia.org/wiki/Bardane}{Bardane (grande).} & Arctium lappa L.(= A. majus [Gaertn.] Bernh.)(= Lappa major Gaertn.). & Asteraceae & Feuille, racine. & En l’état \\ \hline
\vocnoindexref{https://fr.wikipedia.org/wiki/Basilic.basilic}{Basilic.Basilic doux.} & Ocimum basilicum L. & Lamiaceae & Feuille. & En l’état - En poudre \\ \hline
\vocnoindexref{https://fr.wikipedia.org/wiki/Baumier}{Baumier de Copahu.Baume de Copahu.} & Copaifera officinalis L.,C. guyanensis Desf.,C. lansdorfii Desf. & Fabaceae & Oléo-résine dite baume de copahu » . & En l’état \\ \hline
\vocnoindexref{https://fr.wikipedia.org/wiki/Bétoine.}{Bétoine.} & Stachys officinalis (L.) Trevis.(= Betonica officinalis L.). & Lamiaceae & Feuille. & En l’état \\ \hline
\vocnoindexref{https://fr.wikipedia.org/wiki/Bigaradier.}{Bigaradier.} & Voir : Oranger amer. &  &  &  \\ \hline
\vocnoindexref{https://fr.wikipedia.org/wiki/Blé.}{Blé.} & Triticum aestivum L. et cultivars(= T. vulgare Host)(= T. sativum Lam.). & Poaceae & Son. & En l’état - En poudre \\ \hline
\vocnoindexref{https://fr.wikipedia.org/wiki/Bouillon}{Bouillon blanc.} & Verbascum thapsus L.,V. densiflorum Bertol.(= V. thapsiforme Schrad.),V. phlomoides L. & Scrophulariaceae & Corolle mondée. & En l’état \\ \hline
\vocnoindexref{https://fr.wikipedia.org/wiki/Bourrache.}{Bourrache.} & Borago officinalis L. & Boraginaceae & Fleur. & En l’état \\ \hline
\vocnoindexref{https://fr.wikipedia.org/wiki/Bruyère}{Bruyère cendrée.} & Erica cinerea L. & Ericaceae & Fleur. & En l’état \\ \hline
\vocnoindexref{https://fr.wikipedia.org/wiki/Camomille}{Camomille allemande.} & Voir : Matricaire. &  &  &  \\ \hline
\vocnoindexref{https://fr.wikipedia.org/wiki/Camomille}{Camomille romaine.} & Chamaemelum nobile (L.) All.(= Anthemis nobilis L.). & Asteraceae & Capitule. & En l’état \\ \hline
\vocnoindexref{https://fr.wikipedia.org/wiki/Camomille}{Camomille vulgaire.} & Voir : Matricaire. &  &  &  \\ \hline
\vocnoindexref{https://fr.wikipedia.org/wiki/Canéficier.}{Canéficier.} & Cassia fistula L. & Fabaceae & Pulpe de fruit. & En l’état \\ \hline
\end{tabularx}
\newpage
\noindent\begin{tabularx}{\textwidth}{|X|X|X|X|X|}
\hline
\rowcolor{headerbg} \textcolor{white}{\textbf{Nom français}} & \textcolor{white}{\textbf{Nom latin}} & \textcolor{white}{\textbf{Famille}} & \textcolor{white}{\textbf{Parties utilisées}} & \textcolor{white}{\textbf{Forme de préparation}}  \\ \hline
\vocnoindexref{https://fr.wikipedia.org/wiki/Cannelier}{Cannelier de Ceylan.Cannelle de Ceylan.} & Cinnamomum zeylanicum Nees. & Lauraceae & Ecorce de tige raclée = cannelle de Ceylan. & En l’état - En poudre \\ \hline
\vocnoindexref{https://fr.wikipedia.org/wiki/Cannelier}{Cannelier de Chine.Cannelle de Chine.} & Cinnamomum aromaticum Nees,C. cassia Nees ex Blume. & Lauraceae & Ecorce de tige = cannelle de Chine. & En l’état - En poudre \\ \hline
\vocnoindexref{https://fr.wikipedia.org/wiki/Capucine.}{Capucine.} & Tropaeolum majus L. & Tropaeolaceae & Feuille. & En l’état \\ \hline
\vocnoindexref{https://fr.wikipedia.org/wiki/Cardamome.}{Cardamome.} & Elettaria cardamomum (L.) Maton. & Zingiberaceae & Fruit. & En l’état - En poudre \\ \hline
\vocnoindexref{https://fr.wikipedia.org/wiki/Caroubier.gomme}{Caroubier.Gomme caroube.} & Ceratonia siliqua L. & Fabaceae & Graine mondée = gomme caroube. & En l’état - En poudre \\ \hline
\vocnoindexref{https://fr.wikipedia.org/wiki/Carragaheen.mousse}{Carragaheen.Mousse d’Irlande.} & Chondrus crispus Lingby. & Gigartinaceae & Thalle. & En l’état \\ \hline
\vocnoindexref{https://fr.wikipedia.org/wiki/Carthame.}{Carthame.} & Carthamus tinctorius L. & Asteraceae & Fleur. & En l’état \\ \hline
\vocnoindexref{https://fr.wikipedia.org/wiki/Carvi.cumin}{Carvi.Cumin des prés.} & Carum carvi L. & Apiaceae & Fruit. & En l’état - En poudre \\ \hline
\vocnoindexref{https://fr.wikipedia.org/wiki/Cassissier.groseiller}{Cassissier.Groseiller noir.} & Ribes nigrum L. & Grossulariaceae & Feuille, fruit. & En l’état \\ \hline
\vocnoindexref{https://fr.wikipedia.org/wiki/Centaurée}{Centaurée (petite).} & Centaurium erythraea Raf.(= Erythraea centaurium [L.] Persoon)(= C. minus Moench)(= C. umbellatum Gilib.). & Gentianaceae & Sommité fleurie. & En l’état \\ \hline
\vocnoindexref{https://fr.wikipedia.org/wiki/Cerisier}{Cerisier griottier.} & Voir : Griottier. &  &  &  \\ \hline
\vocnoindexref{https://fr.wikipedia.org/wiki/Chicorée.}{Chicorée.} & Cichorium intybus L. & Asteraceae & Feuille, racine. & En l’état \\ \hline
\vocnoindexref{https://fr.wikipedia.org/wiki/Chiendent}{Chiendent (gros).Chiendent pied de poule.} & Cynodon dactylon (L.) Pers. & Poaceae & Rhizome. & En l’état \\ \hline
\vocnoindexref{https://fr.wikipedia.org/wiki/Chiendent.chiendent}{Chiendent.Chiendent (petit).} & Elytrigia repens [L.] Desv. ex Nevski(= Agropyron repens [L.] Beauv.)(= Elymus repens [L.] Goudl.). & Poaceae & Rhizome. & En l’état \\ \hline
\vocnoindexref{https://fr.wikipedia.org/wiki/Citronnelles.}{Citronnelles.} & Cymbopogon sp. & Poaceae & Feuille. & En l’état - En poudre \\ \hline
\end{tabularx}
\newpage
\noindent\begin{tabularx}{\textwidth}{|X|X|X|X|X|}
\hline
\rowcolor{headerbg} \textcolor{white}{\textbf{Nom français}} & \textcolor{white}{\textbf{Nom latin}} & \textcolor{white}{\textbf{Famille}} & \textcolor{white}{\textbf{Parties utilisées}} & \textcolor{white}{\textbf{Forme de préparation}}  \\ \hline
\vocnoindexref{https://fr.wikipedia.org/wiki/Citrouille.}{Citrouille.} & Voir : Courge citrouille. &  &  &  \\ \hline
\vocnoindexref{https://fr.wikipedia.org/wiki/Clou}{Clou de girofle.} & Voir : Giroflier. &  &  &  \\ \hline
\vocnoindexref{https://fr.wikipedia.org/wiki/Cochléaire.}{Cochléaire.} & Cochlearia officinalis L. & Brassicaceae & Feuille. & En l’état \\ \hline
\vocnoindexref{https://fr.wikipedia.org/wiki/Colatier.}{Colatier.} & Voir : Kolatier. &  &  &  \\ \hline
\vocnoindexref{https://fr.wikipedia.org/wiki/Coquelicot.}{Coquelicot.} & Papaver rhoeas L.,P. dubium L. & Papaveraceae & Pétale. & En l’état \\ \hline
\vocnoindexref{https://fr.wikipedia.org/wiki/Coqueret.}{Coqueret.} & Voir : Alkékenge. &  &  &  \\ \hline
\vocnoindexref{https://fr.wikipedia.org/wiki/Coriandre.}{Coriandre.} & Coriandrum sativum L. & Apiaceae & Fruit. & En l’état - En poudre \\ \hline
\vocnoindexref{https://fr.wikipedia.org/wiki/Courge}{Courge citrouille.Citrouille.} & Cucurbita pepo L.. & Cucurbitaceae & Graine. & En l’état \\ \hline
\vocnoindexref{https://fr.wikipedia.org/wiki/Courge.potiron.}{Courge.Potiron.} & Cucurbita maxima Lam. & Cucurbitaceae & Graine. & En l’état \\ \hline
\vocnoindexref{https://fr.wikipedia.org/wiki/Criste}{Criste marine.Perce-pierre.} & Crithmum maritimum L.. & Apiaceae & Partie aérienne. & En l’état \\ \hline
\vocnoindexref{https://fr.wikipedia.org/wiki/Cumin}{Cumin des prés.} & Voir : Carvi. &  &  &  \\ \hline
\vocnoindexref{https://fr.wikipedia.org/wiki/Curcuma}{Curcuma long.} & Curcuma domestica Vahl(= C. longa L.). & Zingiberaceae & Rhizome. & En l’état - En poudre \\ \hline
\vocnoindexref{https://fr.wikipedia.org/wiki/Cyamopsis.gomme}{Cyamopsis.Gomme guar.Guar.} & Cyamopsis tetragonolobus (L.) Taub. & Fabaceae & Graine mondée = gomme guar. & En l’état - En poudre - Extrait sec aqueux \\ \hline
\vocnoindexref{https://fr.wikipedia.org/wiki/Eglantier.cynorrhodon.rosier}{Eglantier.Cynorrhodon.Rosier sauvage.} & Rosa canina L., R. pendulina L. et autres espèces de Rosa. & Rosaceae & Pseudo-fruit = cynorrhodon. & En l’état \\ \hline
\vocnoindexref{https://fr.wikipedia.org/wiki/Eleuthérocoque.}{Eleuthérocoque.} & Eleutherococcus senticosus Maxim. & Araliaceae & Partie souterraine. & En l’état \\ \hline
\end{tabularx}
\newpage
\noindent\begin{tabularx}{\textwidth}{|X|X|X|X|X|}
\hline
\rowcolor{headerbg} \textcolor{white}{\textbf{Nom français}} & \textcolor{white}{\textbf{Nom latin}} & \textcolor{white}{\textbf{Famille}} & \textcolor{white}{\textbf{Parties utilisées}} & \textcolor{white}{\textbf{Forme de préparation}}  \\ \hline
\vocnoindexref{https://fr.wikipedia.org/wiki/Estragon.}{Estragon.} & Artemisia dracunculus L. & Asteraceae & Partie aérienne. & En l’état - En poudre \\ \hline
\vocnoindexref{https://fr.wikipedia.org/wiki/Eucalyptus.eucalyptus}{Eucalyptus.Eucalyptus globuleux.} & Eucalyptus globulus Labill. & Myrtaceae & Feuille. & En l’état \\ \hline
\vocnoindexref{https://fr.wikipedia.org/wiki/Fenouil}{Fenouil amer.} & Foeniculum vulgare Mill. var. vulgare. & Apiaceae & Fruit. & En l’état - En poudre \\ \hline
\vocnoindexref{https://fr.wikipedia.org/wiki/Fenouil}{Fenouil doux.Aneth fenouil.} & Foeniculum vulgare Mill. var. dulcis. & Apiaceae & Fruit. & En l’état - En poudre \\ \hline
\vocnoindexref{https://fr.wikipedia.org/wiki/Fenugrec.}{Fenugrec.} & Trigonella foenum-graecum L. & Fabaceae & Graine. & En l’état - En poudre \\ \hline
\vocnoindexref{https://fr.wikipedia.org/wiki/Févier.}{Févier.} & Voir : Gléditschia. &  &  &  \\ \hline
\vocnoindexref{https://fr.wikipedia.org/wiki/Figuier.}{Figuier.} & Ficus carica L. & Moraceae & Pseudo-fruit. & En l’état \\ \hline
\vocnoindexref{https://fr.wikipedia.org/wiki/Frêne.}{Frêne.} & Fraxinus excelsior L.,F. oxyphylla M. Bieb. & Oleaceae & Feuille. & En l’état \\ \hline
\vocnoindexref{https://fr.wikipedia.org/wiki/Frêne}{Frêne à manne.} & Fraxinus ornus L. & Oleaceae & Suc épaissi dit manne ». & En l’état - En poudre \\ \hline
\vocnoindexref{https://fr.wikipedia.org/wiki/Fucus.}{Fucus.} & Fucus serratus L.,F. vesiculosus L. & Fucaceae & Thalle. & En l’état - En poudre \\ \hline
\end{tabularx}
\newpage
\noindent\begin{tabularx}{\textwidth}{|X|X|X|X|X|}
\hline
\rowcolor{headerbg} \textcolor{white}{\textbf{Nom français}} & \textcolor{white}{\textbf{Nom latin}} & \textcolor{white}{\textbf{Famille}} & \textcolor{white}{\textbf{Parties utilisées}} & \textcolor{white}{\textbf{Forme de préparation}}  \\ \hline
\vocnoindexref{https://fr.wikipedia.org/wiki/Galanga}{Galanga (petit).} & Alpinia officinarum Hance. & Zingiberaceae & Rhizome. & En l’état - En poudre \\ \hline
\vocnoindexref{https://fr.wikipedia.org/wiki/Genévrier.genièvre.}{Genévrier.Genièvre.} & Juniperus communis L. & Cupressaceae & Cône femelle dit baie de genièvre ». & En l’état \\ \hline
\vocnoindexref{https://fr.wikipedia.org/wiki/Gentiane.gentiane}{Gentiane.Gentiane jaune.} & Gentiana lutea L. & Gentianaceae & Partie souterraine. & En l’état - En poudre \\ \hline
\vocnoindexref{https://fr.wikipedia.org/wiki/Gingembre.}{Gingembre.} & Zingiber officinale Roscoe. & Zingiberaceae & Rhizome. & En l’état - En poudre \\ \hline
\vocnoindexref{https://fr.wikipedia.org/wiki/Ginseng.panax}{Ginseng.Panax de Chine.} & Panax ginseng C.A. Meyer(= Aralia quinquefolia Decne. et Planch.). & Araliaceae & Partie souterraine. & En l’état - En poudre - Extrait sec aqueux \\ \hline
\vocnoindexref{https://fr.wikipedia.org/wiki/Giroflier.}{Giroflier.} & Syzygium aromaticum (L.) Merr. et Perry(= Eugenia caryophyllus (Sprengel) Bull. et Harr.). & Myrtaceae & Bouton floral = clou de girofle. & En l’état - En poudre \\ \hline
\vocnoindexref{https://fr.wikipedia.org/wiki/Gléditschia.févier.}{Gléditschia.Févier.} & Gleditschia triacanthos L.,G. ferox Desf. & Fabaceae & Graine. & En l’état - En poudre - Extrait sec aqueux \\ \hline
\vocnoindexref{https://fr.wikipedia.org/wiki/Gomme}{Gomme adragante.} & Voir : Astragale à gomme. &  &  &  \\ \hline
\vocnoindexref{https://fr.wikipedia.org/wiki/Gomme}{Gomme arabique.} & Voir : Acacia à gomme. &  &  &  \\ \hline
\vocnoindexref{https://fr.wikipedia.org/wiki/Gomme}{Gomme caroube.} & Voir : Caroubier. &  &  &  \\ \hline
\vocnoindexref{https://fr.wikipedia.org/wiki/Gomme}{Gomme de sterculia.} & Voir : Sterculia. &  &  &  \\ \hline
\vocnoindexref{https://fr.wikipedia.org/wiki/Gomme}{Gomme guar.} & Voir : Cyamopsis. &  &  &  \\ \hline
\vocnoindexref{https://fr.wikipedia.org/wiki/Gomme}{Gomme Karaya.} & Voir : Sterculia. &  &  &  \\ \hline
\vocnoindexref{https://fr.wikipedia.org/wiki/Gomme}{Gomme M’Bep.} & Voir : Sterculia. &  &  &  \\ \hline
\end{tabularx}
\newpage
\noindent\begin{tabularx}{\textwidth}{|X|X|X|X|X|}
\hline
\rowcolor{headerbg} \textcolor{white}{\textbf{Nom français}} & \textcolor{white}{\textbf{Nom latin}} & \textcolor{white}{\textbf{Famille}} & \textcolor{white}{\textbf{Parties utilisées}} & \textcolor{white}{\textbf{Forme de préparation}}  \\ \hline
\vocnoindexref{https://fr.wikipedia.org/wiki/Griottier.cerisier}{Griottier.Cerisier griottier.Queue de cerise.} & Prunus cerasus L.,P. avium (L.) L. & Rosaceae & Pédoncule du fruit = queue de cerise. & En l’état \\ \hline
\vocnoindexref{https://fr.wikipedia.org/wiki/Groseiller}{Groseiller noir.} & Voir : Cassissier. &  &  &  \\ \hline
\vocnoindexref{https://fr.wikipedia.org/wiki/Guar.}{Guar.} & Voir : Cyamopsis. &  &  &  \\ \hline
\vocnoindexref{https://fr.wikipedia.org/wiki/Guarana.}{Guarana.} & Voir : Paullinia. &  &  &  \\ \hline
\vocnoindexref{https://fr.wikipedia.org/wiki/Guimauve.}{Guimauve.} & Althaea officinalis L. & Malvaceae & Feuille, fleur, racine. & En l’état - En poudre (racine) \\ \hline
\vocnoindexref{https://fr.wikipedia.org/wiki/Hibiscus.}{Hibiscus.} & Voir : Karkadé. &  &  &  \\ \hline
\vocnoindexref{https://fr.wikipedia.org/wiki/Houblon.}{Houblon.} & Humulus lupulus L. & Cannabaceae & Inflorescence femelle dite cône de houblon ». & En l’état \\ \hline
\vocnoindexref{https://fr.wikipedia.org/wiki/Jujubier.}{Jujubier.} & Ziziphus jujuba Mill.(= Z. sativa Gaertn.)(= Z. vulgaris Lam.)(= Rhamnus zizyphus L.). & Rhamnaceae & Fruit privé de graines. & En l’état \\ \hline
\vocnoindexref{https://fr.wikipedia.org/wiki/Karkadé.oseille}{Karkadé.Oseille de Guinée.Hibiscus.} & Hibiscus sabdariffa L. & Malvaceae & Calice et calicule. & En l’état \\ \hline
\vocnoindexref{https://fr.wikipedia.org/wiki/Kolatier.colatier.kola.}{Kolatier.Colatier.Kola.} & Cola acuminata (P. Beauv.) Schott et Endl.(= Sterculia acuminata P. Beauv.),C. nitida (Vent.) Schott et Endl.(= C. vera K. Schum.) et variétés. & Sterculiaceae & Amande dite noix de kola ». & En l’état - En poudre \\ \hline
\vocnoindexref{https://fr.wikipedia.org/wiki/Lamier}{Lamier blanc.Ortie blanche.} & Lamium album L. & Lamiaceae & Corolle mondée, sommité fleurie. & En l’état \\ \hline
\vocnoindexref{https://fr.wikipedia.org/wiki/Laminaire.}{Laminaire.} & Laminaria digitata J.P. Lamour.,L. hyperborea (Gunnerus) Foslie,L. cloustonii Le Jol. & Laminariaceae & Stipe, thalle. & En l’état - Extrait sec aqueux (thalle) \\ \hline
\vocnoindexref{https://fr.wikipedia.org/wiki/Laurier}{Laurier commun.Laurier sauce.} & Laurus nobilis L. & Lauraceae & Feuille. & En l’état - En poudre \\ \hline
\vocnoindexref{https://fr.wikipedia.org/wiki/Lavande.lavande}{Lavande.Lavande vraie.} & Lavandula angustifolia Mill.(= L. vera DC.). & Lamiaceae & Fleur, sommité fleurie. & En l’état \\ \hline
\vocnoindexref{https://fr.wikipedia.org/wiki/Lavande}{Lavande aspic.} & Voir : Aspic. &  &  &  \\ \hline
\end{tabularx}
\newpage
\noindent\begin{tabularx}{\textwidth}{|X|X|X|X|X|}
\hline
\rowcolor{headerbg} \textcolor{white}{\textbf{Nom français}} & \textcolor{white}{\textbf{Nom latin}} & \textcolor{white}{\textbf{Famille}} & \textcolor{white}{\textbf{Parties utilisées}} & \textcolor{white}{\textbf{Forme de préparation}}  \\ \hline
\vocnoindexref{https://fr.wikipedia.org/wiki/Lavande}{Lavande stoechas.} & Lavandula stoechas L. & Lamiaceae & Fleur, sommité fleurie. & En l’état \\ \hline
\vocnoindexref{https://fr.wikipedia.org/wiki/Lavande}{Lavande vraie.} & Voir : Lavande. &  &  &  \\ \hline
\vocnoindexref{https://fr.wikipedia.org/wiki/Lavandin}{Lavandin Grosso ».} & Lavandula × intermedia Emeric ex Loisel. & Lamiaceae & Fleur, sommité fleurie. & En l’état \\ \hline
\vocnoindexref{https://fr.wikipedia.org/wiki/Lemongrass}{Lemongrass de l’Amérique centrale.} & Cymbopogon citratus (DC.) Stapf. & Poaceae & Feuille. & En l’état - En poudre \\ \hline
\vocnoindexref{https://fr.wikipedia.org/wiki/Lemongrass}{Lemongrass de l’Inde.} & Cymbopogon flexuosus (Nees ex Steud.) J.F. Wats. & Poaceae & Feuille. & En l’état - En poudre \\ \hline
\vocnoindexref{https://fr.wikipedia.org/wiki/Lichen}{Lichen d’Islande.} & Cetraria islandica (L.) Ach. sensu latiore. & Parmeliaceae & Thalle. & En l’état \\ \hline
\vocnoindexref{https://fr.wikipedia.org/wiki/Lierre}{Lierre terrestre.} & Glechoma hederacea L.(= Nepeta glechoma Benth.). & Lamiaceae & Partie aérienne fleurie. & En l’état \\ \hline
\vocnoindexref{https://fr.wikipedia.org/wiki/Lin.}{Lin.} & Linum usitatissimum L. & Linaceae & Graine. & En l’état - En poudre \\ \hline
\vocnoindexref{https://fr.wikipedia.org/wiki/Livèche.}{Livèche.} & Levisticum officinale Koch. & Apiaceae & Feuille, fruit, partie souterraine. & En l’état - En poudre \\ \hline
\end{tabularx}
\newpage
\noindent\begin{tabularx}{\textwidth}{|X|X|X|X|X|}
\hline
\rowcolor{headerbg} \textcolor{white}{\textbf{Nom français}} & \textcolor{white}{\textbf{Nom latin}} & \textcolor{white}{\textbf{Famille}} & \textcolor{white}{\textbf{Parties utilisées}} & \textcolor{white}{\textbf{Forme de préparation}}  \\ \hline
\vocnoindexref{https://fr.wikipedia.org/wiki/Marjolaine.origan}{Marjolaine.Origan marjolaine.} & Origanum majorana L.(= Majorana hortensis Moench). & Lamiaceae & Feuille, sommité fleurie. & En l’état - En poudre \\ \hline
\vocnoindexref{https://fr.wikipedia.org/wiki/Maté.thé}{Maté.Thé du Paraguay.} & Ilex paraguariensis St.-Hil.(= I. paraguayensis Lamb.). & Aquifoliaceae & Feuille. & En l’état - Extrait sec aqueux \\ \hline
\vocnoindexref{https://fr.wikipedia.org/wiki/Matricaire.camomille}{Matricaire.Camomille allemande.Camomille vulgaire.} & Matricaria recutita L.(= Chamomilla recutita [L.] Rausch.)(= M. chamomilla L.). & Asteraceae & Capitule. & En l’état \\ \hline
\vocnoindexref{https://fr.wikipedia.org/wiki/Mauve.}{Mauve.} & Malva sylvestris L. & Malvaceae & Feuille, fleur. & En l’état \\ \hline
\vocnoindexref{https://fr.wikipedia.org/wiki/Mélisse.}{Mélisse.} & Melissa officinalis L. & Lamiaceae & Feuille, sommité fleurie. & En l’état \\ \hline
\vocnoindexref{https://fr.wikipedia.org/wiki/Menthe}{Menthe coq.} & Voir : Balsamite odorante. &  &  &  \\ \hline
\vocnoindexref{https://fr.wikipedia.org/wiki/Menthe}{Menthe poivrée.} & Mentha × piperita L. & Lamiaceae & Feuille, sommité fleurie. & En l’état \\ \hline
\vocnoindexref{https://fr.wikipedia.org/wiki/Menthe}{Menthe verte.} & Mentha spicata L. (= M. viridis L.). & Lamiaceae & Feuille, sommité fleurie. & En l’état \\ \hline
\vocnoindexref{https://fr.wikipedia.org/wiki/Ményanthe.trèfle}{Ményanthe.Trèfle d’eau.} & Menyanthes trifoliata L. & Menyanthaceae & Feuille. & En l’état \\ \hline
\vocnoindexref{https://fr.wikipedia.org/wiki/Millefeuille.}{Millefeuille.} & Voir : Achillée millefeuille. &  &  &  \\ \hline
\vocnoindexref{https://fr.wikipedia.org/wiki/Mousse}{Mousse d’Irlande.} & Voir : Carragaheen. &  &  &  \\ \hline
\vocnoindexref{https://fr.wikipedia.org/wiki/Moutarde}{Moutarde junciforme.} & Brassica juncea (L.) Czern. & Brassicaceae & Graine. & En l’état - En poudre \\ \hline
\vocnoindexref{https://fr.wikipedia.org/wiki/Muscadier}{Muscadier aromatique.Macis.Muscade.} & Myristica fragrans Houtt.(= M. moschata Thunb.). & Myristicaceae & Graine dite muscade » ou noix de muscade », arille dite macis ». & En l’état - En poudre (graine) \\ \hline
\vocnoindexref{https://fr.wikipedia.org/wiki/Myrte.}{Myrte.} & Myrtus communis L. & Myrtaceae & Feuille. & En l’état \\ \hline
\end{tabularx}
\newpage
\noindent\begin{tabularx}{\textwidth}{|X|X|X|X|X|}
\hline
\rowcolor{headerbg} \textcolor{white}{\textbf{Nom français}} & \textcolor{white}{\textbf{Nom latin}} & \textcolor{white}{\textbf{Famille}} & \textcolor{white}{\textbf{Parties utilisées}} & \textcolor{white}{\textbf{Forme de préparation}}  \\ \hline
\vocnoindexref{https://fr.wikipedia.org/wiki/Myrtille.airelle}{Myrtille.Airelle myrtille.} & Vaccinium myrtillus L. & Ericaceae & Feuille, fruit. & En l’état \\ \hline
\vocnoindexref{https://fr.wikipedia.org/wiki/Olivier.}{Olivier.} & Olea europaea L. & Oleaceae & Feuille. & En l’état \\ \hline
\vocnoindexref{https://fr.wikipedia.org/wiki/Oranger}{Oranger amer.Bigaradier.} & Citrus aurantium L.(= C. bigaradia Duch.)(= C. vulgaris Risso). & Rutaceae & Feuille, fleur, péricarpe dit écorce » ou zeste. & En l’état - En poudre (péricarpe) \\ \hline
\vocnoindexref{https://fr.wikipedia.org/wiki/Oranger}{Oranger doux.} & Citrus sinensis (L.) Pers.(= C. aurantium L.). & Rutaceae & Péricarpe dit écorce » ou zeste. & En l’état - En poudre \\ \hline
\vocnoindexref{https://fr.wikipedia.org/wiki/Origan.}{Origan.} & Origanum vulgare L. & Lamiaceae & Feuille, sommité fleurie. & En l’état - En poudre \\ \hline
\vocnoindexref{https://fr.wikipedia.org/wiki/Origan}{Origan marjolaine.} & Voir : Marjolaine. &  &  &  \\ \hline
\vocnoindexref{https://fr.wikipedia.org/wiki/Ortie}{Ortie blanche.} & Voir : Lamier blanc. &  &  &  \\ \hline
\vocnoindexref{https://fr.wikipedia.org/wiki/Ortie}{Ortie brûlante.} & Urtica urens L. & Urticaceae & Partie aérienne. & En l’état \\ \hline
\vocnoindexref{https://fr.wikipedia.org/wiki/Ortie}{Ortie dioïque.} & Urtica dioica L. & Urticaceae & Partie aérienne. & En l’état \\ \hline
\vocnoindexref{https://fr.wikipedia.org/wiki/Oseille}{Oseille de Guinée} & Voir : Karkadé. &  &  &  \\ \hline
\vocnoindexref{https://fr.wikipedia.org/wiki/Panax}{Panax de Chine} & Voir : Ginseng. &  &  &  \\ \hline
\vocnoindexref{https://fr.wikipedia.org/wiki/Papayer.}{Papayer.} & Carica papaya L. & Caricaceae & Suc du fruit, feuille. & En l’état - En poudre (suc du fruit) \\ \hline
\vocnoindexref{https://fr.wikipedia.org/wiki/Passerose.}{Passerose.} & Voir : Rose trémière. &  &  &  \\ \hline
\vocnoindexref{https://fr.wikipedia.org/wiki/Paullinia.guarana.}{Paullinia.Guarana.} & Paullinia cupana Kunth.(= P. sorbilis Mart.). & Sapindaceae & Graine, extrait préparé avec la graine = guarana. & En l’état - En poudre (extrait) \\ \hline
\vocnoindexref{https://fr.wikipedia.org/wiki/Pensée}{Pensée sauvage.Violette tricolore.} & Viola arvensis Murray,V. tricolor L. & Violaceae & Fleur, partie aérienne fleurie. & En l’état \\ \hline
\end{tabularx}
\newpage
\noindent\begin{tabularx}{\textwidth}{|X|X|X|X|X|}
\hline
\rowcolor{headerbg} \textcolor{white}{\textbf{Nom français}} & \textcolor{white}{\textbf{Nom latin}} & \textcolor{white}{\textbf{Famille}} & \textcolor{white}{\textbf{Parties utilisées}} & \textcolor{white}{\textbf{Forme de préparation}}  \\ \hline
\vocnoindexref{https://fr.wikipedia.org/wiki/Perce-pierre.}{Perce-pierre.} & Voir : Criste marine. &  &  &  \\ \hline
\vocnoindexref{https://fr.wikipedia.org/wiki/Piment}{Piment de Cayenne.Piment enragé.Piment (petit).} & Capsicum frutescens L. & Solanaceae & Fruit. & En l’état - En poudre \\ \hline
\vocnoindexref{https://fr.wikipedia.org/wiki/Pin}{Pin sylvestre.} & Pinus sylvestris L. & Pinaceae & Bourgeon. & En l’état \\ \hline
\vocnoindexref{https://fr.wikipedia.org/wiki/Pissenlit.dent}{Pissenlit.Dent de lion.} & Taraxacum officinale Web. & Asteraceae & Feuille, partie aérienne. & En l’état \\ \hline
\vocnoindexref{https://fr.wikipedia.org/wiki/Pommier.}{Pommier.} & Malus sylvestris Mill.(= Pyrus malus L.). & Rosaceae & Fruit. & En l’état \\ \hline
\vocnoindexref{https://fr.wikipedia.org/wiki/Potiron.}{Potiron.} & Voir : Courge. &  &  &  \\ \hline
\vocnoindexref{https://fr.wikipedia.org/wiki/Prunier.}{Prunier.} & Prunus domestica L. & Rosaceae & Fruit. & En l’état \\ \hline
\vocnoindexref{https://fr.wikipedia.org/wiki/Queue}{Queue de cerise.} & Voir : Griottier. &  &  &  \\ \hline
\vocnoindexref{https://fr.wikipedia.org/wiki/Radis}{Radis noir.} & Raphanus sativus L. var. niger (Mill.) Kerner. & Brassicaceae & Racine. & En l’état \\ \hline
\vocnoindexref{https://fr.wikipedia.org/wiki/Raifort}{Raifort sauvage.} & Armoracia rusticana Gaertn., B. Mey. et Scherb.(= Cochlearia armoracia L.). & Brassicaceae & Racine. & En l’état - En poudre \\ \hline
\vocnoindexref{https://fr.wikipedia.org/wiki/Réglisse.}{Réglisse.} & Glycyrrhiza glabra L. & Fabaceae & Partie souterraine. & En l’état - En poudre - Extrait sec aqueux \\ \hline
\vocnoindexref{https://fr.wikipedia.org/wiki/Reine-des-prés.ulmaire.}{Reine-des-prés.Ulmaire.} & Filipendula ulmaria (L.) Maxim.(= Spiraea ulmaria L.). & Rosaceae & Fleur, sommité fleurie. & En l’état \\ \hline
\vocnoindexref{https://fr.wikipedia.org/wiki/Romarin.}{Romarin.} & Rosmarinus officinalis L. & Lamiaceae & Feuille, sommité fleurie. & En l’état - En poudre \\ \hline
\vocnoindexref{https://fr.wikipedia.org/wiki/Ronce.}{Ronce.} & Rubus sp. & Rosaceae & Feuille. & En l’état \\ \hline
\vocnoindexref{https://fr.wikipedia.org/wiki/Rose}{Rose trémière.Passerose.} & Alcea rosea L.(= Althaea rosea L.). & Malvaceae & Fleur. & En l’état \\ \hline
\end{tabularx}
\newpage
\noindent\begin{tabularx}{\textwidth}{|X|X|X|X|X|}
\hline
\rowcolor{headerbg} \textcolor{white}{\textbf{Nom français}} & \textcolor{white}{\textbf{Nom latin}} & \textcolor{white}{\textbf{Famille}} & \textcolor{white}{\textbf{Parties utilisées}} & \textcolor{white}{\textbf{Forme de préparation}}  \\ \hline
\vocnoindexref{https://fr.wikipedia.org/wiki/Rosier}{Rosier à roses pâles.} & Rosa centifolia L. & Rosaceae & Bouton floral, pétale. & En l’état \\ \hline
\vocnoindexref{https://fr.wikipedia.org/wiki/Rosier}{Rosier de Damas.} & Rosa damascena Mill. & Rosaceae & Bouton floral, pétale. & En l’état \\ \hline
\vocnoindexref{https://fr.wikipedia.org/wiki/Rosier}{Rosier de Provins.Rosier à roses rouges.} & Rosa gallica L. & Rosaceae & Bouton floral, pétale. & En l’état \\ \hline
\vocnoindexref{https://fr.wikipedia.org/wiki/Rosier}{Rosier sauvage.} & Voir : Eglantier. &  &  &  \\ \hline
\vocnoindexref{https://fr.wikipedia.org/wiki/Safran.}{Safran.} & Crocus sativus L. & Iridaceae & Stigmate. & En l’état - En poudre \\ \hline
\vocnoindexref{https://fr.wikipedia.org/wiki/Sarriette}{Sarriette des jardins.} & Satureja hortensis L. & Lamiaceae & Feuille, sommité fleurie. & En l’état - En poudre \\ \hline
\vocnoindexref{https://fr.wikipedia.org/wiki/Sarriette}{Sarriette des montagnes.} & Satureja montana L. & Lamiaceae & Feuille, sommité fleurie. & En l’état - En poudre \\ \hline
\vocnoindexref{https://fr.wikipedia.org/wiki/Sauge}{Sauge d’Espagne.} & Salvia lavandulifolia Vahl. & Lamiaceae & Feuille, sommité fleurie. & En l’état - En poudre \\ \hline
\vocnoindexref{https://fr.wikipedia.org/wiki/Sauge}{Sauge officinale.} & Salvia officinalis L. & Lamiaceae & Feuille. & En l’état \\ \hline
\vocnoindexref{https://fr.wikipedia.org/wiki/Sauge}{Sauge sclarée.Sclarée toute-bonne.} & Salvia sclarea L. & Lamiaceae & Feuille, sommité fleurie. & En l’état - En poudre \\ \hline
\vocnoindexref{https://fr.wikipedia.org/wiki/Sauge}{Sauge trilobée.} & Salvia fruticosa Mill.(= S. triloba L. f.). & Lamiaceae & Feuille. & En l’état - En poudre \\ \hline
\vocnoindexref{https://fr.wikipedia.org/wiki/Seigle.}{Seigle.} & Secale cereale L. & Poaceae & Fruit, son. & En l’état - En poudre \\ \hline
\vocnoindexref{https://fr.wikipedia.org/wiki/Serpolet.thym}{Serpolet.Thym serpolet.} & Thymus serpyllum L. sensu latiore. & Lamiaceae & Feuille, sommité fleurie. & En l’état - En poudre \\ \hline
\vocnoindexref{https://fr.wikipedia.org/wiki/Sterculia.gomme}{Sterculia.Gomme Karaya.Gomme M’Bep.Gomme de Sterculia.} & Sterculia urens Roxb., S. tomentosa Guill. et Perr. & Sterculiaceae & Exsudation gommeuse = gomme de Sterculia, gomme Karaya, gomme M’Bep. & En l’état - En poudre - Extrait sec aqueux \\ \hline
\vocnoindexref{https://fr.wikipedia.org/wiki/Sureau}{Sureau noir.} & Sambucus nigra L. & Caprifoliaceae & Fleur, fruit. & En l’état \\ \hline
\end{tabularx}
\newpage
\noindent\begin{tabularx}{\textwidth}{|X|X|X|X|X|}
\hline
\rowcolor{headerbg} \textcolor{white}{\textbf{Nom français}} & \textcolor{white}{\textbf{Nom latin}} & \textcolor{white}{\textbf{Famille}} & \textcolor{white}{\textbf{Parties utilisées}} & \textcolor{white}{\textbf{Forme de préparation}}  \\ \hline
\vocnoindexref{https://fr.wikipedia.org/wiki/Tamarinier}{Tamarinier de l’Inde.} & Tamarindus indica L. & Fabaceae & Pulpe de fruit. & En l’état - En poudre \\ \hline
\vocnoindexref{https://fr.wikipedia.org/wiki/Temoe-lawacq.}{Temoe-lawacq.} & Curcuma xanthorrhiza Roxb. & Zingiberaceae & Rhizome. & En l’état \\ \hline
\vocnoindexref{https://fr.wikipedia.org/wiki/Thé}{Thé du Paraguay.} & Voir : Maté. &  &  &  \\ \hline
\vocnoindexref{https://fr.wikipedia.org/wiki/Théier.thé.}{Théier.Thé.} & Camellia sinensis (L.) Kuntze(= C. thea Link)(= Thea sinensis (L.) Kuntze). & Theaceae & Feuille. & En l’état - Extrait sec aqueux \\ \hline
\vocnoindexref{https://fr.wikipedia.org/wiki/Thym.}{Thym.} & Thymus vulgaris L.,T. zygis L. & Lamiaceae & Feuille, sommité fleurie. & En l’état - En poudre \\ \hline
\vocnoindexref{https://fr.wikipedia.org/wiki/Thym}{Thym serpolet.} & Voir : Serpolet. &  &  &  \\ \hline
\vocnoindexref{https://fr.wikipedia.org/wiki/Tilleul.}{Tilleul.} & Tilia platyphyllos Scop., T. cordata Mill.(= T. ulmifolia Scop.) (= T. parvifolia Ehrh.ex Hoffm.) (= T. sylvestris Desf.),T. × vulgaris Heyne ou mélanges. & Tiliaceae & Aubier, inflorescence. & En l’état \\ \hline
\vocnoindexref{https://fr.wikipedia.org/wiki/Trèfle}{Trèfle d’eau.} & Voir : Ményanthe. &  &  &  \\ \hline
\vocnoindexref{https://fr.wikipedia.org/wiki/Ulmaire.}{Ulmaire.} & Voir : Reine-des-prés. &  &  &  \\ \hline
\vocnoindexref{https://fr.wikipedia.org/wiki/Verveine}{Verveine odorante.} & Aloysia citrodora Palau(= Aloysia triphylla (L’Hérit.) Britt.)(= Lippia citriodora H.B.K.). & Verbenaceae & Feuille. & En l’état \\ \hline
\vocnoindexref{https://fr.wikipedia.org/wiki/Vigne}{Vigne rouge.} & Vitis vinifera L. & Vitaceae & Feuille. & En l’état \\ \hline
\vocnoindexref{https://fr.wikipedia.org/wiki/Violette.}{Violette.} & Viola calcarata L.,V. lutea Huds.,V. odorata L. & Violaceae & Fleur. & En l’état \\ \hline
\vocnoindexref{https://fr.wikipedia.org/wiki/Violette}{Violette tricolore.} & Voir : Pensée sauvage. &  &  &  \\ \hline
\end{tabularx}

        
    % Insérer l'index
    \printindex

    \newpage
    \section*{Références}
    \addcontentsline{toc}{section}{Références}
	    
\renewcommand{\refname}{}
\begin{thebibliography}{9}

    \bibitem{laporte2023}
    Florence Laporte,
    \textit{Les plantes des druides},
    Rustica editions, 2023,
    Edelvives Espagne,
    Frédérique Chavances,
    ISBN: 978-2-8153-1020-8.

    \bibitem{plantemed}
    auteur inconnu,
    \textit{Les plantes médicinales},
    ?, ????,
    ?,
    ?,
    ?
    
    \bibitem{plantesante}
    Loïc Girre,
    \textit{Infusions et plantes de santé},
    \'Editions OUEST-FRANCE, 2017,
    Rennes France,
    Jérôme Le Bihan,
    ISBN: 978-2-7373-7511-8.

    \bibitem{plantesmed}
    Frédérique Basset,
    \textit{Je sais utiliser mes plantes médicinales},
    Sepec, 2019,
    France,
    Jérôme Le Bihan,
    \no impression : $18483190487$.

\end{thebibliography}
\end{document}