\label{soleilcp}
\ficherecette
{%titre recette
    Crème après soleil
}
{%liste d'ingrédients commençant par \item directement.
    \item $25\%$ de \voc{macérat huileux} de \voc{calendula}
    \item $35\%$ de \voc{macérat huileux} de \voc{paquerette}
    \item $6\%$ de \voc{cire d'abeille}
    \item $6$g de \voc{miel}.
    \item $8$ pulvérisations d'\voc{eau florale de rose}.
    \item $1$ goutte de \voc{propolis}.
}
{%liste du matériel commençant par \item directement.
    \item $1$ \frquote{\voc{cul de poule}} propre. 
    \item $1$ fouet. 
    \item $1$ pot \textbf{désinfecté}, \voc{étanche} et \voc{sec}.
    \item $1$ spatule. 
    \item $1$ casserole et de l'eau pour le bain marie. 
    \item Des plaques chauffantes. 
}
{%Métnodes et conseils de conservation
    \begin{itemize}[label=\faPen]
        \item Conservation courte $\approx 6$ mois.
        \item Dans un endroit sec, de préférence à l'abri de la lumière. 
    \end{itemize}
}
{%Métnodes et conseils d'utilisation    
    \begin{itemize}[label=\faPen]
        \item En application locale sur la zone irritée.
        \item Utiliser un \textbf{ustensile propre} lors de l'utilisation pour prolonger la durée de conservation. 
    \end{itemize}
}
{%Remarques
    \textit{A priori} il n'y a pas de contre-indication pour être à nouveau au soleil après application contrairement à une préparation contenant du millepertuis.
}
{%chemin de l'illustration dans le dossier 'préparations'
    baume/pommades.jpg
}
{%Titre donné à l'illustration dans le document latex
    Crème au après solaire
}
{%Légende donnée à l'illustration dans le document latex
    Calendula - Paquerettes - 31/07/2024
}