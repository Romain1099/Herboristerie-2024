\label{hemorroides}
\ficherecette
{%titre recette
    Crème anti-hémorroïdes
}
{%liste d'ingrédients commençant par \item directement.
    \item $60$g de \voc{macérat huileux} de \voc{achillée millefeuille}
    \item $6$ gouttes de \voc{teinture mère} d'\voc{achillée millefeuille}
    \item $6 \text{ à }8$g de \voc{cire d'abeille}
    \item $6$g de \voc{miel}.
    \item $\approx 3$ pulvérisations d'\voc{eau florale de rose}.
}
{%liste du matériel commençant par \item directement.
    \item $1$ \frquote{\voc{cul de poule}} propre. 
    \item $1$ fouet. 
    \item $1$ pot \textbf{désinfecté}, \voc{étanche} et \voc{sec}.
    \item $1$ spatule. 
    \item $1$ casserole et de l'eau pour le bain marie. 
    \item Des plaques chauffantes. 
}
{%Métnodes et conseils de conservation
    \begin{itemize}[label=\faPen]
        \item Conservation longue $\approx 2$ ans.
        \item Dans un endroit sec, de préférence à l'abri de la lumière. 
    \end{itemize}
}
{%Métnodes et conseils d'utilisation    
    \begin{itemize}[label=\faPen]
        \item \voc{Diluer} dans de l'eau en respectant la \voc{posologie} ( voir \lien{teinture}{teintures mères} ).
        \item ?? En application locale sur la zone irritée.
    \end{itemize}
}
{%Remarques
    ?
}
{%chemin de l'illustration dans le dossier 'préparations'
    downloads/tm_achillee.jpg
}
{%Titre donné à l'illustration dans le document latex
   Teinture mère d'Achillée millefeuille
}
{%Légende donnée à l'illustration dans le document latex
    source : internet
}