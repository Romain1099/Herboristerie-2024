\newcommandx{\teinture}{\voc{teinture mère}}

\ficherecette
{%titre recette
    Baume pour la lignée féminine
}
{%liste d'ingrédients commençant par \item directement.
    \item $25$g de \voc{millepertuis}.
    \item $35$g d'\voc{achillée millefeuille}.
    \item $11$ gouttes de \teinture d'aubépine.
    \item $6$ pulvérisations d'\voc{eau de rose}.
    \item $8$g de \voc{cire d'abeille}.
}
{%liste du matériel commençant par \item directement.
    \item $1$ \frquote{\voc{cul de poule}} propre. 
    \item $1$ fouet. 
    \item $1$ pot \textbf{désinfecté}, \voc{étanche} et \voc{sec}.
    \item $1$ spatule. 
    \item $1$ casserole et de l'eau pour le bain marie. 
    \item Des plaques chauffantes.  
}
{%Métnodes et conseils de conservation
    \begin{itemize}[label=\faPen]
        \item Conservation courte $\approx 6$ mois.
        \item Dans un endroit sec, de préférence à l'abri de la lumière. 
    \end{itemize}
}
{%Métnodes et conseils d'utilisation    
    \begin{itemize}[label=\faPen]
        \item En application sur le \textbf{point de chakra} du \voc{plexus solaire}.
        \item Application possible sur le \textbf{premier point de chakra}.
        \item Utiliser un \textbf{ustensile propre} lors de l'utilisation pour prolonger la durée de conservation. 
    \end{itemize}
}
{%Remarques
    Le dosage de cire d'abeille correspond à $\dfrac{2}{10}$ de la masse de \macerat.\\
    Le dosage de miel correspond à $\dfrac{1}{10}$ de la masse de \macerat.\\
    La \textbf{texture} doit être un peu plus fluide que celle du \frquote{baume du tigre}.
}
{%chemin de l'illustration dans le dossier 'préparations'
    baume/baume_dessus_2.jpg    
}
{%Titre donné à l'illustration dans le document latex
    Baume pour la lignée féminine
}
{%Légende donnée à l'illustration dans le document latex
    01/08/2024
}