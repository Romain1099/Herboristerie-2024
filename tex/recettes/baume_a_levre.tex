\newcommand{\macerat}{\voc{macérat huileux}}
\newcommand{\macerats}{\voc{macérats huileux}}

\subsubsection{Baume à lèvre apaisant}
    \textbf{Ingrédients :}\\
    \begin{itemize}[label=\mysquare]
        \begin{multicols}{2}
            \item $50$g de \voc{macérat huileux} de \voc{calendula}.
            \item $10$g de \voc{macérat huileux} d'\voc{achillée}\\\textbf{millefeuille}.
            \item $18$g de \voc{cire d'abeille}
            \item $6$g de \voc{miel}.
            \item $3$ pulvérisations d'\voc{eau florale de rose}.
            \item $1$ goutte de \voc{propolis}.
        \end{multicols}
    \end{itemize}
    \textbf{Remarque :} \\
    Le dosage de cire d'abeille correspond à $\dfrac{3}{10}$ de la masse de \macerat.\\
    Le dosage de miel correspond à $\dfrac{1}{10}$ de la masse de \macerat.\\

    \textbf{Matériel :}\\
    \begin{itemize}[label=\mysquare]
        \begin{multicols}{2}
            \item $1$ \frquote{\voc{cul de poule}} propre. 
            \item $1$ fouet. 
            \item $1$ pot \textbf{désinfecté}, \voc{étanche} et \voc{sec}.
            \item $1$ spatule. 
            \item $1$ casserole et de l'eau pour le bain marie. 
            \item Des plaques chauffantes. 
        \end{multicols}
    \end{itemize}

    \begin{minipage}[t]{0.6\textwidth}
        \boite{Préparation :}{
            \begin{enumerate}%{label=\mysquare}
                \item Mélanger les \macerats \ dans un \voc{cul de poule}.
                \item Découper en petits morceaux la cire d'abeille les ajouter au mélgange. 
                \item Chauffer au \voc{bain-marie} en \textbf{remuant} jusqu'à \voc{dissolution} de la cire d'abeille. 
                \item Sortir le mélange du bain-marie et ajouter l'\textit{eau florale} et la \textit{propolis}.
                \item Continuer à remuer pendant le refroidissement jusqu'à obtention d'une \voc{pellicule} sur les rebords du récipient $\Longrightarrow$ \frquote{mayonnaise}. 
                \item \textbf{Verser} le mélange dans le pot de conservation. 
                \item Laisser aérer $24$h dans un endroit sec et propre, à l'abri de la lumière. 
                \item Refermer le pot. 
            \end{enumerate}
        }
    \end{minipage}
    \begin{minipage}[t]{0.35\textwidth}
        images
    \end{minipage}\\
    \textbf{Conservation :}\\
    \begin{itemize}[label=\faPen]
        \item Conservation courte $\approx 6$ mois.
        \item Dans un endroit sec, de préférence à l'abri de la lumière. 
    \end{itemize}
    \textbf{Utilisation :}\\
    \begin{itemize}[label=\faPen]
        \item En application locale sur la zone irritée.
        \item La préparation est commestible.
        \item Utiliser un ustensile propre lors de l'utilisation pour prolonger la durée de conservation. 
    \end{itemize}
