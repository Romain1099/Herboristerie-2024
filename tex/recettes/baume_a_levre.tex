\newcommand{\macerat}{\voc{macérat huileux}}
\newcommand{\macerats}{\voc{macérats huileux}}

\ficherecette
{%titre recette
    Baume à lèvres - anti-inflammatoire
}
{%liste d'ingrédients commençant par \item directement.
    \item $50$g de \voc{macérat huileux} de \voc{calendula}.
    \item $10$g de \voc{macérat huileux} d'\voc{achillée}\\\textbf{millefeuille}.
    \item $18$g de \voc{cire d'abeille}
    \item $6$g de \voc{miel}.
    \item $3$ pulvérisations d'\voc{eau florale de rose}.
    \item $1$ goutte de \voc{propolis}.
}
{%liste du matériel commençant par \item directement.
    \item $1$ \frquote{\voc{cul de poule}} propre. 
    \item $1$ fouet. 
    \item $1$ pot \textbf{désinfecté}, \voc{étanche} et \voc{sec}.
    \item $1$ spatule. 
    \item $1$ casserole et de l'eau pour le bain marie. 
    \item Des plaques chauffantes. 
}
{%Métnodes et conseils de conservation
    \begin{itemize}[label=\faPen]
        \item Conservation courte $\approx 6$ mois.
        \item Dans un endroit sec, de préférence à l'abri de la lumière. 
    \end{itemize}
}
{%Métnodes et conseils d'utilisation    
    \begin{itemize}[label=\faPen]
        \item En application locale sur la zone irritée.
        \item La préparation est commestible.
        \item Utiliser un \textbf{ustensile propre} lors de l'utilisation pour prolonger la durée de conservation. 
    \end{itemize}
}
{%Remarques
    Le dosage de cire d'abeille correspond à $\dfrac{3}{10}$ de la masse de \macerat.\\
    Le dosage de miel correspond à $\dfrac{1}{10}$ de la masse de \macerat.\\
    La \textbf{texture} doit correspondre à celle du \frquote{baume du tigre}.
}
{%chemin de l'illustration dans le dossier 'préparations'
    baume/baume.jpg
}
{%Titre donné à l'illustration dans le document latex
    Baume au Calendula
}
{%Légende donnée à l'illustration dans le document latex
    31/07/2024
}