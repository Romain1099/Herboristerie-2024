
\definecolor{headerbg}{RGB}{0, 128, 0} % Couleur du fond de l'en-tête
\definecolor{rowbg1}{RGB}{245, 245, 245}
\definecolor{rowbg2}{RGB}{255, 255, 255}

% Charger les données depuis le fichier CSV
%\DTLloaddb{liste148}{Ressources/liste_148.csv}

\definecolor{highlight}{RGB}{255, 255, 0} % Couleur de surlignage
\newcommand{\voc}[1]{\textbf{\color{red!65!black}#1}\index{#1}}

\newcolumntype{L}[1]{>{\raggedright\arraybackslash}p{#1}} % Définir une colonne alignée à gauche avec largeur fixe

% Commande pour créer un tableau pour une plante
\newcommand{\plante}[5]{
\begin{tabular}{|L{3cm}|L{4cm}|L{2cm}|L{3cm}|L{3cm}|}
\hline
\rowcolor{headerbg}
\textcolor{white}{\textbf{Nom français}} & \textcolor{white}{\textbf{Nom latin}} & \textcolor{white}{\textbf{Famille}} & \textcolor{white}{\textbf{Parties utilisées}} & \textcolor{white}{\textbf{Forme de préparation}} \\
\hline
\rowcolor{rowbg1}
#1 & #2 & #3 & #4 & #5 \\
\hline
\end{tabular}
}

\renewcommand\boite[2]{
\begin{tcolorbox}[nobeforeafter,title=\bcfleur #1,halign title=flush left,fonttitle=\bfseries,colbacktitle=headerbg,coltitle=white,colback=white]%red!50!black
#2
\end{tcolorbox}
}

% Définition de la commande \boiteidentiteplante
\NewDocumentEnvironment{boiteidentiteplante}{o+b}{
    \begin{tcolorbox}[
        enhanced,
	breakable,
        before skip=2mm,after skip=2mm,
        colback=red!5,colframe=red!50,boxrule=0pt,
        attach boxed title to top left={xshift=1cm,yshift*=1mm-\tcboxedtitleheight},
        varwidth boxed title*=-3cm,
        boxed title style={frame code={
            \path[fill=blue!50]
            ([yshift=-1mm,xshift=-1mm]frame.north west)
            arc[start angle=0,end angle=180,radius=1mm]
            ([yshift=-1mm,xshift=1mm]frame.north east)
            arc[start angle=180,end angle=0,radius=1mm];
            \path[left color=blue!50,right color=blue!50]
            ([xshift=-2mm]frame.north west) -- ([xshift=2mm]frame.north east)
            [rounded corners=1mm]-- ([xshift=1mm,yshift=-1mm]frame.north east)
            -- (frame.south east) -- (frame.south west)
            -- ([xshift=-1mm,yshift=-1mm]frame.north west)
            [sharp corners]-- cycle;
        },interior engine=empty,
        },
        fonttitle=\bfseries,
        title={\large{Fiche d'identité}},
        coltitle =white,
        drop shadow,
        borderline west={0.05mm}{0pt}{blue!50},
        borderline south={0.05mm}{0pt}{blue!50!black},
        overlay={
            \draw[line width=0.5mm, rem!50] 
            ([xshift=0mm,yshift=-0.25mm]frame.south west)--([xshift=0mm]frame.north west); % Bordure gauche
            \draw[line width=0.5mm, rem!50] 
            ([yshift=0mm]frame.south west)--([yshift=0mm]frame.south east); % Bordure du bas
            \ifx#1\empty
            \else
            \node[anchor=north east, fill=white, draw=rem!50, rounded corners] at ([xshift=-4cm]frame.north east) {\begin{minipage}{0.5\textwidth} \centering \textbf{#1} \end{minipage}};
            \fi%
        }
    ]
    #2
    \end{tcolorbox}
}{}

\NewDocumentEnvironment{boiterecette}{o+b}{
    \begin{tcolorbox}[
        enhanced,
	    breakable,
        before skip=2mm,after skip=2mm,
        colback=red!5,colframe=green!30!gray,boxrule=0pt,
        attach boxed title to top left={xshift=1cm,yshift*=1mm-\tcboxedtitleheight},
        varwidth boxed title*=-3cm,
        boxed title style={frame code={
            \path[fill=green!30!gray]
            ([yshift=-1mm,xshift=-1mm]frame.north west)
            arc[start angle=0,end angle=180,radius=1mm]
            ([yshift=-1mm,xshift=1mm]frame.north east)
            arc[start angle=180,end angle=0,radius=1mm];
            \path[left color=green!30!gray,right color=green!30!gray]
            ([xshift=-2mm]frame.north west) -- ([xshift=2mm]frame.north east)
            [rounded corners=1mm]-- ([xshift=1mm,yshift=-1mm]frame.north east)
            -- (frame.south east) -- (frame.south west)
            -- ([xshift=-1mm,yshift=-1mm]frame.north west)
            [sharp corners]-- cycle;
        },interior engine=empty,
        },
        fonttitle=\bfseries,
        title={\large{Recette}},
        coltitle =white,
        drop shadow,
        borderline west={0.05mm}{0pt}{red!40},
        borderline south={0.05mm}{0pt}{red!40!black},
        overlay={
            \draw[line width=0.5mm, rem!50] 
            ([xshift=0mm,yshift=-0.25mm]frame.south west)--([xshift=0mm]frame.north west); % Bordure gauche
            \draw[line width=0.5mm, rem!50] 
            ([yshift=0mm]frame.south west)--([yshift=0mm]frame.south east); % Bordure du bas
            \ifx#1\empty
            \else
            \node[anchor=north east, fill=white, draw=rem!50, rounded corners] at ([xshift=-4cm]frame.north east) {\begin{minipage}{0.5\textwidth} \centering \textbf{#1} \end{minipage}};
            \fi%
        }
    ]
    #2
    \end{tcolorbox}
}{}

% Définition de la commande \ficheidentiteplante
\newcommand{\ficheidentiteplante}[9]{
    \subsubsection{#1}

    \ifx\cita\empty
    \else
    \cita 
    \fi%


    \begin{boiteidentiteplante}[#1]

        \begin{multicols}{2}
            	\boite{Type d'effet :}{#2}
		
		        \columnbreak

            	\boite{Effets recherchés :}{#3}
        \end{multicols}

        \begin{multicols}{2}
            \begin{bclogo}[couleur=blue!30,arrondi=0.1,logo=\bccalendrier]{Cueillette}
                #4
            \end{bclogo}
            \begin{bclogo}[couleur=blue!30,arrondi=0.1,logo=\bcoutil]{Utilisation privilégiée}
                #5
            \end{bclogo}

	        \columnbreak

	        \begin{center}\includeplantenature[0.4]{#7}{#8}{#9}\end{center}
        \end{multicols}

        #6

    \end{boiteidentiteplante}
}
\newcommand{\cita}{}
\newcommand{\ficheidentiteplantelong}[9]{
    \subsubsection{#1}

    \ifx\cita\empty
    \else
    \cita 
    \fi%


    \begin{boiteidentiteplante}[#1]

        \begin{multicols}{2}
            	\boite{Type d'effet :}{#2}
		
		        \columnbreak

            	\boite{Effets recherchés :}{#3}
        \end{multicols}

        \begin{multicols}{2}
            \begin{bclogo}[couleur=blue!30,arrondi=0.1,logo=\bccalendrier]{Cueillette}
                #4
            \end{bclogo}

	        \columnbreak

	        \begin{center}\includeplantenature[0.4]{#7}{#8}{#9}\end{center}
        \end{multicols}
	 \begin{bclogo}[couleur=blue!30,arrondi=0.1,logo=\bcoutil]{Utilisation privilégiée}
                \begin{multicols}{2}
			#5	
		\end{multicols}
          \end{bclogo}
        #6

    \end{boiteidentiteplante}
}
\newcommand{\ficherecette}[9]{
    \subsubsection{#1}
    %citation si citation
    \begin{boiterecette}[#1]
        \begin{multicols}{2}
            \begin{bclogo}[couleur=green!30!white,arrondi=0.1,logo=\bcoeil]{Ingrédients}
                \begin{itemize}[label=\mysquare]
                    #2
                \end{itemize}
            \end{bclogo}
            \begin{bclogo}[couleur=green!40!white,arrondi=0.1,logo=\bcoutil]{Matériel}
                \begin{itemize}[label=\mysquare]
                    #3
                \end{itemize}
            \end{bclogo}
        \end{multicols}

        \boite{Préparation :}{
            \begin{multicols}{2}
                \begin{enumerate}
                    #4
                \end{enumerate}
            \end{multicols}
        }

        \begin{multicols}{2}

            \begin{bclogo}[couleur=green!40!gray,arrondi=0.1,logo=\bcbook]{Conservation / Utilisation}
                #5
            \end{bclogo}

	        \columnbreak

	        \begin{center}\includeprepa[0.4]{#7}{#8}{#9}\end{center}
        \end{multicols}
        \begin{Remarque}
            #6
        \end{Remarque}
    \end{boiterecette}
}

% Définition de la commande \lien
\newcommand{\lien}[2]{\hyperref[#1]{\bfseries\color{red!50!yellow}#2}}

\newcommand{\Potins}[2]{
    \begin{bclogo}[couleur=blue!10, arrondi=0.1, logo=\bcbook, marge=10]{Potins des plantes - #1}
        #2% Votre contenu ici
    \end{bclogo}
}

\newcommand{\subtil}[2]{
    \begin{bclogo}[couleur=blue!10, arrondi=0.1, logo=\bcyin, marge=10]{Sur le plan subtile - #1}
        \begin{itemize}[label=\faPen]
            #2% Votre contenu ici
        \end{itemize}
    \end{bclogo}
}