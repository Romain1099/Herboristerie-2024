
\subsection{Le métier d'herboriste}
Le métier d'herboriste n'est plus reconnu depuis 1941. Depuis, il y a un monopole des pharmaciens et des médecins sur le conseil des propriétés des plantes.
\begin{Remarque}
    Chez l'herboriste actuel(le), on trouve de l'ortie et de la camomille qui n'est pas français (le salaire des cueilleurs est trop élevé).
\end{Remarque}

\subsection{Les étapes de transformation d'une plante}

Il est important de connaître certaines informations avant d'utiliser une plante. \\
La \textbf{première question à se poser}, c'est \textit{que veut-on faire de la plante ?}, quels \textit{principes actifs} voulons-nous \textbf{extraire}.\\

La réponse à cette question permet de choisir l'un des \textbf{trois solvants} principalement utilisés pour effectuer des préparations.\\
\begin{itemize}[label=\faPen]
    \item L'eau
    \item L'alcool
    \item L'huile
\end{itemize}
Les utilisations de ces solvants seront décrites dans la section \textbf{Transformations} de ce document.\\
Dans les grandes lignes, chacun de ces solvant permet d'extraire exclusivement certaines molécules. En outre, chaque solvant a ses particularités de \textbf{préparation} et de \textbf{conservation}.
