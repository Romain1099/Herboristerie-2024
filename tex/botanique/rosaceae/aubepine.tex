\ficheidentiteplante
{Aubépine}
{%effet général
    L'\voc{aubépine}, en latin \textit{Crataegus}, est aussi appelée \voc{épine blanche} en raison de ses fleurs blanches.\\
    La signature de ses branches et de ses épines symbolise le \textbf{coeur qui saigne} et tout ce qui est lié à notre \voc{affectif}.

}
{%utilisation privilégiée
    C'est un arbre qui possède des propriétés \voc{apaisantes} et \textbf{régulatrices} de la \voc{tension}.\\

    Elle est recommandée pour lutter contre les \voc{angoisses} et les \voc{insomnies}.
}
{%infos cueillette
    On la trouve dans les haies, les bordures de forêts et de bois.\\
    C'est un arbre \textbf{épineux} mais peu volumineux. \\
    Ses fleurs sont \textbf{blanches} avec un \vocref{https://fr.wikipedia.org/wiki/Pistil}{pistil} rose.\\

    On récolte l'\voc{écorce}, les \voc{baies}, les \voc{feuilles} mais surtout les \voc{fleurs} en \textbf{mai - juin} au début de la floraison.
}
{%sous quelle forme utiliser
    
    \begin{itemize}[label = \bcplume]
        \item L'écorce
        \item Les fruits ( \voc{cénelles} )
        \item Les fleurs
    \end{itemize}
}
{%remarques
    \begin{multicols}{2}

        \boite{Usage interne :}{
            Une cure d'\textbf{infusion} d'aubépine : 
            \begin{itemize}[label = \bctrefle]
                \item On utilise principalement son fruit la \vocref{https://fr.wikipedia.org/wiki/Cenelle}{cénelle} pour ses \voc{antioxydants}.
                \item Angoisses, insomnies
                \item angine
                \item troubles nerveux
                \item mauvaise circulation du sang
            \end{itemize}

            Suivre la cure pendant 3 mois en prenant $1$ cuillère à café par tasse d'eau chaude, $3$ à $4$ fois par jour. \\

            Une \textbf{décoction} à l'aubépine a un effet \voc{anti-diarrhéique} et permet d'expulser les calculs des reins.
        }

        \columnbreak

        \boite{Usage externe :}{
            \begin{itemize}[label = \bccrayon]
                \item Permet de réhydrater les peaux sèches :\\
                        En \textbf{lotion} ou en \lien{decoction}{décoction}, éventuellement versée dans un bain.
            \end{itemize}
        }

    \end{multicols}    

}
{%image
    downloads/aubepine.jpg
}
{%titre photo
    Aubépine monogyne
}
{%description photo : "lieu - date"
    source : wikipedia
}
\begin{Remarque}
    La \textbf{fleur} de l'aubépine n'est pas autorisée à être utilisée en tisane.\\

    La fleur de l'aubépine est un \voc{cardiotonique}, mais qui soigne aussi le cœur émotionnel, elle dé-serre le cœur au niveau \voc{subtil}.
\end{Remarque}

\Potins{Aubépine}{
    Dans la tradition celtique
    Pour les Celtes, c’est un arbre sacré qui a le pouvoir d’éloigner la foudre. La principale de ses vertus est la protection. \\

    La foudre ne tombe que rarement sur un buisson d’aubépine et les oiseaux sont nombreux à y faire leur nid. 
    Les buissons d’aubépine leur assurent en effet, grâce à leurs épines, une protection efficace contre les prédateurs. 
    De par ces observations, les Celtes en ont fait un arbre protecteur dont on suspendait les rameaux aux berceaux des nouveau-nés, 
    ainsi qu’aux entourages des portes et des fenêtres, afin de se protéger des mauvais sorts et des maladies.\\
    On attendait de rentrer dans la partie lumineuse de l’année pour se marier, à partir du 1er mai. 
    L’aubépine, en fleurs au mois de mai, était symbole de chasteté et de pureté, mais également de bonheur, prospérité et 
    fidélité conjugale. C’était l’arbre des mariages. \\
    Dans de nombreuses régions, on tressait autrefois des couronnes d’aubépine en offrande aux fées et aux anges qui venaient 
    danser la nuit autour des buissons en fleurs.\\ 

    L’aubépine était l’arbre sacré de la fête de Beltaine (1er mai),
    elle avait le pouvoir de nous faire entrer dans le monde magique des esprits de la nature.\\

    Autrefois, le bois servait pour la fabrication de manches d’outils et de cannes.
    On plantait de l’aubépine aux abords des maisons car on pensait que sa proximité permettait de conserver la viande, 
    empêchait de faire tourner le lait et faisait fuir les serpents.\\
    Les fruits de l’aubépine sont comestibles en confitures ou compotes. Il est préférable d’attendre les premières gelées pour 
    les cueillir.\\

    C’est un régulateur du cœur et de la tension.
    On l’a surnommée \frquote{bonnet de nuit} en raison de ses propriétés à provoquer le
    sommeil.
}