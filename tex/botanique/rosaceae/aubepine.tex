\ficheidentiteplante
{Aubépine}
{%effet général
    L'\voc{aubépine}, en latin \textit{Crataegus}, est aussi appelée \voc{épine blanche} en raison de ses fleurs blanches.\\
    La signature de ses branches et de ses épines symbolise le \textbf{coeur qui saigne} et tout ce qui est lié à notre \voc{affectif}.

}
{%utilisation privilégiée
    C'est un arbre qui possède des propriétés \voc{apaisantes} et \textbf{régulatrices} de la \voc{tension}.\\

    Elle est recommandée pour lutter contre les \voc{angoisses} et les \voc{insomnies}.
}
{%infos cueillette
    On la trouve dans les haies, les bordures de forêts et de bois.\\
    C'est un arbre \textbf{épineux} mais peu volumineux. \\
    Ses fleurs sont \textbf{blanches} avec un \vocref{https://fr.wikipedia.org/wiki/Pistil}{pistil} rose.\\

    On récolte l'\voc{écorce}, les \voc{baies}, les \voc{feuilles} mais surtout les \voc{fleurs} en \textbf{mai - juin} au début de la floraison.
}
{%sous quelle forme utiliser
    
    \begin{itemize}[label = \bcplume]
        \item L'écorce
        \item Les fruits ( \voc{cénelles} )
        \item Les fleurs
    \end{itemize}
}
{%remarques
    \begin{multicols}{2}

        \boite{Usage interne :}{
            Une cure d'\textbf{infusion} d'aubépine : 
            \begin{itemize}[label = \bctrefle]
                \item On utilise principalement son fruit la \vocref{https://fr.wikipedia.org/wiki/Cenelle}{cénelle} pour ses \voc{antioxydants}.
                \item Angoisses, insomnies
                \item angine
                \item troubles nerveux
                \item mauvaise circulation du sang
            \end{itemize}

            Suivre la cure pendant 3 mois en prenant $1$ cuillère à café par tasse d'eau chaude, $3$ à $4$ fois par jour. \\

            Une \textbf{décoction} à l'aubépine a un effet \voc{anti-diarrhéique} et permet d'expulser les calculs des reins.
        }

        \columnbreak

        \boite{Usage externe :}{
            \begin{itemize}[label = \bccrayon]
                \item Permet de réhydrater les peaux sèches :\\
                        En \textbf{lotion} ou en \lien{decoction}{décoction}, éventuellement versée dans un bain.
            \end{itemize}
        }

    \end{multicols}    

}
{%image
    downloads/aubepine.jpg
}
{%titre photo
    Aubépine monogyne
}
{%description photo : "lieu - date"
    source : wikipedia
}
\begin{Remarque}
    La \textbf{fleur} de l'aubépine n'est pas autorisée à être utilisée en tisane.\\

    La fleur de l'aubépine est un \voc{cardiotonique}, mais qui soigne aussi le cœur émotionnel, elle dé-serre le cœur au niveau \voc{subtil}.
\end{Remarque}