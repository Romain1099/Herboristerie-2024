\label{reinepres}
\ficheidentiteplante
{Reine des prés}
{%effet général
    La \vocref{https://fr.wikipedia.org/wiki/Filipendula_ulmaria}{reine des prés}, en latin \textit{Filipendula ulmaria} est également nommée
    \textbf{spirée ulmaire} car ses fruits sont tordus en forme de spirale.\\


    Elle est utilisée dans la pharmacopée en \textit{infusion} contre les \textbf{reflux} ou les \textbf{aigreurs d'estomac}.\\

    Ses \textbf{feuilles} sont utilisées pour teindre la laine en \textbf{jaune}.\\

    Elle était utilisée comme alternative au tabac.
}
{%utilisation privilégiée
    On cherche à extraire l'\voc{acide salicylique} ( aspirine ) de cette plante. \\

    La reine des prés a un effet \voc{anti-inflammatoire} et \voc{diurétique}.\\

    Son utilisation est susceptible de faire \voc{baisser la fièvre}.
}
{%infos cueillette

    Toutes les parties de la plante peuvent être utilisées.\\
    On récolte les \voc{sommités fleuries} de \textbf{juin à aout} avant leur complet épanouissement. \\

    La reine-des-prés pousse dans les lieux humides, en bordure de ruisseaux ou de rivières.\\
    Comme l'indique son nom, c'est une très grande plante : on la voit de loin !
}
{%sous quelle forme utiliser
    \begin{itemize}[label = \bcplume]
        \item Principalement en infusion :\\
                Réaliser une \textbf{cure} de 3 à 5 tasses par jour. \\
                Dosage : $20$ à $30$g de plantes par litre d'eau.
    \end{itemize}
}
{%remarques
    \begin{multicols}{2}

        \boite{Usage interne :}{
            Une \textbf{infusion} de reine des prés est un remède pour les affections suivantes : 
            \begin{itemize}[label = \bctrefle]
                \item Goutte
                \item Rétention de liquides
                \item Arthrites et douleurs articulaires
                \item Cellulites
                \item Certaines affections des voies urinaires
                \item Obésité graisseuse
            \end{itemize}
        }

        \columnbreak

        \boite{Usage externe :}{
            \begin{itemize}[label = \bccrayon]
                \item Application directe de la feuille sur une plaie pour en accélérer la \voc{cicatrisation}.
            \end{itemize}
        }

    \end{multicols}
}
{%image
    downloads/reine_des_pres.jpg
}
{%titre photo
    Reine des prés en fleurs
}
{%description photo : "lieu - date"
    source : wikipedia
}

\begin{Remarque}
    La reine des prés est \textbf{contre-indiquée} pour les personnes qui ne \textbf{supportent pas l'aspirine}.\\

    \textbf{Associations :}\\
    Pour accentuer les effets de la reine-des-prés, on pourra l'associer à la \voc{menthe} au \voc{cassis}, au \voc{frêne} et au \voc{romarin}.
\end{Remarque}