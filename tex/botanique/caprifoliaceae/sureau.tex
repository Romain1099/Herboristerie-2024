\subsubsection{Sureau}
\Potins{Sureau}{
    La mère sureau ou grand-mère sureau ou fée du sureau serait une présence féminine, gardienne du sureau et de la mère Terre. 
    Cette mère sureau serait peut-être la femme du dieu Pan, dieu de la nature, de la forêt et des animaux.\\

    \textbf{Dans la tradition celtique :}
    Le sureau est le treizième arbre du calendrier des arbres celtiques, placé en fin d’année. \\
    Il est l’un des derniers à perdre ses feuilles à l’automne, et le premier à les sortir au printemps. \\
    Il représente le lien entre la mort et la vie. On a retrouvé beaucoup de graines lors de fouilles archéologiques. \\
    Le sureau était déjà utilisé à une époque très lointaine pour le culte des morts.\\

    Le sureau a la capacité de se régénérer très facilement. Il suffit de planter une baguette de sureau dans la terre et il y a 
    de fortes chances qu’elle se réimplante et redevienne un arbuste l’année suivante. \\
    Pour les Celtes, il représente la vitalité, la vie éternelle, la mort et la renaissance. \\
    Il facilite le passage entre les mondes.\\

    Les Celtes s’en servaient pour confectionner des flûtes magiques pour faciliter les conversations avec les morts.\\
    Le sureau était considéré comme un arbre sacré dont le bois creux abritait des divinités, des esprits de la forêt, des fées. \\
    Si jamais l’on osait couper cet arbre malgré l’interdiction, cela portait malheur.\\ 
    Il est le lien avec les esprits de la nature.\\

    Le bois creux du sureau permettait de fabriquer des baguettes magiques, dans lesquelles on pouvait glisser des objets aux 
    pouvoirs guérisseurs.\\
    C’est une plante de protection que l’on porte sur soi ou que l’on suspend au-dessus des portes et des fenêtres. 
    On dit que le sureau n’est jamais frappé par la foudre.\\

    Mettre des feuilles fraîches de sureau dans les terriers permet d’en éloigner les rongeurs. 
    Pour éloigner les mouches, on suspendra dans la maison des feuilles fraîches de sureau.\\

    Des fleurs de sureau séchées, étalées dans le fond d’un cageot de pommes, permettront de conserver ces fruits plus longtemps 
    et leur donneront en plus un léger goût d’ananas.\\

    Le sureau influence le sommeil et les rêves.
    Il est possible d’utiliser les baies de sureau pour se teindre les cheveux. \\
    Cela permet en plus de les nourrir et de leur redonner de la force.
}