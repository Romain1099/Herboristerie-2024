\ficheidentiteplante
{Astragalus membranaceus}
%{%effet général
  %
% }
{%utilisation privilégiée
   L'Astragalus membranaceus est principalement utilisé pour renforcer le \voc{système immunitaire}, améliorer la \voc{fonction cardiaque} et \voc{réduire le stress}.\\
   Elle est connue pour ses propriétés \voc{immunostimulantes}, \voc{antioxydantes} et \voc{anti-inflammatoires}. 
 }
{%infos cueillette
    \begin{itemize}[label = \bcplume]
		\item \textbf{Parties utilisées :} Les racines de la plante sont principalement utilisées.
		\item \textbf{Période de cueillette :} Les racines sont généralement récoltées à l'automne.
		\item \textbf{Lieu de cueillette :} L'Astragalus membranaceus pousse principalement en Chine et en Mongolie.
	\end{itemize}
    }

{%sous quelle forme utiliser
   \begin{itemize}
		\item \textbf{Infusion :} Faire bouillir les racines dans de l'eau pendant 20-30 minutes.
		\item \textbf{Poudre :} Les racines peuvent être réduites en poudre et ajoutées à des smoothies ou des boissons.
		\item \textbf{Extrait :} Les extraits d'Astragalus membranaceus sont disponibles sous forme de teintures ou de capsules.
		\item \textbf{Complément alimentaire :} Disponible sous forme de capsules ou de comprimés.
\end{itemize}

}
{%supplément
\begin{multicols}{2}

    \boite{Usage interne :}{
        \begin{itemize}[label = \bctrefle]
            \item Augmentation de l'énergie
            \item Amélioration de la performance physique et mentale
            \item Renforcement du système immunitaire
        \end{itemize}
    }

    \columnbreak

    \boite{Usage externe}{
        \begin{itemize}[label = \bccrayon]
            \item Pas d'usage externe connu
        \end{itemize}
    }

\end{multicols}
}
{%image
    astragalus-membranaceus.jpg
}
{%titre photo
    Astragalus membranaceus
}
{%description photo : "lieu - date"
    Site \frquote{\vocref{https://beneficial-herbs.blogspot.com/2023/07/astragalus-astragalus-membranaceus.html}{en anglais sur les plantes médicinales}} - 12/08/2024 

}


\begin{Remarque}
    L'Astragalus membranaceus peut interagir avec certains médicaments, notamment les \voc{immunosuppresseurs}. 
\end{Remarque}