\ficheidentiteplante
{Reishi}
{%effet général
    Le \textbf{Reishi} (\textit{Ganoderma lucidum}), aussi appelé "plante de l'immortalité", est 
    reconnu pour ses propriétés puissantes de soutien du \voc{système immunitaire}, 
    de destruction des cellules cancéreuses, et d'apport d'énergie et de solidité. \\
    En Asie, c'était la plante de l'empereur en raison de ses vertus associées à l'immortalité et 
    au ralentissement du vieillissement.\\
    C'est aussi une plante de l'intuition et de la sagesse, agissant sur le \voc{mental}, la \voc{mémoire}, et 
    l'évolution personnelle.
}
{%utilisation privilégiée
    Le Reishi est particulièrement utilisé pour soutenir l'immunité, en augmentant les cellules 
    qui combattent les infections ( action \voc{immunomodulatrice} ). \\
    Il est aussi privilégié pour réduire les effets secondaires de la \voc{chimiothérapie}. \\
    Au Québec, les patients arrêtent sa consommation deux jours avant la chimiothérapie, puis 
    reprennent après le traitement pour minimiser les effets secondaires. \\
    Le Reishi est également bénéfique pour les \voc{infections virales} longues, comme la \voc{mononucléose}, 
    et pour les personnes ayant un système immunitaire trop faible ou trop actif.
}
{%infos cueillette
    \begin{itemize}[label = \bcplume]
        \item Le Reishi est principalement récolté pour ses parties ligneuses, le plus souvent sous 
                forme de champignon séché.
        \item La cueillette se fait généralement à maturité, lorsque le champignon est bien 
                développé et riche en polysaccharides.
    \end{itemize}
}
{%sous quelle forme utiliser
    \begin{itemize}
        \item \textbf{Teinture mère :} Préparée à partir du Reishi séché, elle est utilisée pour 
                ses effets immunomodulateurs.
        \item \textbf{Décoction :} Utilisée pour extraire les composés actifs du Reishi, notamment 
                les polysaccharides, afin de renforcer le système immunitaire et protéger le foie.
    \end{itemize}
}
{%supplément
\begin{multicols}{2}

    \boite{Usage interne :}{
        \begin{itemize}[label = \bctrefle]
            \item Le Reishi est pris en interne sous forme de teinture mère ou de décoction pour 
                détruire les cellules cancéreuses, soutenir l'immunité, et réduire les effets 
                    secondaires de la chimiothérapie. \\
                    Il est également utilisé pour renforcer l'énergie vitale et la solidité du corps.
        \end{itemize}
    }

    \columnbreak

    \boite{Usage externe}{
        \begin{itemize}[label = \bccrayon]
            \item Le Reishi est rarement utilisé en usage externe, mais ses extraits peuvent être 
                    appliqués pour protéger la peau ou pour des préparations destinées à la guérison 
                    spirituelle.
        \end{itemize}
    }

\end{multicols}
}
{%image
    downloads/reishi.jpg
}
{%titre photo
    Reishi
}
{%description photo : "lieu - date"
    Internet - 12/08/2024 
}

\begin{Remarque}
    Le Reishi, bien que très bénéfique, doit être utilisé avec prudence. Chez certaines personnes, en particulier celles ayant des traumas non résolus, sa consommation peut faire remonter des émotions ou souvenirs enfouis. Il est important d'être conscient de cet effet potentiel avant de l'intégrer dans un régime thérapeutique.

\end{Remarque}

\subtil{Reishi}{
    \item utilisée lorsque des événements du passé ne sont pas digérés
    \item  ouvre l’inconscient
    \item  permet une connexion avec l’au-delà
    \item  plante de l’intuition et de la sagesse
}