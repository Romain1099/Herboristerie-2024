\ficheidentiteplante
{Schisandra de chine - Shisandra sinensis}
{%effet général
   }
{%utilisation privilégiée
    Le Schisandra chinensis est connu pour ses propriétés adaptogènes, qui aident le corps à s'adapter au stress. Il est souvent utilisé pour améliorer la performance physique et mentale, renforcer le système immunitaire et améliorer la fonction hépatique.
	Le Schisandra chinensis est principalement utilisé pour améliorer la performance physique et mentale, renforcer le système immunitaire et améliorer la fonction hépatique.
 }
{%infos cueillette
    \begin{itemize}[label = \bcplume]
		\item \textbf{Parties utilisées :} Les baies de la plante sont principalement utilisées.
		\item \textbf{Période de cueillette :} Les baies sont généralement récoltées à l'automne.
		\item \textbf{Lieu de cueillette :} Le Schisandra chinensis pousse principalement en Chine et en Russie.
	\end{itemize}
    }

{%sous quelle forme utiliser
	\begin{itemize}
		\item \textbf{Infusion :} Faire bouillir les baies dans de l'eau pendant 20-30 minutes.
		\item \textbf{Poudre :} Les baies peuvent être réduites en poudre et ajoutées à des smoothies ou des boissons.
		\item \textbf{Extrait :} Les extraits de Schisandra chinensis sont disponibles sous forme de teintures ou de capsules.
		\item \textbf{Complément alimentaire :} Disponible sous forme de capsules ou de comprimés.
	\end{itemize}
}

{%supplément
\begin{multicols}{2}

    \boite{Usage interne :}{
        \begin{itemize}[label = \bctrefle]
            \item Amélioration de la performance physique et mentale
			\item Renforcement du système immunitaire
			\item Amélioration de la fonction hépatique
        \end{itemize}
    }

    \columnbreak

    \boite{Usage externe}{
        \begin{itemize}[label = \bccrayon]
            \item Pas d'usage externe connu
        \end{itemize}
    }

\end{multicols}
}

{%remarques
  Le Schisandra chinensis peut interagir avec certains médicaments, notamment les anticoagulants. Il est recommandé de consulter un professionnel de santé avant de commencer à utiliser le Schisandra chinensis.
 }

{%image
    schisandra-de-chine.jpg
}

{%titre photo
    Schisandra de chine
}

{%description photo : "lieu - date"
   Site de \vocref{https://laforetcomestible.org/wp-content/uploads/2020/04/schisandra-chinensis-wuweizi.jpg}{\frquote{site de la forêt comestible} - 12/08/2024 
}
}