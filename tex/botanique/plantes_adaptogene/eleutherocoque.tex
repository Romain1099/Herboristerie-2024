\ficheidentiteplante
{Éleuthérocoque}
{%effet général
  L'Éleuthérocoque est connu pour ses propriétés \voc{adaptogènes}. Il est souvent utilisé pour \voc{augmenter l'endurance}, \voc{améliorer} la \voc{performance physique} et \voc{mentale}, et renforcer le \voc{système immunitaire}.
}
{%utilisation privilégiée
   L'Éleuthérocoque est principalement utilisé pour augmenter l'endurance, améliorer la performance physique et mentale, et renforcer le système immunitaire.
}
{%infos cueillette
   \begin{itemize}[label = \bcplume]
        \item \textbf{Parties utilisées :} Les \voc{racines} de la plante sont principalement utilisées.
        \item \textbf{Période de cueillette :} Les racines sont généralement récoltées à l'\voc{automne}.
        \item \textbf{Lieu de cueillette :} L'Éleuthérocoque pousse principalement en \voc{Sibérie} et en \voc{Chine}.
    \end{itemize}
}
{%sous quelle forme utiliser
    \begin{itemize}
        \item 
    \end{itemize}
}
{%supplément
    \begin{multicols}{2}

        \boite{Usage interne :}{
            \begin{itemize}[label = \bctrefle]
                \item \textbf{Infusion :} Faire bouillir les racines dans de l'eau pendant 20-30 minutes.
                \item \textbf{Poudre :} Les racines peuvent être réduites en poudre et ajoutées à des smoothies ou des boissons.
                \item \textbf{Extrait :} Les extraits d'Éleuthérocoque sont disponibles sous forme de teintures ou de capsules.
                \item \textbf{Complément alimentaire :} Disponible sous forme de capsules ou de comprimés.
            \end{itemize}
        }

        \columnbreak

        \boite{Usage externe}{
            \begin{itemize}[label = \bccrayon]
                \item Pas d'usage externe connu
            \end{itemize}
        }

    \end{multicols}
}
{%image
    eleutherocoque.jpg
}
{%titre photo
    Éleuthérocoque - Eleutherococcus senticosus
}
{%description photo : "lieu - date"
    Site de \vocref{https://nantes-naturopathe.fr/solution/eleutherocoque/}{Nantes-naturopathe} - 12/08/2024 
}

\begin{Remarque}
    L'Éleuthérocoque peut \voc{interagir} avec certains médicaments, notamment les \voc{anticoagulants}. 

\end{Remarque}