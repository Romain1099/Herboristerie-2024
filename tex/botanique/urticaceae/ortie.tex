\label{ortie}
\renewcommand{\cita}{
    \phantom{a}\citer{La vérité était que la vie nous avait jetés aux orties, l'un et l'autre\ldots\\ et c'est toujours ce qu'on appelle une rencontre.}{\href{https://fr.wikipedia.org/wiki/Romain_Gary}{Romain Gary}}

}
\ficheidentiteplante
{Ortie}
{%effet général
    L'\voc{ortie}, en latin \textit{Urtica dioica}, est une des plantes médicinales les plus \textbf{efficaces} et les plus \textbf{communes} que nous cotoyons.

    De manière générale, l'ortie a des effets reminéralisants, régénère le sang, stimule les \voc{fonctions digestives} par exemple en diminuant la \voc{glycémie}.\\ 
    Elle est \voc{fortifiante}, améliore la \voc{concentration} et aide à réduire l'\voc{anxiété}.\\
}
{%utilisation privilégiée
    L'ortie serait une \voc{plante adaptogène}. \\ Elle aidera le corps à réagir à de nombreuses situations de \voc{stress}.\\
    On l'utilise pour \voc{nettoyer}, \voc{récurrer}, \voc{lustrer} et \voc{patiner} grâce à sa teneur en \voc{silice}.\\

    Elle stimule la production de \voc{globules rouges} et aide à \voc{éliminer les bactéries} et en cas de maladie \voc{virale}.\\


    On peut utiliser la racine ou la plante entière pour obtenir des \textbf{teintes jaunes}.
}
{%infos cueillette
    \begin{itemize}[label=\faPen]
        \item On préférera utiliser des \textbf{jeunes pousses} car en vieillissant, l'ortie se charge en \voc{minéraux} pouvant provoquer des \voc{blocages rénaux}.
        \item On utilise la \textbf{plante entière}
        \item Cueillir au \textbf{printemps}
    \end{itemize}
}
{%sous quelle forme utiliser
    \begin{itemize}[label = \bcplume]
        \item En cure au \textbf{printemps} $\longrightarrow$ $2$L par jour
        \item En décoction pour un usage externe
        \item En \voc{teinture mère} pour un usage aussi bien interne que externe.
    \end{itemize}
}
{%remarques
\begin{multicols}{2}

    \boite{Usage interne :}{
        Doser $20$g de plante par litre d'eau et consommer jusqu'à $1$L par jour de cette tisane. \\
        On peut l'utiliser en cas de :
        \begin{itemize}[label = \bctrefle]
            \item \voc{Diabète}
            \item \voc{Anémie}
            \item \voc{Hépatite}
            \item \voc{Hémorragie utérine}
            \item \voc{Goutte}
            \item \voc{Rhumatismes}, arthrose
            \item \voc{Diarrhées}
            \item Troubles du foie et de la rate
            \item \voc{Ulcères} d'estomac ou intestinaux
        \end{itemize}
        \textbf{En teinture mère : }
                \begin{itemize}[label = \faPen]
                    \item Allergies : 20 gouttes $3$ fois par jour pendant 3 semaines.
                \end{itemize}
    }

    \columnbreak

    \boite{Usage externe :}{
        \begin{itemize}[label = \bcoeil]
            \item En décoction :
                \begin{itemize}[label = \faPen]
                    \item Angine ( gargarismes )
                    \item Affection de la peau ( eczéma, acnée, herpès )
                    \item nettoyage du cuir chevelu
                \end{itemize}
            \item En teinture mère : 
                \begin{itemize}[label = \faPen]
                    \item Appliquer des frictions ou cataplasmes sur les \voc{rhumatismes}, les \voc{entorses}, les \voc{sciatiques}
                    \item Nettoyage du cuir chevelu
                    \item En gargarismes dilué dans de l'eau pour les gingivites, infections, aphtes, angine.
                \end{itemize}
        \end{itemize}
    }

\end{multicols}
\boite{Au jardin :}{
        \begin{multicols}{2}
            Le \voc{purin} d'orties est un \voc{engrais} efficace
            \begin{itemize}[label = \faPen]
                \item Riche en \voc{azote}
                \item en éléments organiques
                \item en \voc{minéraux}
                \item en \voc{oligoéléments}
            \end{itemize}
            Il est \textbf{préventif} contre le \voc{mildiou} ou la \voc{rouille} ou l'\voc{oïdium}.\\
            On peut également l'utiliser comme \voc{insecticide}
            Voir \cite{laporte2023} page 103 pour plus de détails.
        \end{multicols}
    }

}
{%image
    ortie.jpg
}
{%titre photo
    Ortie
}
{%description photo : "lieu - date"
    Logis médiéval de Tessé - 29/07/2024 
}
\begin{Remarque}

    \bcattention Attention à la consommation des \textbf{graines} qui peuvent être toxiques au delà d'une certaine quantité.\\

    \bcattention La plante ne doit pas être consommée en cas d'oedème par rétention due à une insuffisance cardiaque ou rénale.\\

    \bcattention L'ortie peut \textbf{influencer} des traitements en cours aussi bien en les accélérant qu'en les ralentissant.\\
    Il est nécessaire de bien se renseigner lorsqu'on l'utilise en parallèle d'autres traitements.

    \bcattention \'Eviter l'ortie crue pendant la \textbf{grossesse}.
\end{Remarque}
\renewcommand{\cita}{}

\Potins{Ortie}{
    Dans la tradition celtique
    Plante de feu, l’ortie symbolise la force, l’énergie et le courage. Malgré son aspect rugueux et irritant, c’est une plante 
    d’une grande générosité, offrant la nourriture aux hommes et aux animaux, des vêtements, des remèdes.\\
    Elle est vénérée par les peuples celtes, symbolisant la foi et la persévérance qui nous sont nécessaires pour aller au- delà des 
    apparences, nous donnant alors accès à des trésors cachés.\\
    L’ortie s’est progressivement installée dans les clairières, permettant grâce à sa fibre extrêmement solide, de fabriquer des 
    cordes, des nasses et des tissus.\\

    Depuis très longtemps, la fibre d’ortie est utilisée pour faire du tissu (proche du chanvre). Les momies égyptiennes étaient entourées de bandelettes tissées de fibre d’ortie, et c’est sans doute grâce aux qualités de cette matière que les momies se sont conservées si longtemps. On y reviendra peut-être mais, actuellement, le travail de la fibre d’ortie étant difficile à mécaniser, personne
    n’y a plus recours.\\

    L’ortie serait une plante adaptogène. Autrement dit, elle aide le corps à s’adapter aux différents stress environnementaux et 
    psychologiques. Elle est reminéralisante.\\
    On l’utilise pour nettoyer, récurer, lustrer et patiner grâce à sa teneur en silice.\\
    On utilise la racine ou la plante entière pour obtenir des teintes jaunes. \\

    L’ortie pousse autour des maisons dans les endroits où l’homme a vécu. C’est un signe de bonne santé du sol, elle régularise le 
    fer et l’azote. Elle semble transformer les ondes négatives et assainir les terres.\\
    Elle aide à protéger les tomates et les pommes de terre contre le mildiou.\\

    Le voisinage de l’ortie augmente la teneur en huile essentielle des plantes médicinales. 
    Le suc de ces plantes voisines s’altère moins vite.\\
    Elle favorise la transformation des déchets en humus lorsqu’elle est ajoutée au compost (avant la floraison). 
    En couverture, elle protège et nourrit le sol.
}