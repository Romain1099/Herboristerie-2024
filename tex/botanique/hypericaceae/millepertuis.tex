\ficheidentiteplante
{Millepertuis}
{%effet général
    Le \voc{millepertuis} en latin \textit{hypericum perforatum} est aussi appelé le \voc{soleil intérieur}.\\

    C'est une plante qui a un très fort pouvoir de \voc{guérison} et de \voc{protection}.\\
    Elle est \textbf{riche en mélatonine} et a une action \textbf{régulatrice du sommeil}.
}
{%utilisation privilégiée
    \begin{itemize}[label=\bcoutil]
        \item Contre les dépressions
        \item Contre les \textbf{maux d'hiver} : \\
        \item Dépressions 
        \item Stress
        \item Système nerveux fragilisé
    \end{itemize}
}
{%infos cueillette
    \begin{itemize}[label=\faPen]
        \item Penser à se protéger du soleil.
        \item Vérifier qu'il s'agit d'\textit{hypericum perforatum} en \\
                \begin{itemize}[label = \bcoeil]
                    \item Observant des \textbf{petits trous} sous ses feuilles.
                    \item \textbf{\'Ecrasant la fleur} pour voir s'échapper un \textbf{liquide rouge}. 
                \end{itemize}
    \end{itemize}
}
{%sous quelle forme utiliser
    \begin{itemize}[label = \bcplume]
        \item On utilise principalement les \voc{sommités fleuries}.
        \item On peut l'utiliser en fumigation $\longrightarrow$ vertus protectrices et purificatrices
        \item Peut être utilisé comme plante \voc{teinctoriale} pour colorer en \textbf{rouge}.
    \end{itemize}
}
{%remarques
    
    \begin{minipage}[t]{0.35\textwidth}
        \boite{Utilisation interne :}{
            On peut faire une \lien{infusion}{infusion} ou une \lien{decoction}{decoction} en utilisant $15$ à $20$ grammes de plantes séchées \textbf{par litre d'eau}.\\
            $2$ à $4$ tasses par jour.\\

            \begin{itemize}[label=\faPen]
                \item Affections pulmonaire chronique
                \item Troubles hépatiques et circulatoires
                \item Asthme
                \item Cystite
                \item Herpès
            \end{itemize}
        }
    \end{minipage}
    \hfill
    \begin{minipage}[t]{0.65\textwidth}
        \boite{Utilisation externe :}{
            \setlength{\columnseprule}{1pt}
            \begin{multicols}{2}
                \textbf{En macérat huileux :}
                \begin{itemize}[label=\faPen]
                    \item Douleurs rhumatismales
                    \item Sciatiques
                    \item Tendinites, torticoli, luxation
                    \item Ulcères
                    \item Goutte
                    \item Plaies
                    \item Brûlures
                    \item Hématomes
                    \item Coupures
                    \item Contusions
                \end{itemize}
                
                \columnbreak


                \textbf{En pommade : }
                \begin{itemize}[label=\faPen]
                    \item Douleurs nerveuses
                    \item Engorgement mammaire
                \end{itemize}
                \vspace{0.5cm}
                \textbf{En infusion :} comme \voc{lotion} pour les peaux grasses et \voc{acnéeiques}.
            \end{multicols}
        }
    \end{minipage}
}
{%image
    millepertuis (1).jpeg
}
{%titre photo
    Millepertuis
}
{%description photo : "lieu - date"
    Jardin de  - 30/07/2024 
}

\begin{Remarque}
    \begin{itemize}[label=\faPen]
        \item Le millepertuis induit un effet \voc{photosensible} pouvant occasionner des \textbf{brulures}.
    \end{itemize}
\end{Remarque}