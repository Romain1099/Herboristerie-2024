
\ficheidentiteplante
{Armoise annuelle}
{%effet général
    L'\vocref{https://fr.wikipedia.org/wiki/Armoise_annuelle}{Armoise annuelle} .
}
{%utilisation privilégiée
    \begin{itemize}[label = \bcplume]
        \item \textbf{Antipaludique}, 
        \item \textbf{stimulant digestif}
    \end{itemize}
}
{%infos cueillette
    \begin{itemize}[label = \bcplume]
        \item \textbf{Feuilles}, et \textbf{tiges}.
        \item fin d'été.
    \end{itemize}
}
{%sous quelle forme utiliser
    \begin{itemize}[label = \bccrayon]
        \item \voc{Purin} : conseillé en cas de piéride du chou. Efficace en pulvérisation contre les limaces.\\ Mettez à macérer 1kg de plante fraîche dans 10L d'eau pendant 24 heures.
        \item \voc{Tisanes} : 15 à 30g par litre d'eau, 3 à 5 tasses par jour.
        \item \voc{Teinture mère} : 3 à 5 gouttes par jour.
        \item \voc{Elixir floraux} : pour aider les hypersensibles à s'enraciner. Cela apaise harmonise et rééquilibre la sensibilité.
    \end{itemize}
}
{%supplément
    \begin{multicols}{2}

        \boite{Usage interne :}{
            \begin{itemize}[label = \bctrefle]
                \item l'armoise est surtout connue pour être \voc{emménagogue} pour régulariser ou provoquer les règles, contre les douleurs menstuelles et lors de la ménopause.
                \item Elle est également efficace pendant l'accouchement surtout en cas de rétention de placenta.
                \item On l'utilise aussi pour les digestion lente, l'inappétence, les crampes d'estomac, les parasites intestinaux, l'anémie, l'épilepsie, les troubles nerveux.
            \end{itemize}
        }

        \columnbreak

        \boite{Usage externe}{
            \begin{itemize}[label = \bccrayon]
                \item Les bains d'armoises sont conseillés contre la goutte et les rhumatismes.
                \item les cendres de la plante arrêtent les saignements de nez.
                \item Elle est souvent utilisée comme une aternative au tabac
            \end{itemize}
        }

    \end{multicols}

}
{%image
    armoise_vulgaris.jpg
}
{%titre photo
    Armoise annuelle
}
{%description photo : "lieu - date"
    Excursion - Tessé - 02/08/2024\\Dessous des feuilles \frquote{argenté}
}

\begin{Remarque}
    \begin{itemize}[label = \bcplume]
        \item Cette plante est fortement déconseillée en cas de grossesse et d'allaitement.
        \item Elle est toxique en cas d'emploi prolongé
        \item Ne pas confondre avec l'ambroisie qui n'a pas le dessous des feuilles argentée
    \end{itemize}
\end{Remarque}

\Potins{Armoise}{
    Elle nettoie et travaille sur le sang, les ancêtres, la divination comme la sauge 
    Plante d'Armoise commune ( plante à rêve )
    Dans certaines cultures traditionnelles, l’armoise est considérée comme une herbe chamanique. \\
    Elle est souvent utilisée pour ses propriétés psychoactives et est considérée comme ayant des propriétés spirituelles. \\
    Dans les cultures indigènes d'Amérique du Nord, l'armoise était utilisée pour les cérémonies.\\
    Elle était brûlée pour produire de la fumée qui était utilisée pour nettoyer les espaces et les personnes, 
    et pour aider les chamans à entrer dans des états modifiés de conscience pour communiquer avec les esprits et les 
    ancêtres. 
}