
\ficheidentiteplante
{Plantain lancéolé}
{%effet général
    L'\vocref{https://fr.wikipedia.org/wiki/Plantago_lanceolata}{Plantain lancéolé}, en latin \textit{lantago lanceolata}.
}
{%utilisation privilégiée
    \begin{itemize}[label = \bcplume]
        \item En premier lieu contre les maladies des organes respiratoires.
        \item Effet astringent
        \item Effet cicatrisant
        \item S'utilise contre les inflammation et les \voc{hémorroïdes}
    \end{itemize}
}
{%infos cueillette
    \begin{itemize}[label = \bcplume]
        \item La plante entière est à utiliser ( y compris les racines ).
        \item La floraison a lieux d'\textbf{avril} à \textbf{octobre} sur le littoral méditerranéen.
    \end{itemize}
}
{%sous quelle forme utiliser
    \begin{itemize}[label = \bccrayon]
        \item En \lien{infusion}{infusion} : $1$ c.a.c de feuilles avec $0{,}25$cL d'eau.
        \item En \voc{cataplasme} de feuilles broyées.
        \item En \voc{sirop}.
        \item Contre les piqûres (moustiques, guêpes, orties…) et les démangeaisons.\\
                Frotter une ou plusieurs feuilles sur l’endroit de la piqûre jusqu’à en extraire le suc.
        \item Les feuilles fraîches, riches en mucilages, peuvent être utilisées en cataplasme pour arrêter les \voc{saignements} ou soigner les \voc{ampoules}.
    \end{itemize}
}
{%supplément
        \boite{Usage interne :}{
            \begin{itemize}[label = \bctrefle]
                \item Toute la plante est \voc{comestible}.
                \item Les feuilles tendres ont un goût de \textbf{champignon} se mangent \textbf{crues} en salade.\\
                \item Les feuilles plus âgées peuvent être mangées en soupe.
            \end{itemize}
        }

}
{%image
    plantain_lanceole.jpeg
}
{%titre photo
    Plantain lancéolé
}
{%description photo : "lieu - date"
    Jardin des thermes de \frquote{Cassinomagus} - 04/08/2024
}

\Potins{Plantain}{
    Les druides connaissaient bien le plantain et l’utilisaient pour se soigner et probablement également pour se nourrir. 
    Pour calmer les inflammations, ils utilisaient une lotion à base de plantain, de chèvrefeuille et d’autres plantes.\\
    Le plantain fait partie du « charme des neuf plantes », comprenant également l’armoise, la cardamine hérissée (cressonnette), 
    le pied de coq, la camomille, l’ortie, la pomme sauvage, le thym et le fenouil.\\
    Il symbolise l’intelligence de la mère nature, qui dépose à nos pieds ce dont nous avons besoin. Loin de se laisser perturber 
    par le fait qu’on le piétine et qu’on lui marche dessus (c’est une belle leçon de vie !), il développe une force plus grande 
    encore. Ainsi, il nous invite à prendre notre place quoiqu’il se passe.\\

    Le druide Philip Carr-Gomm nous dit que plus le plantain est piétiné, plus il renferme des ingrédients apaisants dans ses 
    feuilles et que, si dure que soit la vie, il nous montre que la guérison vient de l’intérieur.\\

    La feuille du grand plantain ressemble un peu à l’empreinte de la plante d’un pied humain, d’où son nom. \\
    Aucun rapport, bien sûr, avec la banane plantain.\\
    Le psyllium est une variété de plantain dont les graines permettent de combattre la constipation ; il est cultivé à cet effet.\\
    Les oiseaux sont très friands des graines de plantain.
}