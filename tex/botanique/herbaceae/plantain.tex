
\ficheidentiteplante
{Plantain lancéolé}
{%effet général
    L'\vocref{https://fr.wikipedia.org/wiki/Plantago_lanceolata}{Plantain lancéolé}, en latin \textit{lantago lanceolata}.
}
{%utilisation privilégiée
    \begin{itemize}[label = \bcplume]
        \item En premier lieu contre les maladies des organes respiratoires.
        \item Effet astringent
        \item Effet cicatrisant
        \item S'utilise contre les inflammation et les \voc{hémorroïdes}
    \end{itemize}
}
{%infos cueillette
    \begin{itemize}[label = \bcplume]
        \item La plante entière est à utiliser ( y compris les racines ).
        \item La floraison a lieux d'\textbf{avril} à \textbf{octobre} sur le littoral méditerranéen.
    \end{itemize}
}
{%sous quelle forme utiliser
    \begin{itemize}[label = \bccrayon]
        \item En \lien{infusion}{infusion} : $1$ c.a.c de feuilles avec $0{,}25$cL d'eau.
        \item En \voc{cataplasme} de feuilles broyées.
        \item En \voc{sirop}.
        \item Contre les piqûres (moustiques, guêpes, orties…) et les démangeaisons.\\
                Frotter une ou plusieurs feuilles sur l’endroit de la piqûre jusqu’à en extraire le suc.
        \item Les feuilles fraîches, riches en mucilages, peuvent être utilisées en cataplasme pour arrêter les \voc{saignements} ou soigner les \voc{ampoules}.
    \end{itemize}
}
{%supplément
        \boite{Usage interne :}{
            \begin{itemize}[label = \bctrefle]
                \item Toute la plante est \voc{comestible}.
                \item Les feuilles tendres ont un goût de \textbf{champignon} se mangent \textbf{crues} en salade.\\
                \item Les feuilles plus âgées peuvent être mangées en soupe.
            \end{itemize}
        }

}
{%image
    plantain_lanceole.jpeg
}
{%titre photo
    Plantain lancéolé
}
{%description photo : "lieu - date"
    Jardin des thermes de \frquote{Cassinomagus} - 04/08/2024
}
