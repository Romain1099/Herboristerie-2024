\label{houblon}
\ficheidentiteplante
{Houblon}
{%effet général
    L'\vocref{https://fr.wikipedia.org/wiki/Houblon}{houblon} ou 
    \textit{humulus lupulus} est une plante \voc{vivace} et \voc{grimpante}.

    Son huile essentielle contenue principalement dans ses \voc{inflorescences femelles} peut agir comme \voc{sédatif} du système nerveux.\\

    On peut utiliser ses tiges pour faire de la \voc{vannerie}.

}
{%utilisation privilégiée
    \begin{itemize}[label = \bcplume]
        \item L'huile essentielle de houblon est principalement utilisée pour son action sédative sur le système nerveux.
    \end{itemize}
}
{%infos cueillette
    On récupère les \voc{cônes} des \textbf{plantes femelles} riches en composés chimiques.\\
    On les récolte vers la \textbf{fin aout} lorsque les cônes sont à 
    \voc{maturité} c'est à dire quand les inflorescences sont chargées 
    de liquide \textbf{jaune-orangé} et les cônes \textbf{humides}.
}
{%sous quelle forme utiliser
    \begin{itemize}[label = \bcoeil]
        \item Pour lutter contre l'insomnie, on pourra utiliser une \voc{infusion} de cônes de \lien{houblon}{houblon}.\\
        Infuser $10$g de cônes pour $1$L d'eau.\\
        Consommer 1 à 2 tasses au coucher.
        \item On peut également consommer cette tisane pour faciliter la digestion.
    \end{itemize}
}
{%supplément
    \begin{Remarque}
        Prendre des précautions au moment de la cueillette :\\
        Le houblon peut provoquer certains effets secondaires comme des \voc{céphalées}, des \voc{effets narcotiques}, \voc{bradycardie}, \voc{anorexie}.\\

        De plus, le houblon contient un composé \vocref{https://fr.wikipedia.org/wiki/Flavonoïde}{flavonoïde} ayant un puissant pouvoir \voc{oestrogénique} 
        et qui est également responsable de la teinte jaune-orangée.
        Lors de la cueillette, il peut être absorbée par voie \voc{transcutanée} et provoquer des \voc{troubles menstruels} chez les femmes.\\
    \end{Remarque}
}
{%image
    downloads/houblon_bouba.jpg
}
{%titre photo
    Houblon
}
{%description photo : "lieu - date"
    source : wikipédia
}

