\renewcommand{\cita}{
    \phantom{a}\citer{En tant que plante magique, elle servait pour les prophéties et la divinitation.\\On la fumait}{Les plantes des druides}
}
\ficheidentiteplantelong
{Armoise annuelle}
{%effet général
    L'\vocref{https://fr.wikipedia.org/wiki/Armoise_annuelle}{Armoise annuelle}, en latin \textit{Artemisia vulgaris}, doit son nom
    à la déesse grecque \textit{Artémis} de la \textbf{lune} et de l'\textbf{accouchement}.\\
    Elle porte aussi le nom de \textbf{garde-robe} puisqu'on en mettait dans les vêtements et les armoires pour \textbf{éloigner les mites}.\\

    Très aromatiques, ses feuilles froissées dégagent une \textbf{odeur proche de l'absinthe}.
    Dans les jardins, l'armoise à la réputation de repousser les limaces, les escargots et les rongeurs.
}
{%utilisation privilégiée
    \begin{itemize}[label = \bcplume]
        \item \textbf{Antipaludique}
        \item \textbf{stimulant digestif}
        \item Aide à repousser les insectes
    \end{itemize}
}
{%infos cueillette
    \begin{itemize}[label = \bcplume]
        \item C'est une grande plante pouvant atteindre $2$m de hauteur.
        \item Ne pas confondre avec l'\textbf{ambroisie} : on les distingue par le \textbf{dessous des feuilles} qui est \textbf{argenté} pour l'\textbf{armoise}.
        \item On la trouve dans les terrains vagues, les prairies et le bordures de chemins.
        \item On utilise surtout les \textbf{feuilles} et les \textbf{tiges} qui sont anguleuses et de couleur rouge.
        \item Les fruits sont des \vocref{https://fr.wikipedia.org/wiki/Akène}{akènes} d'environ $2$mm.
        \item On cueille à la \textbf{fin de l'été} à la floraison.
    \end{itemize}
}
{%sous quelle forme utiliser
    \begin{itemize}[label = \bccrayon]
        \item \voc{Purin} : conseillé en cas de piériode du chou. Efficace en pulvérisation contre les limaces.\\ Mettez à macérer 1kg de plante fraîche dans 10L d'eau pendant 24 heures.
        \item \voc{Tisanes} : 15 à 30g par litre d'eau, 3 à 5 tasses par jour.
        \item \voc{Teinture mère} : 3 à 5 gouttes par jour.
        \item \voc{Elixir floraux} : pour aider les hypersensibles à s'enraciner. Cela apaise harmonise et rééquilibre la sensibilité.
        \item Les \voc{cendres} de la plante arrêtent les \voc{saignements de nez}.
    \end{itemize}
}
{%supplément
    \begin{multicols}{2}

        \boite{Usage interne :}{
            \begin{itemize}[label = \bctrefle]
                \item l'armoise est surtout connue pour être \vocref{https://fr.wikipedia.org/wiki/Emménagogue}{emménagogue} pour régulariser ou provoquer les règles, contre les douleurs menstuelles et lors de la ménopause.
                \item Elle est églament efficace pendant l'accouchement surtout en cas de rétention de placenta.
                \item On l'utilise aussi pour les digestion lente, l'inappétence, les crampes d'estomac, les parasites intestinaux, l'anémie, l'épilepsie, les troubles nerveux.
            \end{itemize}
        }

        \columnbreak

        \boite{Usage externe :}{
            \begin{itemize}[label = \bccrayon]
                \item Les bains d'armoises sont conseillés contre la goutte et les rhumatismes.
                \item les cendres de la plante arrêtent les saignements de nez.
                \item Elle est souvent utilisée comme une alternative au tabac
            \end{itemize}
        }

    \end{multicols}

}
{%image
    armoise_vulgaris.jpg
}
{%titre photo
    Armoise annuelle
}
{%description photo : "lieu - date"
    Excursion - Tessé - 02/08/2024\\Dessous des feuilles \frquote{argenté}
}
\renewcommand{\cita}{}
\begin{Remarque}
    \begin{itemize}[label = \bcplume]
        \item Cette plante est fortement \textbf{déconseillée} en cas de \textbf{grossesse et d'allaitement}.
        \item Elle est toxique en cas d'emploi prolongé
        \item Ne pas confondre avec l'ambroisie qui n'a pas le dessous des feuilles argentée
    \end{itemize}
\end{Remarque}
