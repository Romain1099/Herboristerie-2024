\ficheidentiteplante
{Achillée millefeuille}
{%effet général
    L'\vocref{https://fr.wikipedia.org/wiki/Achill\%C3\%A9e_millefeuille}{achillée millefeuille} est considérée comme la \voc{plante de la femme} de base.\\ 
    Elle est utilisée dans la pharmacopée pour son effet sur les \voc{saignements} et \voc{désinfectant}.\\
    On dit que c'est une plante \vocref{https://fr.wikipedia.org/wiki/Emm\%C3\%A9nagogue}{emménagogue}.\\

    La plante a également un effet sur la \textbf{digestion}.
}
{%utilisation privilégiée
    \begin{itemize}[label = \bcplume]
        \item Dans le cas d'un oedème
        \item \textbf{règles}
        \item \textbf{varices}
        \item Saignements ( même importants )
    \end{itemize}
}
{%infos cueillette
    ?
}
{%sous quelle forme utiliser
    \begin{itemize}[label = \bccrayon]
        \item Plante sèche
        \item \voc{Macérat huileux}
        \item \voc{Teinture mère}
    \end{itemize}
}
{%remarques
    \begin{itemize}[label = \bcplume]
        \item C'est une plante plutôt \voc{amère}, \voc{astreingeante}.
        \item Se combine avec des \textbf{feuilles de framboises} pour faciliter la \voc{reminéralisation} après un saignement.
        \item C'est une plante \vocref{https://fr.wikipedia.org/wiki/Liste_de_plantes_tinctoriales}{teinctoriale} associée à la couleur \textbf{vert kacki}.\\Elle est utilisée pour tremper les vêtements militaires ce qui leur donne cette couleur. 
        \item[\bcattention] En cas d'urgence, \textbf{mâcher} la plante de sorte à constituer une \textbf{pâte}.\\ \textbf{Appliquer directement sur la plaie}.
    \end{itemize}
}
{%image
    achillee_millefeuille_wiki.jpeg
}
{%titre photo
    Achillée millefeuille
}
{%description photo : "lieu - date"
    source : Wikipedia 
}
