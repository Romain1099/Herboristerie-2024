\ficheidentiteplante
{Achillée millefeuille}
{%effet général
    L'\vocref{https://fr.wikipedia.org/wiki/Achillée_millefeuille}{achillée millefeuille} en latin \textit{Achillea millefolium} est considérée comme la \voc{plante de la femme} de base.\\ 
    On l'appelle aussi \textit{herbe au sang} ou encore \textit{sourcils de vénus}.
    Elle est utilisée dans la pharmacopée pour son effet sur les \voc{saignements} et \voc{désinfectant}.\\
    
    On dit que c'est une plante \vocref{https://fr.wikipedia.org/wiki/Emm\%C3\%A9nagogue}{emménagogue}.\\
    
    La plante a également un effet sur la \textbf{digestion}.
}
{%Effets recherchés
    \begin{itemize}[label = \bcplume]
        \item Dans le cas d'un oedème
        \item \textbf{Règles}
        \item \textbf{Varices}
        \item Saignements ( même importants )
        \item Problèmes de digestion
        \item Propriétés \vocref{https://fr.wikipedia.org/wiki/Antispasmodique}{antispasmodiques}
    \end{itemize}
}
{%infos cueillette
    L'Achillée fleurit de \textbf{mai à septembre} et peut mesurer jusqu'à $2$m de hauteur.\\

    Ses fleurs se présentent en \textbf{amas} nommés \vocref{https://fr.wikipedia.org/wiki/Corymbe}{corymbes} et ses fleurs sont formées de \vocref{https://fr.wikipedia.org/wiki/Capitule_\%28botanique\%29}{capitules}.
    
}
{%Utilisation privilégiée
    \begin{itemize}[label = \bccrayon]
        \item Plante sèche
        \item \voc{Macérat huileux}$\longrightarrow$ extraction des terpènes
        \item \voc{Teinture mère}
        \item Tisanes
    \end{itemize}
}{%remarques.
    \begin{Remarque}

        C'est une plante plutôt commune en France.\\
        C'est une plante \vocref{https://fr.wikipedia.org/wiki/Plante_vivace}{vivace} et \vocref{https://fr.wikipedia.org/wiki/Polymorphisme}{polymorphe}\\
        Il y a beaucoup de \textbf{sous-variétés}, donc il est indispensable de connaître le \textbf{nom latin} lors d'un \textbf{achat}.
        \begin{itemize}[label = \bcplume]
            \item C'est une plante plutôt \voc{amère}, \voc{astreingeante}.
            \item Consommée \textbf{avant les repas} elle permet d'\textbf{éviter la rétention d'eau} $\longrightarrow$ fleurs
            \item Se combine avec des \textbf{feuilles de framboises} pour faciliter la \voc{reminéralisation} après un saignement.
            \item C'est une plante \vocref{https://fr.wikipedia.org/wiki/Liste_de_plantes_tinctoriales}{teinctoriale} associée à la couleur \textbf{vert kaki}.\\
                Elle est utilisée pour tremper les vêtements militaires ce qui leur donne cette couleur. 
            \item[\bcattention] En cas d'urgence, \textbf{mâcher} la plante de sorte à constituer une \textbf{pâte}.\\ 
                \textbf{Appliquer directement sur la plaie}.
        \end{itemize}
    \end{Remarque}
}
{%image
    achillee_millefeuille_wiki.jpeg
}
{%titre photo
    Achillée millefeuille
}
{%description photo : "lieu - date"
    source : Wikipedia 
}
\Potins{Achillée millefeuille}{
    Les Celtes utilisaient ses tiges dans leurs rites divinatoires. 
    Les premiers occupants du continent Américain s'en servaient pour repousser les maléfices et les mauvais esprits. 

Cette plante, qui doit son nom actuel à l'usage que, sur le conseil de Venus, en fit Achille pour guérir la blessure 
de Télèphe \\
\textit{Il existe une autre version de la guérison de Télèphe: selon une prédiction d'Apollon, 
ce qui avait blessé Télèphe, allait le guérir}.\\

Éclairé par un conseil d'Ulysse, Achille, qui avait blessé Télèphe 
de sa lance, pris une de rouille de sa lance et la disposa sur la blessure qui cicatrisa - voir \cite{dicomythologie} - 
avait chez les anciens la réputation d'avoir un puissant pouvoir vulnéraire. 
 
En Chine, les tiges d’achillée découpées en bâtonnets servent pour les oracles (selon les instructions du Yi King). 
C’est une plante importante pour la médecine traditionnelle chinoise.\\
Les jeunes feuilles hachées font un excellent condiment pour parfumer du fromage blanc, du beurre, des salades, 
des plats cuisinés, ainsi que certaines liqueurs.\\
Elles sont riches en vitamines et en oligoéléments. 
Les fleurs ont servi à aromatiser la bière, et les vignerons auraient utilisé les graines comme conservateur dans le 
vin. \\
On brûlait l’achillée dans les étables ou dans les maisons pour repousser les insectes.
}