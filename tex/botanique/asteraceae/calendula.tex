

    
\ficheidentiteplante
{Calendula}
{%effet général
    Le \vocref{https://fr.wikipedia.org/wiki/Calendula}{calendula} en latin \voc{Calendula officinalis} – le souci des jardins.\\
    Il fait partie de la famille des \voc{asteraceae} ou \textbf{fleurs composées}.
    %Miryam fait mention de quelque chose de différent ( erreur ? ) a check
    %brassicacées (crucifères) : 4 pétales : le chou, le navet, le radis, la moutarde, l'alliaire officinale ;

}
{%utilisation privilégiée
    \begin{itemize}[label = \bcplume]
        \item S'utilise pour apaiser et réparer

    \end{itemize}
}
{%infos cueillette
    \begin{itemize}[label = \bcplume]
        \item a
    \end{itemize}
}
{%sous quelle forme utiliser
    \begin{itemize}[label = \bccrayon]
        \item a
    \end{itemize}
}
{%supplément
    

}
{%image
    downloads/calendula_officinalis1.jpg
}
{%titre photo
    Fleur de calendula
}
{%description photo : "lieu - date"
    source - wikipedia
}


