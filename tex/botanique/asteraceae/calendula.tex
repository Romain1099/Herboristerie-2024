

\label{calendula}
\renewcommand{\cita}{
    \phantom{a}\citer{Calendula, le souci qui chasse les soucis, est un guérisseur du cœur aussi bien que du corps.}{}
}
\ficheidentiteplante
{Calendula}
{%effet général
    Le \vocref{https://fr.wikipedia.org/wiki/Calendula}{calendula} en latin \voc{Calendula officinalis} – le souci des jardins - \frquote{pot-marigold}.\\
    Il fait partie de la famille des \voc{asteraceae} ou \textbf{fleurs composées}.\\

    C'est une plante à \vocref{https://fr.wikipedia.org/wiki/Mucilage}{mucilage} ce qui lui confère des vertus \textbf{apaisantes}.\\

    Les \vocref{}{flavonoïdes} contenus dans la plante sont des \voc{anti-inflammatoires} et des \voc{anti-oxydants}.\\

    Contient des \vocref{https://fr.wikipedia.org/wiki/Saponine}{saponides} et des \vocref{https://fr.wikipedia.org/wiki/Coumarine}{coumarines}

    Il est riche \voc{vitamines C et E} et contient de l'\voc{acide salicylique}.

    %Miryam fait mention de quelque chose de différent ( erreur ? ) a check
    %brassicacées (crucifères) : 4 pétales : le chou, le navet, le radis, la moutarde, l'alliaire officinale ;


}
{%utilisation privilégiée
    Ce sont les \textbf{fleurs} qui ont des vertus thérapeutiques.
    \begin{itemize}[label = \bcplume]
        \item S'utilise pour apaiser et réparer de façon générale $\longrightarrow$ \vocref{https://fr.wikipedia.org/wiki/Mucilage}{mucilage}
        \item Calme les irritations et douleurs \voc{cutanées} ou \voc{musculaire}.
        \item L'infusion peut s'appliquer pour soulager les \voc{piqûres d'insectes}
    \end{itemize}
}
{%infos cueillette
    \begin{itemize}[label = \bcplume]
        \item On récolte la \voc{fleur} en \voc{été} juste \textbf{après son éclosion} pour en faire un macérat huileux. 
    \end{itemize}
}
{%sous quelle forme utiliser
    \begin{itemize}[label = \bccrayon]
        \item Les fleurs en \voc{macérat huileux}
        \item Les fleurs en \voc{infusion}\\
                $25$g de plante pou $1$L d'eau\\
                En cure, consommer $4$ tasses par jour entre les repas.
        \item On peut utiliser l'infusé comme lotion \voc{antiseptique} et  \voc{cicatrisante}.
        \item L'huile essentielle de calendula peut également être utilisée.
    \end{itemize}
}
{%supplément
    \begin{multicols}{2}

        \boite{Utilisation interne :}{

                \textbf{Utilisation culinaire :}\\
                Les pétales de calendula sont appelés \frquote{safran du pauvre} et peuvent être utilisés en remplacement.\\

                \textbf{En infusion :}\\
                Préparée avec 1 à 2 cuillères à café de fleurs séchées et 200 ml d’eau bouillante. \\
                Elle soulage les mycoses et les troubles digestifs. \\
                Elle est efficace pour réguler le cycle menstruel en stimulant la production d’œstrogènes. \\
                La posologie est de deux à trois prises par jour.
        }

        \columnbreak

        \boite{Utilisation externe}{
            Le \voc{macérat huileux} de calendula est appliquée en massage pour apaiser les peaux \voc{sèches etirritées}, deux à trois fois par jour. \\
            Ses propriétés \voc{adoucissantes} et calmantes conviennent à la peau délicate des bébés et des jeunes enfants.\\

            Les \lien{baume}{baumes}, gel, crème, ou pommades basés sur le calendula sont susceptibles de soigner : 
            \begin{itemize}[label = \bccrayon]
                \item les \voc{écorchures}
                \item les \voc{brûlures}
                \item les \voc{irritations}
                \item \voc{crevasses cutanées}
            \end{itemize}
        }
    \end{multicols}


}
{%image
    downloads/calendula_officinalis1.jpg
}
{%titre photo
    Fleur de calendula
}
{%description photo : "lieu - date"
    source - wikipedia
}
\renewcommand{\cita}{}
\begin{Remarque}
    \bcattention Les fleurs sont réputées \voc{oestrogéniques}, il faut donc \voc{éviter durant la grossesse}.
\end{Remarque}
