\ficheidentiteplante
{Sauge officinale}
{%effet général
    La voc{sauge} est considérée comme une \voc{plante de la femme}.\\ 
    Elle est utilisée dans la pharmacopée pour permettre un \voc{rééquilibre hormonal}.
}

{%utilisation privilégiée
    Lors d'épisodes de déséquilibre hormonal chez la femme, l'utilisation peut réduire entre autres la \voc{transpiration}. 
}

{%infos cueillette
    \begin{itemize}[label = \bcplume]
        \item \textbf{feuilles} et les \textbf{sommités fleuries} sont les parties utilisées de la plante.
        \item La sauge officinale se récolte idéalement au \voc{début de l'été}, juste avant la \voc{floraison}, lorsque les feuilles sont les plus riches en principes actifs. 
        \item La cueillette doit se faire de préférence le matin, par temps sec, en prenant soin de ne pas abîmer la plante.
        \item  \textbf{après séchage}.
    \end{itemize}
    .

}
{%sous quelle forme utiliser
    \begin{itemize}[label = \bcplume]
        \item \textbf{Bain de bouche}
        \item \textbf{Fumigation} après séchage.
    \end{itemize}
}

{%supplément
\begin{multicols}{2}

    \boite{Usage interne :}{
        \begin{itemize}[label = \bctrefle]
            \item \textbf{infusion} : pour réguler les hormones, stimuler la digestion, et apaiser les troubles digestifs. 
            \item Elle est également connue pour son action tonique et ses propriétés antioxydantes.
        \end{itemize}
    }

    \columnbreak

    \boite{Usage externe}{
        \begin{itemize}[label = \bccrayon]
            \item \textbf{bain de bouche} : pour ses propriétés antiseptiques et anti-inflammatoires, particulièrement efficace pour traiter les maux de gorge et les infections buccales. 
            \item \textbf{compresse} : pour soulager les affections cutanées comme l'acné ou les piqûres d'insectes.
            \item \textbf{fumigation} : après séchage pour purifier l'air et chasser les énergies négatives.
        \end{itemize}
    }

\end{multicols}
}

{%remarques
    La sauge officinale est connue pour son efficacité mais doit être utilisée avec précaution. Elle est déconseillée aux \voc{femmes enceintes} et \voc{allaitantes}, ainsi qu'aux personnes souffrant d'\voc{hypertension}. 
}
{%image
    sauge_officinale.jpeg
}
{%titre photo
    Sauge officinale
}
{%description photo : "lieu - date"
    Jardin des thermes de \frquote{Cassinomagus} - 04/08/2024 
}
