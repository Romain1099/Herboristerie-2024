\ficheidentiteplantelong
{Romarin}
{%effet général
    Le \voc{romarin}, en latin \textit{Salvia rosmarinus} est également appelée \frquote{herbe aux couronnes} ou \frquote{encensier} en raison de son \textit{odeur camphrée}.\\

    C'est une plante \voc{condimentaire} \voc{aromatique} et \vocref{https://fr.wikipedia.org/wiki/Flore_mellifère}{mélifère} qui est originaire du bassin méditerranéen.\\
    

}
{%Effets recherchés
Fraîche ou séchée, cette herbe condimentaire fait partie de la cuisine méditerranéenne, et une variété se cultive dans les jardins.  \\
C'est également un produit fréquemment utilisé en parfumerie. \\
Enfin, diverses vertus phytothérapeutiques ont été étudiées.
}
{%infos cueillette
    La floraison commence dès le mois de \textbf{février}, parfois en janvier, et se poursuit jusqu'en \textbf{avril-mai}.
    Il est reconnaissable en toute saison à ses feuilles persistantes \vocref{https://fr.wikipedia.org/wiki/Pétiole}{sans pétiole}, coriaces, beaucoup plus longues que larges, 
    aux \textbf{bords légèrement enroulés}, \textbf{vert sombre} luisant sur le dessus, \textbf{blanchâtres}w en dessous.\\

    Il se multiplie facilement au printemps ou à l'automne par \vocref{https://fr.wikipedia.org/wiki/Bouturage}{bouturage} ou \vocref{https://fr.wikipedia.org/wiki/Marcottage}{marcottage} ; 
    plus difficilement en été par semis car sa germination est lente.
}
{%utilisation privilégiée
\textbf{En teinture mère :}
    \begin{itemize}[label = \faPen]
        \item Effets sur le système nerveux
        \item Effet anti-dépressif
    \end{itemize}
    \textbf{En macérat huileux :}
    \begin{itemize}[label = \faPen]
        \item Activer la digestion $\longrightarrow$ effet \voc{hépatoprotecteur}
        \item Effets sur la circulation sanguine
        \item Effet anti-coagulant
    \end{itemize}
    \textbf{En extrait aqueux :}
    \begin{itemize}[label = \faPen]
        \item Effet anti-spasmodique
    \end{itemize}
    \textbf{En huile essentielle :}
    \begin{itemize}[label = \faPen]
        \item Effet anti-bactérien
    \end{itemize}
}
{%remarques
    \begin{Remarque}
        L'huile essentielle de romarin peut avoir des effets neurotoxiques, déclencher \voc{convulsions} et \voc{crises d’épilepsie}.\\
        Par voie orale, et à part l'utilisation en cuisine, il est déconseillé aux femmes enceintes ou allaitantes.
    \end{Remarque}
}
{%image
    romarin.jpg
}
{%titre photo
    Romarin
}
{%description photo : "lieu - date"
    Jardin des thermes de \frquote{Cassinomagus} - 04/08/2024 
}

