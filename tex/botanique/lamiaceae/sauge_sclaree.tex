\ficheidentiteplantelong
{Sauge sclarée}
{%effet général
    La sauge sclarée (\textit{Salvia sclarea}) est une plante médicinale réputée pour ses multiples bienfaits. 
    Elle est considérée comme une herbe \voc{adaptogène}. \\

    Elle est utilisée dans la pharmacopée traditionnelle pour ses propriétés \voc{anti-inflammatoires}, \voc{antispasmodiques}, 
    \voc{antiseptiques}, et \voc{aphrodisiaques}. \\

    La sauge sclarée est particulièrement appréciée pour son \voc{action équilibrante} sur les \voc{hormones},
    notamment pour soulager les symptômes de la \voc{ménopause} et des \voc{menstruations douloureuses}.
}
{%utilisation privilégiée
    L'utilisation privilégiée de la sauge sclarée se situe principalement dans le domaine des \voc{soins hormonaux}, 
    en particulier pour les \voc{femmes}. \\

    Elle est couramment utilisée pour soulager les \voc{bouffées de chaleur}, l'\voc{irritabilité}, et d'autres symptômes associés 
    à la ménopause. Elle est également utilisée pour atténuer les \voc{douleurs menstruelles} et \voc{réguler les cycles irréguliers}.\\

    De plus, l'\voc{huile essentielle} de sauge sclarée est souvent employée en aromathérapie pour ses \voc{effets calmants} 
    et \voc{anti-dépressifs}.
}
{%infos cueillette
    \begin{itemize}[label = \bcplume]
        \item \textbf{feuilles} et les \textbf{sommités fleuries} sont les parties utilisées de la plante.
        \item La sauge sclarée se récolte généralement en été, de préférence le matin, lorsque les fleurs sont bien ouvertes et 
                que la concentration en huiles essentielles est à son maximum. 
        \item Faire sécher dans un endroit ombragé pendant 1 à 2 semaines. La sauge est prête lorsque les feuilles sont sèches au toucher 
                et se cassent facilement.
    \end{itemize}
}
{%sous quelle forme utiliser
    \begin{itemize}
        \item \textbf{Infusion :} Les feuilles séchées peuvent être utilisées pour préparer des infusions.
        \item \textbf{Capsules :} Sous forme de complément alimentaire, souvent pour les troubles hormonaux.
        \item \textbf{Huile essentielle :} Utilisée en massage, en diffusion, ou en bain.
        \item \textbf{Fumigation après séchage :} Utilisée pour purifier l'air ou pour des rituels apaisants.
    \end{itemize}
}
{%supplément
\begin{multicols}{2}

    \boite{Usage interne :}{
        \begin{itemize}[label = \bctrefle]
            \item \textbf{tisane ou d'infusion} : cela aide à réguler les hormones, à soulager les troubles digestifs, et à apaiser le stress et l'anxiété. 
            \item \textbf{capsules ou extraits} : sous forme de compléments et souvent utilisés pour leur action régulatrice sur les hormones.
        \end{itemize}
    }

    \columnbreak

    \boite{Usage externe :}{
        \begin{itemize}[label = \bccrayon]
            \item \textbf{bain et feuilles séchées} : Elle peut aussi être ajoutée à l'eau du bain pour ses effets relaxants ou être appliquée en compresse sur la peau pour traiter des affections cutanées comme l'eczéma ou les inflammations. 
            \item \textbf{huile essentielle} : Celle-ci peut être diluée dans une huile végétale pour être utilisée en massage sur le bas-ventre, afin de soulager les crampes menstruelles ou les douleurs liées à la ménopause. 
            \item \textbf{fumigation} : après séchage pour purifier l'air et créer une ambiance apaisante.
        \end{itemize}
    }

\end{multicols}
}
{%image
    sauge-sclaree.jpg
}
{%titre photo
    Sauge sclarée
}
{%description photo : "lieu - date"
    Site de \vocref{https://www.aromatherapie-huiles-essentielles.com/}{\frquote{aromatherapie}}% - 12/08/2024 
}

\begin{Remarque}
        
    %remarques
    La sauge sclarée est généralement bien tolérée, mais elle doit être utilisée avec prudence en raison de ses \voc{effets hormonaux} 
    potentiellement \voc{puissants}. \\

    Elle est déconseillée aux \voc{femmes enceintes} ou \voc{allaitantes}, ainsi qu'aux personnes souffrant de certaines pathologies 
    comme l'\voc{épilepsie}. 

\end{Remarque}