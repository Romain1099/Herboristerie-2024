En France, la vente des plantes médicinales (inscrite à la \voc{pharmacopée}), est réservée aux pharmaciens, à l’exception de 148 espèces libérées et d’une centaine d’aromates et épices.


Il existe deux listes de plantes médicinales, la liste A et B. \\
La liste A comprend les plantes testées scientifiquement et bonnes par la santé .\\
la liste B comprend les plantes poison.\\
Nous parlerons que de la liste A, qui comprend 148 plantes autorisées à la production et à la vente. D'un point de vue légal, il n'y a que 7 plantes autorisées.\\

On utilise la \lien{biglist}{liste des 148} qui répertorie les informations de base sur les plantes librement accessibles. \\

Je me suis basé sur le site internet suivant pour la construire : \\
\begin{center}\href{https://www.passerelleco.info/spip.php?page=article\&id_article=407}{https://www.passerelleco.info/spip.php?page=article\&id\_article=407}\end{center}

\begin{Defi}[Sommité fleurie]
    \label{sommite}
    La \voc{sommité fleurie} d'une plante désigne la partie aérienne contenant l'\vocref{https://fr.wikipedia.org/wiki/Inflorescence}{inflorescence} sommitale ou apicale.\\
    Cette sommité fleurie se compose de la zone florale avec les fleurs, les feuilles et la tige.\\
    Elle inclut une grande partie de la tige florale, où débute la première fleur (ou fleuron).
\end{Defi}

\begin{Defi}[Partie aérienne]
    \label{aerienne}
    La \voc{partie aérienne} d'une plante désigne la structure émergée de la plante.\\
    Elle a pour fonction de soutenir et de structurer la plante, en soutenant ses autres organes végétaux aériens, tels que les feuilles et les fleurs.\\ Une autre de ses principales caractéristiques est qu'elle a un géotropisme négatif, ce qui signifie qu'elle pousse dans la direction opposée à la gravité.\\
    source : \href{https://www.projetecolo.com/anatomie-d-une-plante-les-parties-d-une-plante-avec-schema-31.html}{internet}.
\end{Defi}

\boite{Parties utilisées dans les plantes :}{
\begin{minipage}[t]{0.55\textwidth}
    Il y a plusieurs parties de la plante qui peuvent être utilisées selon les plantes : 
\end{minipage}
\citer{Quand on voit ce qu'on voit, et qu'on entend ce qu'on entend, au bout d'un moment c'est quand même normal de penser ce qu'on pense\ldots}{Un grand monsieur}

\begin{multicols}{2}
    \begin{itemize}[label = \bcfleur]
        \item Les \vocref{https://fr.wikipedia.org/wiki/Bourgeon_(botanique)}{bourgeons}
        \item Les \vocref{https://fr.wikipedia.org/wiki/Fleur}{fleurs}
        \item Les \lien{sommite}{sommités fleuries}
        \item Les \vocref{https://fr.wikipedia.org/wiki/Racine_(botanique)}{racines}
        \item Les \vocref{https://fr.wikipedia.org/wiki/Pétale}{pétales}
        \item Les \vocref{https://fr.wikipedia.org/wiki/Racine_(botanique)}{parties souterraines}
        \item Les \vocref{https://fr.wikipedia.org/wiki/Feuille}{feuilles}
        \item La \vocref{https://fr.wikipedia.org/wiki/Sève}{sève}
        \item Les \vocref{https://fr.wikipedia.org/wiki/Rhizome}{rhizomes}
        \item Les \lien{aerienne}{parties aériennes}
        \item Les \vocref{https://fr.wikipedia.org/wiki/Graine}{graines} ou \vocref{https://fr.wikipedia.org/wiki/Semence_(agriculture)}{semences}
        \item Les plantes entières (l'ortie)
        \item Les \vocref{https://fr.wikipedia.org/wiki/Cône_(botanique)}{cônes}. 
    \end{itemize}
\end{multicols}
}

\boite{Nommer les plantes}{
    Il existe plusieurs manières de nommes les plantes : le \textbf{nom latin} mais aussi des \textbf{noms communs} qui donnent des informations sur l'usage de la plante.\\
    Par exemple, le plantain, appelé aussi l'herbe à cinq coutures, le lancéolé et le major.\\

    Pour une \textbf{utilisation médicinale}, il est important de noter le \textbf{nom latin} afin d'être certain des \textbf{propriétés de la plante}.
}

\boite{Propriétés des plantes}{
    Toutes les plantes ont des propriétés \cite{plantemed}. 
}

\boite{Les familles de plantes}{
    Toutes les familles des plantes se terminent par -acée (en français) ou par -aceae (en latin).\\
    On pourra trouver des compléments ici sur la \vocref{https://parlonssciences.ca/ressources-pedagogiques/documents-dinformation/la-taxonomie-vegetale}{taxonomie végétale} ou là pour les \vocref{https://verstige.fr/familles}{familles de plantes}.
    
 
}

\subsection{Cueillette}

\bcattention Pour \voc{cueillir} les plantes, on \textbf{préserve} le plus possible la plante. \\
Il est donc préférable de les cueillir délicatement et \textbf{les plus entières possibles}. 


\subsection{Conservation}

\bcattention Pour \voc{conserver} les plantes séchées, on les \textbf{laisse entières} pour qu'elles conservent leurs propriétés.

