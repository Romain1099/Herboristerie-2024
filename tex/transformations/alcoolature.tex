\subsection{Teintures mères}

\begin{Defi}[Teinture mère]

    Une \voc{teinture mère} désigne une préparation de plantes infusées dans de l'\voc{alcool}. \\
    L'utilisation professionnelle de ce terme est \textit{réservé} aux pharmaciens. Nous utiliserons donc le terme commun d'\voc{alcoolature} dans la suite de ce document.\\

    On parle de \voc{teinture officinale} lorsque la préparation est réalisée à l'aide de \textbf{plantes sèches}.\\

    On parle de \voc{dynamiser} une solution lorsqu'on la \textbf{mélange}.\\

    La \voc{diffusion} des principes actifs s'effectue en \voc{dynamisant} le bocal \textbf{chaque jour}. 
\end{Defi}
\begin{Remarque}[]%\monimage{0.05}{avec_sourire.png}

    \begin{itemize}[label=\faPen]
        \item Plus le degré d'alcool est élevé, plus les principes actifs se diffuseront efficacement. 
        \item Permet d'effectuer une préparation même si on dispose d'une \textbf{faible quantité de plantes}. 
        \item On peut utiliser indiféremment des plantes \voc{fraîches} ou \voc{sèches}.
        \item Cela permet d'extraire des \voc{principes actifs} \voc{complexes} de la plante. Par exemple :
                \begin{multicols}{3}
                    \begin{itemize}
                        \item gomme et résine.
                        \item \vocref{https://fr.wikipedia.org/wiki/Alcalo\%C3\%AFde}{alcaloïdes} 
                        \item les principes \voc{volatiles}
                    \end{itemize}
                \end{multicols}
    \end{itemize}
\end{Remarque}
\begin{multicols}{2}
    \boite{Préparation :}{
        Utiliser un alcool assez \textbf{fort} ( type rhum, absinthe\ldots ). \\
        \textbf{Degré d'alcool souhaité} : entre $40\degres$ et $90\degres$. \\
        \begin{itemize}[label=\mysquare]
            \item Découper les plantes \textbf{séchées} à l'aide d'un \voc{sécateur} ou d'une \voc{paire de ciseaux}.\\Il est également possible de les broyer à l'aide d'un mortier.
            \item Les placer dans un bocal adapté à la taille de la cueillette, \textbf{à ras} et \textbf{sans tasser}.
            \item Couvrir d'\voc{alcool} en veillant à \textbf{éliminer} les \textbf{bulles d'air}.
            \item Refermer le bocal et conserver dans un environnement \textbf{propre} et \voc{à l'abri de la lumière}.
        \end{itemize}
    }

        
    \columnbreak 


    \boite{Macération :}{
        \begin{itemize}[label=\mysquare]
            \item \voc{Dynamiser} chaque jour pendant \textbf{28 jours} pour permettre la \voc{diffusion} des principes actifs. 
        \end{itemize}
    }\\
    \hfill
	\includeprepa[0.2]{couper_plantes.jpg}{Découper les plantes}{30/07/2024}
\hfill
    \includeprepa[0.2]{alcoolature_coquelicot.jpg}{Remplir le bocal}{30/07/2024}
\hfill


\end{multicols}
\begin{center}

\boite{Conservation :}{
    \begin{itemize}[label=\faPen]
        \item Se conserve \voc{à l'abri de l'humidité}.
        \item La teinture mère se conserve sur une période allant de 2 à 5 ans.% \monimage{0.1}{groupe/tetra.png}
    \end{itemize}
    
}
\end{center}
\boite{Conseils d'utilisation}{
    \begin{itemize}[label=\faPen]
        \item Une attention toute particulière est à porter à l'utilisation de teinture mère. 
        \item L'automédication n'est pas une pratique à prendre à la légère et il est conseillé de s'entretenir avec un professionnel. 
        \item Les dosages ne sont plus soumis à autant de contraintes qu'auparavant.\\
                Les entreprises pharmaceutiques sont libres de doser la quantité de plante dans leurs teintures mères sans l'indiquer. 
                Néanmoins, voici quelques bonnes pratiques concernant la posologie : \\
    \end{itemize}
    \boite{Posologie :}{
        \begin{itemize}[label=\faPen]
            \item Pour un alcool à $50\degres$, la posologie indiquée est $\mathbf{30}$\textbf{gouttes par jour}. 
            \item De façon générale, commencer avec une dose réduite puis augmenter progressivement. 
        \end{itemize}
    }
}

