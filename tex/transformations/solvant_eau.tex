\subsection{Solvant eau}


\begin{Defi}[Infusion]
	
	Une \voc{infusion} consiste à faire \voc{macérer} une plante dans de l'\voc{eau chaude}.\\

	C'est l'effet de la \textbf{chaleur} qui permet de diffuser les \voc{principes actifs}.
	
	\boite{Préparation :}{
		Pour préparer une infusion, de \textbf{recouvrir} les plantes d'une eau à $75\degres$.\\
		
		Laisser \textbf{infuser} $15$ minutes.
	}
\end{Defi}



\begin{Remarque}
	\begin{itemize}[label = \faPen]
		\item \textbf{Couvrir} le mélange pour garder les \voc{principes volatiles}.
		\item L'eau ne doit pas être bouillante $\longrightarrow$ casse les molécules.
	\end{itemize}
\end{Remarque}

\begin{Defi}[Décoction]
	
	Une \voc{décotion} consiste à faire \voc{macérer} une plante dans de l'\voc{eau chaude}.\\

	A la \textbf{différence} de l'infusion, la \textbf{décoction} demande un \textbf{départ à froid}. \\
	
	Ce procédé d'extraction est intéressant lorsque la plante est \textbf{dure} ( par exemple de l'\textbf{écorce}, les \textbf{baies}... ).

	\boite{Préparation :}{
		\begin{itemize}
			\item[\bcoutil] Pour préparer une décoction, \textbf{recouvrir} les plantes d'une eau à $\approx20\degres$.\\

			\item[\bchorloge] Porter à \voc{ébullition} pendant $5\text{ à }30$ minutes.

			\item[\bcoutil] Filtrer et laisser reposer selon l'utilisation future du mélange.
		\end{itemize}
	}
\end{Defi}

\boite{Intérets de l'utilisation du solvant \frquote{eau} :}{
	Un \textbf{intéret pratique} de l'utilisation de l'eau est la \textbf{rapidité} d'accès au produit fini.\\

	De plus, il faut boire presque $2$ litres d'eau par jour.\\
	Les tisanes sont donc un excellent moyen d'éviter de se déshydrater en plus des effets apportés par les plantes. 
	\begin{multicols}{2}
	\textbf{Principes actifs récupérés : }
	\begin{itemize}[label = \faPen]
		\item \voc{Vitamines}
		\item \voc{Minéraux}
		\item \voc{Sucres} ( saccharides )
		\item \voc{Principes amers} $\longrightarrow$ digestion
	\end{itemize}

	\textbf{Modes d'utilisation : }
	\begin{itemize}[label = \bcoutil]
		\item \voc{Cataplasme}
		\item \voc{Bain de plantes}
		\item \voc{Compresses}
		\item \voc{Inhalation}
	\end{itemize}

	\columnbreak

	\includeprepa[0.4]{presentation_infusion_sacrees_plantes_6.jpg}{Infusion}{31/07/2024}
\end{multicols}
}

\begin{Remarque}[Mémoire de l'eau]
	Les travaux sur l'\voc{homéopathie} et la \voc{mémoire de l'eau}, pourront intéresser le lecteur curieux. 
	Les teintures mères sont les souches en homéopathie.\\
	Elles sont ensuite diluées dans de l'eau.
	\textbf{Exemple} : 2CH correspond à $2$ dillutions successives à $1\%$. 

\end{Remarque}