\subsection{Macérat huileux}

\begin{Defi}[Macérat huileux]

    Un \voc{macérat huileux} ou \voc{macérat solaire} désigne une infusion de plantes dans un \voc{corps gras}. \\
    Ici, c'est l'effet de la \textbf{chaleur} qui permet la \voc{diffusion} des principes actifs.
\end{Defi}

\begin{multicols}{2}
    \boite{Préparation}{
        \begin{itemize}[label=\mysquare]
            \item Découper les plantes \textbf{séchées}.
            \item Les placer dans un bocal \textbf{à ras} et \textbf{sans tasser}.
            \item Couvrir d'huile \textbf{végétale} en veillant à \textbf{éliminer} les \textbf{bulles d'air}.
            \item Laisser le bocal ouvert 24h dans un environnement \textbf{propre} et \voc{ensoleillé} ( ou chaud ). 
        \end{itemize}
    }

    \boite{Macération}{
        \begin{itemize}[label=\mysquare]
            \item Refermer le pot en éliminant la \voc{condensation}. 
            \item Conserver \voc{au soleil} ou \voc{au chaud} pendant $\mathbf{28}$\textbf{ jours}.
        \end{itemize}
    }
\end{multicols}

\boite{Conservation}{
    \begin{itemize}[label=\faPen]
        \item Se conserve \voc{à l'abri de l'humidité}.
        \item Le macérat huileux peut se conserver sur une période allant de 6 mois à 2 ans.% \monimage{0.1}{groupe/tetra.png}
    \end{itemize}
    
}
\boite{Conseils d'utilisation}{
    Il est conseillé de \voc{filtrer} le mélange avant d'utiliser l'huile. \\
    Selon l'huile choisie, on pourra consommer le macérat lors des repas. 
}

\begin{Remarque}[]%\monimage{0.05}{avec_sourire.png}
    Les huiles les plus \voc{stables} sont l'\voc{huile d'olive} et l'\voc{huile de tournesol} et permettent une conservation sur \textbf{2 ans}. \\
    Les autres huiles végétales ont une durée de conservation de \textbf{6 mois}.
\end{Remarque}