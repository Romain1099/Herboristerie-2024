\label{huileux}
\subsection{Macérat huileux}

\begin{Defi}[Macérat huileux]

    Un \voc{macérat huileux} ou \voc{macérat solaire} désigne une infusion de plantes dans un \voc{corps gras}. \\
    Ici, c'est l'effet de la \textbf{chaleur} qui permet la \voc{diffusion} des principes actifs.
\end{Defi}

\begin{multicols}{2}
    \boite{Préparation}{
        \begin{enumerate}
            \item Découper les plantes \textbf{séchées}.
            \item Les placer dans un bocal \textbf{à ras} et \textbf{sans tasser}.
            \item Couvrir d'huile \textbf{végétale} en veillant à \textbf{éliminer} les \textbf{bulles d'air} à l'aide d'une petite spatule.
            \item Laisser le bocal en le \textbf{couvrant} avec une \voc{étamine} pendant 24h dans un environnement \textbf{propre} et \voc{ensoleillé} ( ou chaud ). 
        \end{enumerate}
    }

    \boite{Macération}{
        \begin{itemize}[label=\mysquare]
            \item Refermer le pot en éliminant la \voc{condensation}. 
            \item Conserver \voc{au soleil} ou \voc{au chaud} pendant $\mathbf{28}$\textbf{ jours}.
            \item Passé cette période : filtrer les plantes et verser l'huile dans un bocal.
        \end{itemize}
    }
\end{multicols}

\boite{Conservation}{
    \begin{itemize}[label=\faPen]
        \item Se conserve \voc{à l'abri de l'humidité}.
        \item Le macérat huileux peut se conserver sur une période allant de 6 mois à 2 ans.% \monimage{0.1}{groupe/tetra.png}
    \end{itemize}
    
}
\boite{Conseils d'utilisation}{
    Il est conseillé de \voc{filtrer} le mélange avant d'utiliser l'huile. \\
    Selon l'huile choisie, on pourra consommer le macérat lors des repas. \\

    On peut les utiliser pour des bougies, des huiles de massages, pour soigner (interne/externe) ou pour utilisation cosmétique.
}

\begin{Remarque}[]%\monimage{0.05}{avec_sourire.png}
    \begin{itemize}[label=\faPen]
        \item Les huiles les plus \voc{stables} sont l'\voc{huile d'olive} et l'\voc{huile de tournesol} et permettent une conservation sur \textbf{2 ans}. \\
        \item Les autres huiles végétales ont une durée de conservation de \textbf{6 mois}.
        \item Les plantes doivent être séchées au préalable afin qu'il n'y ait \voc{pas d'insecte} et \textbf{pas d'humidiité}.
    \end{itemize}
\end{Remarque}