\subsection{Solvant vinaigre}

\begin{Defi}[Macération au vinaigre]
	
	Une \voc{macération au vinaigre} consiste à faire \voc{macérer} une plante dans du \voc{vinaigre}.\\

	C'est l'effet du \textbf{mouvement des plantes} chaque jour qui permet de diffuser les \voc{principes actifs}.\\

    L'utilisation du \textbf{solvant vinaigre} permet un compromis entre les propriétés des \textbf{teintures mères} et des \textbf{infusions}.\\
	
	

\end{Defi}

\begin{multicols}{2}
    \boite{Préparation :}{
		Pour préparer une macération au vinaigre, on respecte en général un \voc{dosage} de $40$g de \textbf{plantes fraîches} pour $1$L de \textbf{vinaigre}.\\
		\begin{itemize}[label = \faPen]
            \item Dans un bocal \voc{hermétique} et \voc{désinfecté}, remplir de \voc{plantes fraîches} à \textbf{ras}.
            \item \textbf{Couvrir} le mélange de vinaigre.\\On pourra utiliser du vinaigre de \voc{cidre de pommes}. 
            \item 
        \end{itemize}
	}

        
    \columnbreak 


    \boite{Macération :}{
        \begin{itemize}[label=\mysquare]
            \item \voc{Dynamiser} chaque jour pendant \textbf{28 jours} pour permettre la \voc{diffusion} des principes actifs. 
            \item Conserver \textbf{à l'abri de la lumière}.
        \end{itemize}
    }\\
    \hfill
	\includeprepa[0.35]{vinaigre_plantes_internet.jpg}{Vinaigre de plantes}{source : Internet}
\hfill



\end{multicols}

\begin{Remarque}
	\begin{itemize}[label = \faPen]
		\item \textbf{Couvrir} le mélange pour garder les \voc{principes volatiles}.
		\item L'eau ne doit pas être bouillante $\longrightarrow$ casse les molécules.
	\end{itemize}
\end{Remarque}