\subsection{Solvant vinaigre}

\begin{Defi}[Macération au vinaigre]
	
	\label{vinaigre}
    Une \voc{macération au vinaigre} consiste à faire \voc{macérer} une plante dans du \voc{vinaigre}.\\

	C'est l'effet du \textbf{mouvement des plantes} chaque jour qui permet de diffuser les \voc{principes actifs}.\\

    L'utilisation du \textbf{solvant vinaigre} permet un compromis entre les propriétés des \textbf{teintures mères} et des \textbf{infusions}.\\
	
	

\end{Defi}

\begin{multicols}{2}
    \boite{Préparation :}{
		Pour préparer une macération au vinaigre, on respecte en général un \voc{dosage} de $40$g de \textbf{plantes fraîches} pour $1$L de \textbf{vinaigre}.\\
		\begin{itemize}[label = \faPen]
            \item Dans un bocal \voc{hermétique} et \voc{désinfecté}, remplir de \voc{plantes fraîches} à \textbf{ras}.
            \item \textbf{Couvrir} le mélange de vinaigre.\\On pourra utiliser du vinaigre de \voc{cidre de pommes}.
            \item Noter sur une étiquette les informations essentielles de la préparation.\\Exemples : Nom des plantes, type de vinaigre, lieu de récolte, date... 
        \end{itemize}
	}

        
    \columnbreak 


    \boite{Macération :}{
        \begin{itemize}[label=\mysquare]
            \item \voc{Dynamiser} chaque jour pendant \textbf{28 jours} pour permettre la \voc{diffusion} des principes actifs. 
            \item Conserver \textbf{à l'abri de la lumière}.
        \end{itemize}
    }\\
    \hfill
	\includeprepa[0.35]{vinaigre_plantes_internet.jpg}{Vinaigre de plantes}{source : Internet}
\hfill



\end{multicols}

\begin{Remarque}
    Une macération au vinaigre permet d'extraire :
	\begin{multicols}{2}
        \begin{itemize}[label = \faPen]
		\item Certains \textbf{acides} comme par exemmple la \voc{vitamine C}.
		\item Des tanins.
		\item Des antioxydants.
		\item Bien que le vinaigre ne soit pas le meilleur solvant pour extraire les huiles essentielles, certains composants volatils ou semi-volatils peuvent être extraits en petite quantité.
		\item Les saponines peuvent être extraites ainsi.\\ Elles sont utiles pour récupérer les principes \voc{expectorants} et \voc{anti-inflammatoires}.
		\item Certains \voc{minéraux} notamment le \voc{calcium}, le \voc{magnésium} et le \voc{fer}.
        \end{itemize}
\end{Remarque}