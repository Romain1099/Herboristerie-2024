\label{baume}
\subsection{Baumes}

    \begin{Defi}[Baume]
        Pour obtenir un \voc{baume}, on utilise le mélange d'un \voc{corps gras} avec de la \voc{cire d'abeille} utilisée comme agent texturant.\\

        Généralement, la masse de cire d'abeille utilisée correspond à $\dfrac{3}{10}$ de la masse de corps gras. 
    \end{Defi}


    \newcommand{\macerat}{\voc{macérat huileux}}
\newcommand{\macerats}{\voc{macérats huileux}}

    \ficherecette
    {%titre recette
        Baume à lèvres - anti-inflammatoire
    }
    {%liste d'ingrédients commençant par \item directement.
        \item $50$g de \voc{macérat huileux} de \voc{calendula}.
        \item $10$g de \voc{macérat huileux} d'\voc{achillée}\\\textbf{millefeuille}.
        \item $18$g de \voc{cire d'abeille}
        \item $6$g de \voc{miel}.
        \item $3$ pulvérisations d'\voc{eau florale de rose}.
        \item $1$ goutte de \voc{propolis}.
    }
    {%liste du matériel commençant par \item directement.
        \item $1$ \frquote{\voc{cul de poule}} propre. 
        \item $1$ fouet. 
        \item $1$ pot \textbf{désinfecté}, \voc{étanche} et \voc{sec}.
        \item $1$ spatule. 
        \item $1$ casserole et de l'eau pour le bain marie. 
        \item Des plaques chauffantes. 
    }
    {%Métnodes et conseils de conservation
        \begin{itemize}[label=\faPen]
            \item Conservation courte $\approx 6$ mois.
            \item Dans un endroit sec, de préférence à l'abri de la lumière. 
        \end{itemize}
    }
    {%Métnodes et conseils d'utilisation    
        \begin{itemize}[label=\faPen]
            \item En application locale sur la zone irritée.
            \item La préparation est commestible.
            \item Utiliser un \textbf{ustensile propre} lors de l'utilisation pour prolonger la durée de conservation. 
        \end{itemize}
    }
    {%Remarques
        Le dosage de cire d'abeille correspond à $\dfrac{3}{10}$ de la masse de \macerat.\\
        Le dosage de miel correspond à $\dfrac{1}{10}$ de la masse de \macerat.\\
        La \textbf{texture} doit correspondre à celle du \frquote{baume du tigre}.
    }
    {%chemin de l'illustration dans le dossier 'préparations'
        baume.jpg
    }
    {%Titre donné à l'illustration dans le document latex
        Baume au Calendula
    }
    {%Légende donnée à l'illustration dans le document latex
        31/07/2024
    }
    \newpage
    \newcommandx{\teinture}{\voc{teinture mère}}

\ficherecette
{%titre recette
    Baume pour la lignée féminine
}
{%liste d'ingrédients commençant par \item directement.
    \item $25$g de \voc{millepertuis}.
    \item $35$g d'\voc{achillée millefeuille}.
    \item $11$ gouttes de \teinture d'aubépine.
    \item $6$ pulvérisations d'\voc{eau de rose}.
    \item $8$g de \voc{cire d'abeille}.
}
{%liste du matériel commençant par \item directement.
    \item $1$ \frquote{\voc{cul de poule}} propre. 
    \item $1$ fouet et $1$ spatule. 
    \item $1$ pot \textbf{désinfecté}, \voc{étanche} et \voc{sec}.
    \item $1$ casserole et de l'eau pour le bain marie. 
    \item Des plaques chauffantes.  
}
{%Métnodes de préparation
    \item Verser les macérats huileux dans un cul de poule. 
    \item Ajouter la cire d'abeille \textbf{émiétée} pour faciliter son incorporation durant la \textbf{chauffe}.
    \item Faire chauffer au \textbf{bain-marie} en \textbf{remuant}.
    \item Une fois la cire d'abeille incorporée, sortir du bain-marie et ajouter les ingrédients restants. 
    \item Continuer de mélanger jusqu'à l'apparition d'une \textbf{mayonnaise} sur les rebords de la préparation. \\
            Cela caractérise un refroidissement suffisant.
    \item Verser dans un bocal hermétique, sec, et propre. 
    \item Laisser sécher pendant \textbf{24h} puis refermer le bocal. 
}
{%Métnodes et conseils d'utilisation    
    \textbf{Conseils de conservation :}
    \begin{itemize}[label=\faPen]
        \item Conservation courte $\approx 6$ mois.
        \item Dans un endroit sec, de préférence à l'abri de la lumière. 
    \end{itemize}
    \textbf{Conseils d'utilisation :}
    \begin{itemize}[label=\faPen]
        \item En application sur le \textbf{point de chakra} du \voc{plexus solaire}.
        \item Application possible sur le \textbf{premier point de chakra}.
        \item Utiliser un \textbf{ustensile propre} lors de l'utilisation pour prolonger la durée de conservation. 
    \end{itemize}
}
{%Remarques
    Le dosage de cire d'abeille correspond à $\dfrac{2}{10}$ de la masse de \macerat.\\
    Le dosage de miel correspond à $\dfrac{1}{10}$ de la masse de \macerat.\\
    La \textbf{texture} doit être un peu plus fluide que celle du \frquote{baume du tigre}.
}
{%chemin de l'illustration dans le dossier 'préparations'
    baume/baume_dessus_2.jpg    
}
{%Titre donné à l'illustration dans le document latex
    Baume pour la lignée féminine
}
{%Légende donnée à l'illustration dans le document latex
    01/08/2024
}
\newpage
\label{onguent}
\subsection{Onguents}

    \begin{Defi}[Onguent]
        Pour obtenir un \voc{onguent}, on utilise le mélange d'un \voc{corps gras} avec de la \voc{cire d'abeille} utilisée comme agent texturant dans une proportion moindre part rapport au baume.\\

        Généralement, la masse de cire d'abeille utilisée correspond à $\dfrac{1}{10}$ de la masse de corps gras. 
    \end{Defi}

    \label{soleilcp}
\ficherecette
{%titre recette
    Crème après soleil
}
{%liste d'ingrédients commençant par \item directement.
    \item $25\%$ de \voc{macérat huileux} de \voc{calendula}
    \item $35\%$ de \voc{macérat huileux} de \voc{paquerette}
    \item $6\%$ de \voc{cire d'abeille}
    \item $6$g de \voc{miel}.
    \item $8$ pulvérisations d'\voc{eau florale de rose}.
    \item $1$ goutte de \voc{propolis}.
}
{%liste du matériel commençant par \item directement.
    \item $1$ \frquote{\voc{cul de poule}} propre. 
    \item $1$ fouet. 
    \item $1$ pot \textbf{désinfecté}, \voc{étanche} et \voc{sec}.
    \item $1$ spatule. 
    \item $1$ casserole et de l'eau pour le bain marie. 
    \item Des plaques chauffantes. 
}
{%Métnodes et conseils de conservation
    \begin{itemize}[label=\faPen]
        \item Conservation courte $\approx 6$ mois.
        \item Dans un endroit sec, de préférence à l'abri de la lumière. 
    \end{itemize}
}
{%Métnodes et conseils d'utilisation    
    \begin{itemize}[label=\faPen]
        \item En application locale sur la zone irritée.
        \item Utiliser un \textbf{ustensile propre} lors de l'utilisation pour prolonger la durée de conservation. 
    \end{itemize}
}
{%Remarques
    \textit{A priori} il n'y a pas de contre-indication pour être à nouveau au soleil après application contrairement à une préparation contenant du millepertuis.
}
{%chemin de l'illustration dans le dossier 'préparations'
    baume/pommades.jpg
}
{%Titre donné à l'illustration dans le document latex
    Crème au après solaire
}
{%Légende donnée à l'illustration dans le document latex
    Calendula - Paquerettes - 31/07/2024
}
    \newpage
    \label{hemorroides}
\ficherecette
{%titre recette
    Crème anti-hémorroïdes
}
{%liste d'ingrédients commençant par \item directement.
    \item $60$g de \voc{macérat huileux} de \voc{achillée millefeuille}
    \item $6$ gouttes de \voc{teinture mère} d'\voc{achillée millefeuille}
    \item $6 \text{ à }8$g de \voc{cire d'abeille}
    \item $6$g de \voc{miel}.
    \item $3$ pulv. d'\voc{eau florale de rose}.
}
{%liste du matériel commençant par \item directement.
    \item $1$ \frquote{\voc{cul de poule}} propre. 
    \item $1$ fouet. 
    \item $1$ pot \textbf{désinfecté}, \voc{étanche} et \voc{sec}.
    \item $1$ spatule. 
    \item $1$ casserole et de l'eau pour le bain marie. 
    \item Des plaques chauffantes. 
}
{%Métnodes de préparation
    \item Verser les macérats huileux dans un cul de poule. 
    \item Ajouter la cire d'abeille \textbf{émiétée} pour faciliter son incorporation durant la \textbf{chauffe}.
    \item Faire chauffer au \textbf{bain-marie} en \textbf{remuant}.
    \item Une fois la cire d'abeille incorporée, sortir du bain-marie et ajouter les ingrédients restants. 
    \item Continuer de mélanger jusqu'à l'apparition d'une \textbf{mayonnaise} sur les rebords de la préparation. \\
            Cela caractérise un refroidissement suffisant.
    \item Verser dans un bocal hermétique, sec, et propre. 
    \item Laisser sécher pendant \textbf{24h} puis refermer le bocal. 

}
{%Métnodes et conseils d'utilisation    
    \textbf{Conseils de conservation :}

    \begin{itemize}[label=\faPen]
        \item Conservation longue $\approx 2$ ans.
        \item Dans un endroit sec, de préférence à l'abri de la lumière. 
    \end{itemize}
    \textbf{Conseils d'utilisation :}

    \begin{itemize}[label=\faPen]
        %\item \voc{Diluer} dans de l'eau en respectant la \voc{posologie} ( voir \lien{teinture}{teintures mères} ).
        \item En application locale sur la zone irritée.
    \end{itemize}
}
{%Remarques
    Pour la teinture mère, \voc{diluer} dans de l'eau en respectant la \voc{posologie} ( voir \lien{teinture}{teintures mères} )
}
{%chemin de l'illustration dans le dossier 'préparations'
    downloads/tm_achillee.jpg
}
{%Titre donné à l'illustration dans le document latex
   Teinture mère d'Achillée millefeuille
}
{%Légende donnée à l'illustration dans le document latex
    source : internet
}
\newpage
\subsection{Huiles}
\newpage
\subsection{Tisanes}

    \label{melangeinsomnie}
\subsubsection{Mélanges contre l'insomnie}

\begin{multicols}{2}
    \boite{Recette 1 : }{
        \textbf{Ingrédients :}
        \begin{itemize}[label=\faPen]
            \item Aubépine $25$g - sommités fleuries
            \item Coquelicot $25$g  - pétales
            \item Passiflore $25$g - plante entière
            \item Saule blanc $25$g - écorce
            \item oranger amer $25$g - fleurs ou feuilles
        \end{itemize}

        \begin{enumerate}
            \item Prenez $5$g du mélange d'ingrédients
            \item Faites une infusion de $10$ minutes dans $250$mL d'eau.
            \item Buvez $1$ tasse à thé de cet infusé \textbf{le soir} au coucher.
        \end{enumerate}
    }

\columnbreak
    \boite{Recette 2 : }{
        \textbf{Ingrédients :}
        \begin{itemize}[label=\faPen]
            \item Valériane $60$g 
            \item Mélisse $60$g 
            \item Passiflore $60$g 
            \item Saule blanc $60$g 
        \end{itemize}

        \begin{enumerate}
            \item Prenez $5$g du mélange d'ingrédients
            \item Faites une infusion $>15$ minutes dans $250$mL d'eau.
            \item Buvez $1$ tasse à thé de cet infusé \textbf{le soir} au coucher.
        \end{enumerate}
    }
\end{multicols}

\begin{multicols}{2}
    \boite{Recette 3 : }{
        \textbf{Ingrédients :}
        \begin{itemize}[label=\faPen]
            \item Houblon $50$g 
            \item Valériane $50$g 
            \item Mélisse $50$g 
        \end{itemize}

        \begin{enumerate}
            \item Prenez $5$g du mélange d'ingrédients
            \item Faites une infusion de $10$ minutes dans $250$mL d'eau.
            \item Buvez $1$ tasse à thé de cet infusé \textbf{le soir} au coucher.
        \end{enumerate}
    }

\columnbreak
    \boite{Recette 4 : }{
        \textbf{Ingrédients :}
        \begin{itemize}[label=\faPen]
            \item Valériane $40$g 
            \item Aubépine $40$g 
            \item Passiflore $30$g 
            \item Verveine odorante $20$g 
            \item Tilleul
        \end{itemize}

        \begin{enumerate}
            \item Prenez $5$g du mélange d'ingrédients
            \item Faites une infusion $>15$ minutes dans $250$mL d'eau.
            \item Buvez $1$ tasse à thé de cet infusé \textbf{le soir} au coucher.
        \end{enumerate}
    }
\end{multicols}

\newpage
\subsection{Vinaigres}

    \ficherecette
{%titre recette
    Vinaigre d'origan
}
{%liste d'ingrédients commençant par \item directement.
    \item 40g d'origan
    \item 1L de \textbf{vinaigre de cidre}
}
{%liste du matériel commençant par \item directement.
    \item Un bocal adapté à la taille de la cueillette.
    \item Une spatule pour tasser les plantes.

}
{%Métnodes et conseils de conservation
    Conserver pendant $28$ jour à l'abri du soleil. \\
    \voc{Dynamiser} chaque jour pour permettre la \voc{diffusion} des minéraux dans le vinaigre.
}
{%Métnodes et conseils d'utilisation    
    \textbf{En cure :}\\
    $1$ cuillère à soupe diluée dans de l'eau chaude.\\
    La préparation est à consommer \textbf{chaque matin} durant \textbf{trois semaines}.\\

    \textbf{Après une cure :}\\
    Arrêter de consommer pendant \textbf{une semaine}. \\
    Reprendre ensuite en \textbf{changeant de plante}.
}
{%Remarques
    Voir les détails de préparation dans la partie \lien{vinaigre}{préparation des vinaigres}.\\

    Il est utile de bien \voc{sécher les plantes}.
}
{%chemin de l'illustration dans le dossier 'préparations'
    vinaigre_plantes_internet.jpg
}
{%Titre donné à l'illustration dans le document latex
    Vinaigre de plantes
}
{%Légende donnée à l'illustration dans le document latex
    source : internet
}
