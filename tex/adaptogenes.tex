\subsection{Plantes adaptogènes}

\begin{Defi}[Plante adaptogène]

    Le concept de \vocref{https://fr.wikipedia.org/wiki/Adaptog\%C3\%A8ne}{plantes adaptogènes} nous vient du Dr. \voc{Nicolai Lazarev}, un toxicologue russe, 
    qui cherchait à définir le type d'action de plantes comme le ginseng en 1947.\\

    De façon générale, une plante \voc{adaptogène} permet d'aider à \voc{gérer un stress} employé içi au sens large.\\

    Pour être considérée comme \voc{adaptogène}, une plante doit satisfaire un certain nombre de critères :
    \begin{enumerate}
        \item Être \textbf{non toxique}
        \item Déclencher une réponse \textbf{non-spécifique} du corps
        \item Déclencher une \textbf{action régulatrice} sur les \voc{processus physiologiques}, peu importe le sens du déséquilibre.
    \end{enumerate} 
    On peut les considérer comme des plantes intelligentes : peuvent équilibrer le niveau hormonal et protéger tout le corps.\\

    On considère qu'au bout de sept jours, une plante adaptogène doit faire de l'effet. 

    
\end{Defi}

\begin{Remarque}
    Equivalent aux \frquote{toniques supérieures} en médecine chinoise, qui est globale et holistique.\\
    Ces plantes \textbf{stimulent} le système nerveux / immunitaire et endocrinien.\\
    Ces plantes ont généralement un effet antioxydant, hépatoprotecteur, cardioprotecteur.\\
    En général, elle soutient les fonctions surrénales, ce qui contre les effets néfastes du stress.\\
    Elle active les cellules du corps pour accéder à plus d'énergie, elle débarrasse les cellules de leurs déchets métaboliques toxiques, 
    elle fournit un effet \vocref{https://fr.wikipedia.org/wiki/Anabolisme}{anabolique} 
    et aide le corps à utiliser mieux l'oxygène et accélère la régulation des biorythmes.\\

\end{Remarque}

\begin{Exemple}[Plantes adaptogènes]

    Les plantes adaptogènes les plus connues et utilisées sont :
    \begin{itemize}
        \item La \vocref{https://fr.wikipedia.org/wiki/Rhodiola_rosea}{Rhodiola} ou \voc{orpin rose}
        \item L'\vocref{https://fr.wikipedia.org/wiki/\%C3\%89leuth\%C3\%A9rocoque}{éleuthérocoque}
        \item Le \vocref{https://fr.wikipedia.org/wiki/Ginseng}{Ginseng rouge}
        \item La \vocref{https://fr.wikipedia.org/wiki/Schisandra}{Schisandra de Chine}
    \end{itemize}
    Les champignons adaptogènes les plus connus et utilisés sont : 
    \begin{itemize}[label = \bcfleur]
        \item Le \vocref{https://fr.wikipedia.org/wiki/Inonotus_obliquus}{Chaga} – Lonotus obliquus
        \item Le \vocref{https://fr.wikipedia.org/wiki/Cordyceps}{Cordyceps} – Cordyceps sinensis, vient du Tibet
        \item L'\vocref{https://fr.wikipedia.org/wiki/Ashwagandha}{Ashwagandha} – Vithania Somnifera
        \item Le \vocref{https://fr.wikipedia.org/wiki/Ganoderme_luisant}{Reishi}
    \end{itemize}
\end{Exemple}
