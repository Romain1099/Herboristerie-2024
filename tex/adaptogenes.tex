
\begin{Defi}[Plante adaptogène]

    Le concept de \vocref{https://fr.wikipedia.org/wiki/Adaptog\%C3\%A8ne}{plantes adaptogènes} nous vient du Dr. \voc{Nicolai Lazarev}, un toxicologue russe, 
    qui cherchait à définir le type d'action de plantes comme le ginseng en 1947.\\

    De façon générale, une plante \voc{adaptogène} permet d'aider à \voc{gérer un stress} employé içi au sens large.\\

    Pour être considérée comme \voc{adaptogène}, une plante doit satisfaire un certain nombre de critères :
    \begin{enumerate}
        \item Être \textbf{non toxique}
        \item Déclencher une réponse \textbf{non-spécifique} du corps
        \item Déclencher une \textbf{action régulatrice} sur les \voc{processus physiologiques}, peu importe le sens du déséquilibre.
    \end{enumerate} 
    On peut les considérer comme des plantes intelligentes : peuvent équilibrer le niveau hormonal et protéger tout le corps.\\

    On considère qu'au bout de sept jours, une plante adaptogène doit faire de l'effet. 
    
\end{Defi}

\begin{Remarque}
    Equivalent aux \frquote{toniques supérieures} en médecine chinoise ou au \frquote{rasayana} en \voc{ayurvéda}.\\
    Ces plantes \textbf{stimulent} le système nerveux / immunitaire et endocrinien.\\
    Ces plantes ont généralement un effet antioxydant, hépatoprotecteur, cardioprotecteur.\\
    En général, elle soutient les fonctions surrénales, ce qui contre les effets néfastes du stress.\\
    Elle active les cellules du corps pour accéder à plus d'énergie, elle débarrasse les cellules de leurs déchets métaboliques toxiques, 
    elle fournit un effet \vocref{https://fr.wikipedia.org/wiki/Anabolisme}{anabolique} 
    et aide le corps à utiliser mieux l'oxygène et accélère la régulation des biorythmes.\\

\end{Remarque}
\begin{Remarque}
    \begin{itemize}
        \item Attention au dosage !\\
            Ce qui peut provoquer du stress : l’envie de vivre différemment du quotidien et 
            difficultés à changer. \\
            Or la plante peut accentuer cela.
        \item Il existe une action cumulative de la plante adaptogène.
        \item Quelques retours de personnes ayant suivis une cure de plantes adaptogènes :\\
                \begin{multicols}{3}
                    \begin{itemize}
                        \item Cela me rend plus fort
                        \item Je ressens plus d'énergie au quotidien
                        \item Je me sens plus calme
                    \end{itemize}
                \end{multicols}
        \item Ces plantes sont utilisables aussi bien en prévention que lorsque la maladie est déclarée.
    \end{itemize}
\end{Remarque}
\begin{Exemple}[Plantes adaptogènes]

    Les plantes adaptogènes les plus connues et utilisées sont :
    \begin{itemize}
        \item La \vocref{https://fr.wikipedia.org/wiki/Rhodiola_rosea}{Rhodiola} ou \voc{orpin rose}
        \item L'\vocref{https://fr.wikipedia.org/wiki/\%C3\%89leuth\%C3\%A9rocoque}{éleuthérocoque}
        \item Le \vocref{https://fr.wikipedia.org/wiki/Ginseng}{Ginseng rouge}
        \item La \vocref{https://fr.wikipedia.org/wiki/Schisandra}{Schisandra de Chine}
    \end{itemize}
    Les champignons adaptogènes les plus connus et utilisés sont : 
    \begin{itemize}[label = \bcfleur]
        \item Le \vocref{https://fr.wikipedia.org/wiki/Inonotus_obliquus}{Chaga} – Lonotus obliquus
        \item Le \vocref{https://fr.wikipedia.org/wiki/Cordyceps}{Cordyceps} – Cordyceps sinensis, vient du Tibet
        \item L'\vocref{https://fr.wikipedia.org/wiki/Ashwagandha}{Ashwagandha} – Vithania Somnifera
        \item Le \vocref{https://fr.wikipedia.org/wiki/Ganoderme_luisant}{Reishi}
    \end{itemize}
\end{Exemple}
\newpage
\ficheidentiteplante
{Astragalus membranaceus}
%{%effet général
  %
% }
{%utilisation privilégiée
   L'Astragalus membranaceus est principalement utilisé pour renforcer le \voc{système immunitaire}, améliorer la \voc{fonction cardiaque} et \voc{réduire le stress}.\\
   Elle est connue pour ses propriétés \voc{immunostimulantes}, \voc{antioxydantes} et \voc{anti-inflammatoires}. 
 }
{%infos cueillette
    \begin{itemize}[label = \bcplume]
		\item \textbf{Parties utilisées :} Les racines de la plante sont principalement utilisées.
		\item \textbf{Période de cueillette :} Les racines sont généralement récoltées à l'automne.
		\item \textbf{Lieu de cueillette :} L'Astragalus membranaceus pousse principalement en Chine et en Mongolie.
	\end{itemize}
    }

{%sous quelle forme utiliser
   \begin{itemize}
		\item \textbf{Infusion :} Faire bouillir les racines dans de l'eau pendant 20-30 minutes.
		\item \textbf{Poudre :} Les racines peuvent être réduites en poudre et ajoutées à des smoothies ou des boissons.
		\item \textbf{Extrait :} Les extraits d'Astragalus membranaceus sont disponibles sous forme de teintures ou de capsules.
		\item \textbf{Complément alimentaire :} Disponible sous forme de capsules ou de comprimés.
\end{itemize}

}
{%supplément
\begin{multicols}{2}

    \boite{Usage interne :}{
        \begin{itemize}[label = \bctrefle]
            \item Augmentation de l'énergie
            \item Amélioration de la performance physique et mentale
            \item Renforcement du système immunitaire
        \end{itemize}
    }

    \columnbreak

    \boite{Usage externe}{
        \begin{itemize}[label = \bccrayon]
            \item Pas d'usage externe connu
        \end{itemize}
    }

\end{multicols}
}
{%image
    astragalus-membranaceus.jpg
}
{%titre photo
    Astragalus membranaceus
}
{%description photo : "lieu - date"
    Site \frquote{\vocref{https://beneficial-herbs.blogspot.com/2023/07/astragalus-astragalus-membranaceus.html}{en anglais sur les plantes médicinales}} - 12/08/2024 

}


\begin{Remarque}
    L'Astragalus membranaceus peut interagir avec certains médicaments, notamment les \voc{immunosuppresseurs}. 
\end{Remarque}
\newpage
\ficheidentiteplante
{Éleuthérocoque}
{%effet général
  L'Éleuthérocoque est connu pour ses propriétés \voc{adaptogènes}. Il est souvent utilisé pour \voc{augmenter l'endurance}, \voc{améliorer} la \voc{performance physique} et \voc{mentale}, et renforcer le \voc{système immunitaire}.
}
{%utilisation privilégiée
   L'Éleuthérocoque est principalement utilisé pour augmenter l'endurance, améliorer la performance physique et mentale, et renforcer le système immunitaire.
}
{%infos cueillette
   \begin{itemize}[label = \bcplume]
        \item \textbf{Parties utilisées :} Les \voc{racines} de la plante sont principalement utilisées.
        \item \textbf{Période de cueillette :} Les racines sont généralement récoltées à l'\voc{automne}.
        \item \textbf{Lieu de cueillette :} L'Éleuthérocoque pousse principalement en \voc{Sibérie} et en \voc{Chine}.
    \end{itemize}
}
{%sous quelle forme utiliser
    \begin{itemize}
        \item 
    \end{itemize}
}
{%supplément
    \begin{multicols}{2}

        \boite{Usage interne :}{
            \begin{itemize}[label = \bctrefle]
                \item \textbf{Infusion :} Faire bouillir les racines dans de l'eau pendant 20-30 minutes.
                \item \textbf{Poudre :} Les racines peuvent être réduites en poudre et ajoutées à des smoothies ou des boissons.
                \item \textbf{Extrait :} Les extraits d'Éleuthérocoque sont disponibles sous forme de teintures ou de capsules.
                \item \textbf{Complément alimentaire :} Disponible sous forme de capsules ou de comprimés.
            \end{itemize}
        }

        \columnbreak

        \boite{Usage externe}{
            \begin{itemize}[label = \bccrayon]
                \item Pas d'usage externe connu
            \end{itemize}
        }

    \end{multicols}
}
{%image
    eleutherocoque.jpg
}
{%titre photo
    Éleuthérocoque - Eleutherococcus senticosus
}
{%description photo : "lieu - date"
    Site de \vocref{https://nantes-naturopathe.fr/solution/eleutherocoque/}{Nantes-naturopathe} - 12/08/2024 
}

\begin{Remarque}
    L'Éleuthérocoque peut \voc{interagir} avec certains médicaments, notamment les \voc{anticoagulants}. 

\end{Remarque}
\newpage
\ficheidentiteplantelong
{Ginseng rouge}
{%effet général
    Il est souvent utilisé pour augmenter l'énergie, améliorer la performance physique et mentale, et renforcer le système immunitaire.\\

	Le Ginseng rouge est principalement utilisé pour augmenter l'énergie, améliorer la performance physique et mentale, et 
    renforcer le système immunitaire.
}
{%utilisation privilégiée
    Le \voc{Ginseng rouge} ou \textit{Panax ginseng Meyer} est connu pour ses propriétés adaptogènes, qui aident le corps à s'adapter au 
    stress.\\

}
{%infos cueillette
    \begin{itemize}[label = \bcplume]
		\item \textbf{Parties utilisées :} Les racines de la plante sont principalement utilisées.
		\item \textbf{Période de cueillette :} Les racines sont généralement récoltées à l'automne.
		\item \textbf{Lieu de cueillette :} Le Ginseng rouge pousse principalement en Corée et en Chine.
    \end{itemize}
}
{%sous quelle forme utiliser
    \begin{itemize}
		\item \textbf{Infusion :} Faire bouillir les racines dans de l'eau pendant 20-30 minutes.
		\item \textbf{Poudre :} Les racines peuvent être réduites en poudre et ajoutées à des smoothies ou des boissons.
		\item \textbf{Extrait :} Les extraits de Ginseng rouge sont disponibles sous forme de teintures ou de capsules.
		\item \textbf{Complément alimentaire :} Disponible sous forme de capsules ou de comprimés.
    \end{itemize}
}
{%supplément
    \begin{multicols}{2}

        \boite{Usage interne :}{
            \begin{itemize}[label = \bctrefle]
                \item Augmentation de l'énergie
                \item Amélioration de la performance physique et mentale
                \item Renforcement du système immunitaire
            \end{itemize}
        }

        \columnbreak

        \boite{Usage externe :}{
            \begin{itemize}[label = \bccrayon]
                \item Pas d'usage externe connu
            \end{itemize}
        }

    \end{multicols}
}
{%image
    ginseng-rouge.jpg
}
{%titre photo
    Ginseng rouge
}
{%description photo : "lieu - date"
    Site de \frquote{\vocref{https://aquasol.co.uk/red-ginseng}{tisanes et compléments alimentaires}} - 12/08/2024
}

\begin{Remarque}
    Le Ginseng rouge peut interagir avec certains médicaments, notamment les anticoagulants. \\
    Il est recommandé de consulter un professionnel de santé avant de commencer à utiliser le Ginseng rouge.
\end{Remarque}
\newpage
\ficheidentiteplante
{Reishi}
{%effet général
    Le \textbf{Reishi} (\textit{Ganoderma lucidum}), aussi appelé "plante de l'immortalité", est reconnu pour ses propriétés puissantes de soutien du système immunitaire, de destruction des cellules cancéreuses, et d'apport d'énergie et de solidité. En Asie, c'était la plante de l'empereur en raison de ses vertus associées à l'immortalité et au ralentissement du vieillissement. C'est aussi une plante de l'intuition et de la sagesse, agissant sur le mental, la mémoire, et l'évolution personnelle.
}
{%utilisation privilégiée
    Le Reishi est particulièrement utilisé pour soutenir l'immunité, en augmentant les cellules qui combattent les infections et en modulant le système immunitaire. Il est aussi privilégié pour réduire les effets secondaires de la chimiothérapie. Au Québec, les patients arrêtent sa consommation deux jours avant la chimiothérapie, puis reprennent après le traitement pour minimiser les effets secondaires. Le Reishi est également bénéfique pour les infections virales longues, comme la mononucléose, et pour les personnes ayant un système immunitaire trop faible ou trop actif.
}
{%infos cueillette
    \begin{itemize}[label = \bcplume]
        \item Le Reishi est principalement récolté pour ses parties ligneuses, le plus souvent sous forme de champignon séché.
        \item La cueillette se fait généralement à maturité, lorsque le champignon est bien développé et riche en polysaccharides.
    \end{itemize}
}
{%sous quelle forme utiliser
    \begin{itemize}
        \item \textbf{Teinture mère :} Préparée à partir du Reishi séché, elle est utilisée pour ses effets immunomodulateurs.
        \item \textbf{Décoction :} Utilisée pour extraire les composés actifs du Reishi, notamment les polysaccharides, afin de renforcer le système immunitaire et protéger le foie.
    \end{itemize}
}
{%supplément
\begin{multicols}{2}

    \boite{Usage interne :}{
        \begin{itemize}[label = \bctrefle]
            \item Le Reishi est pris en interne sous forme de teinture mère ou de décoction pour détruire les cellules cancéreuses, soutenir l'immunité, et réduire les effets secondaires de la chimiothérapie. Il est également utilisé pour renforcer l'énergie vitale et la solidité du corps.
        \end{itemize}
    }

    \columnbreak

    \boite{Usage externe}{
        \begin{itemize}[label = \bccrayon]
            \item Le Reishi est rarement utilisé en usage externe, mais ses extraits peuvent être appliqués pour protéger la peau ou pour des préparations destinées à la guérison spirituelle.
        \end{itemize}
    }

\end{multicols}
}
{%remarques
    Le Reishi, bien que très bénéfique, doit être utilisé avec prudence. Chez certaines personnes, en particulier celles ayant des traumas non résolus, sa consommation peut faire remonter des émotions ou souvenirs enfouis. Il est important d'être conscient de cet effet potentiel avant de l'intégrer dans un régime thérapeutique.
}
{%image
    reishi.jpg
}
{%titre photo
    Reishi
}
{%description photo : "lieu - date"
    Site de \frquote{} - 12/08/2024 
}
\newpage
\ficheidentiteplante
{Schisandra de chine - Shisandra sinensis}
{%effet général
   }
{%utilisation privilégiée
    Le Schisandra chinensis est connu pour ses propriétés adaptogènes, qui aident le corps à s'adapter au stress. Il est souvent utilisé pour améliorer la performance physique et mentale, renforcer le système immunitaire et améliorer la fonction hépatique.
	Le Schisandra chinensis est principalement utilisé pour améliorer la performance physique et mentale, renforcer le système immunitaire et améliorer la fonction hépatique.
 }
{%infos cueillette
    \begin{itemize}[label = \bcplume]
		\item \textbf{Parties utilisées :} Les baies de la plante sont principalement utilisées.
		\item \textbf{Période de cueillette :} Les baies sont généralement récoltées à l'automne.
		\item \textbf{Lieu de cueillette :} Le Schisandra chinensis pousse principalement en Chine et en Russie.
	\end{itemize}
    }

{%sous quelle forme utiliser
	\begin{itemize}
		\item \textbf{Infusion :} Faire bouillir les baies dans de l'eau pendant 20-30 minutes.
		\item \textbf{Poudre :} Les baies peuvent être réduites en poudre et ajoutées à des smoothies ou des boissons.
		\item \textbf{Extrait :} Les extraits de Schisandra chinensis sont disponibles sous forme de teintures ou de capsules.
		\item \textbf{Complément alimentaire :} Disponible sous forme de capsules ou de comprimés.
	\end{itemize}
}

{%supplément
\begin{multicols}{2}

    \boite{Usage interne :}{
        \begin{itemize}[label = \bctrefle]
            \item Amélioration de la performance physique et mentale
			\item Renforcement du système immunitaire
			\item Amélioration de la fonction hépatique
        \end{itemize}
    }

    \columnbreak

    \boite{Usage externe}{
        \begin{itemize}[label = \bccrayon]
            \item Pas d'usage externe connu
        \end{itemize}
    }

\end{multicols}
}

{%remarques
  Le Schisandra chinensis peut interagir avec certains médicaments, notamment les anticoagulants. Il est recommandé de consulter un professionnel de santé avant de commencer à utiliser le Schisandra chinensis.
 }

{%image
    schisandra-de-chine.jpg
}

{%titre photo
    Schisandra de chine
}

{%description photo : "lieu - date"
   Site de \vocref{https://laforetcomestible.org/wp-content/uploads/2020/04/schisandra-chinensis-wuweizi.jpg}{\frquote{site de la forêt comestible} - 12/08/2024 
}
}